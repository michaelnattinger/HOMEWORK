% !TEX TS-program = pdflatex
% !TEX encoding = UTF-8 Unicode

% This is a simple template for a LaTeX document using the "article" class.
% See "book", "report", "letter" for other types of document.

\documentclass[11pt]{article} % use larger type; default would be 10pt

\usepackage[utf8]{inputenc} % set input encoding (not needed with XeLaTeX)

%%% Examples of Article customizations
% These packages are optional, depending whether you want the features they provide.
% See the LaTeX Companion or other references for full information.

%%% PAGE DIMENSIONS
\usepackage{geometry} % to change the page dimensions
\geometry{a4paper} % or letterpaper (US) or a5paper or....
% \geometry{margin=2in} % for example, change the margins to 2 inches all round
% \geometry{landscape} % set up the page for landscape
%   read geometry.pdf for detailed page layout information

\usepackage{graphicx} % support the \includegraphics command and options

% \usepackage[parfill]{parskip} % Activate to begin paragraphs with an empty line rather than an indent
\usepackage{amssymb}
\usepackage{amsmath}
%%% PACKAGES
\usepackage{booktabs} % for much better looking tables
\usepackage{array} % for better arrays (eg matrices) in maths
\usepackage{paralist} % very flexible & customisable lists (eg. enumerate/itemize, etc.)
\usepackage{verbatim} % adds environment for commenting out blocks of text & for better verbatim
\usepackage{subfig} % make it possible to include more than one captioned figure/table in a single float
% These packages are all incorporated in the memoir class to one degree or another...

%%% HEADERS & FOOTERS
\usepackage{fancyhdr} % This should be set AFTER setting up the page geometry
\pagestyle{fancy} % options: empty , plain , fancy
\renewcommand{\headrulewidth}{0pt} % customise the layout...
\lhead{}\chead{}\rhead{}
\lfoot{}\cfoot{\thepage}\rfoot{}

%%% SECTION TITLE APPEARANCE
\usepackage{sectsty}
\allsectionsfont{\sffamily\mdseries\upshape} % (See the fntguide.pdf for font help)
% (This matches ConTeXt defaults)

%%% ToC (table of contents) APPEARANCE
\usepackage[nottoc,notlof,notlot]{tocbibind} % Put the bibliography in the ToC
\usepackage[titles,subfigure]{tocloft} % Alter the style of the Table of Contents

\renewcommand{\cftsecfont}{\rmfamily\mdseries\upshape}
\renewcommand{\cftsecpagefont}{\rmfamily\mdseries\upshape} % No bold!
\DeclareMathOperator*{\argmax}{arg\,max}
\DeclareMathOperator*{\argmin}{arg\,min}

\newcount\colveccount
\newcommand*\colvec[1]{
        \global\colveccount#1
        \begin{pmatrix}
        \colvecnext
}
\def\colvecnext#1{
        #1
        \global\advance\colveccount-1
        \ifnum\colveccount>0
                \\
                \expandafter\colvecnext
        \else
                \end{pmatrix}
        \fi
}

\newcommand{\norm}[1]{\left\lVert#1\right\rVert}

\title{Econometrics HW3}
\author{Michael B. Nattinger\footnote{I worked on this assignment with my study group: Alex von Hafften, Andrew Smith, and Ryan Mather. I have also discussed problem(s) with Emily Case, Sarah Bass, and Danny Edgel.}}

\begin{document}
\maketitle

\section{7.28}
\subsection{Part A}
\begin{center}
\begin{tabular}{lllll}
& Edu & Exp & Exp\^{}2/100 & Constant \\ 
\hline 
Coefficient & 0.14431 & 0.042633 & -0.095056 & 0.53089 \\ 
Robust SE & 0.011726 & 0.012422 & 0.033796 & 0.20005 \\ 
\hline 
\end{tabular}
\end{center}
\subsection{Part B}
The derivative of log wage with respect to education is $\beta_1$ and the derivative of log wage with respect to experience is $\beta_2 + \beta_3 exp/50$ so $\theta = \frac{\beta_1}{\beta_2 + \beta_3 exp/50}$. Therefore, for 10 experience, our estimate implied by our regressions is the following:

\begin{align}
\hat{\theta} &= \frac{\hat{\beta_1}}{\hat{\beta_2} + \hat{\beta}_3 exp/50}\\
&=  \frac{0.1443}{0.0426 - 0.0951 (10)/50}\\
&= 6.109
\end{align}

\subsection{Part C}
We can find the asymptotic standard error as the square root of the asymptotic variance of the $\hat{\theta}$ estimator, which we can calculate through the delta method:

\begin{align*}
s(\hat{\theta}) &= \sqrt{g'(\beta)'V g'(\beta) },
\end{align*}
where V is the asymptotic covariance matrix of the non-intercept coefficients, and $g(\beta) = \frac{\hat{\beta_1}}{\hat{\beta_2} + \hat{\beta}_3 exp/50}$. Then,
\begin{align*}
g'(\beta) &= \colvec{3}{\frac{1}{\beta_2 + \beta_3 exp/50}}{\frac{-\beta_1}{(\beta_2 + \beta_3 exp/50)^2}}{\frac{-\beta_1 exp/50}{(\beta_2 + \beta_3 exp/50)^2}}
\end{align*}

We can calculate an estimate for $s(\hat{\theta})$ by plugging in OLS estimates of $\beta$ and our robust standard error matrix we used in Part A. Our 90\% CI is $[\hat{\theta} - 1.645s(\hat{\theta}),\hat{\theta} + 1.645s(\hat{\theta})]$.

\subsection{Part D}
 Our computed $\hat{\theta}$, $s(\hat{\theta})$, and confidence interval are the following:

\begin{align*}
\hat{\theta} &= 6.109 \\
s(\hat{\theta}) &= 1.6178 \\
CI &= [4.4912,7.7269]
\end{align*}.

\section{8.1}
Let $\beta = [\beta_1,\beta_2]$ be the CLS estimator of $Y = X_1'\beta_1 + X_2' \beta_2 +e$ subject to the constraint that $\beta_2 = 0.$ From definition (8.3),

\begin{align*}
\beta &= \argmin_{\beta_2 = 0}  (Y-X_1\beta_1 -X_2\beta_2)'(Y-X_1\beta_1 -X_2\beta_2)\\
\Rightarrow \mathcal{L} &= (Y-X_1\beta_1 -X_2\beta_2)'(Y-X_1\beta_1 -X_2\beta_2) + \lambda'(\beta_2 - 0)\\
\Rightarrow 0&= -2X_1'(Y-X_1\beta_1 - X_2 \beta_2)\\
\Rightarrow X_1'Y &= (X_1'X_1)\beta_1 \\
\Rightarrow \beta_1 &= (X_1'X_1)^{-1}X_1'Y.
\end{align*}

\section{8.3}
%\begin{align*} % whoops solved the wrong problem on accident
%\beta &= \argmin_{\beta_2 = c}  (Y-X_1\beta_1 -X_2\beta_2)'(Y-X_1\beta_1 -X_2\beta_2)\\
%\Rightarrow \mathcal{L} &= (Y-X_1\beta_1 -X_2\beta_2)'(Y-X_1\beta_1 -X_2\beta_2) + \lambda'(\beta_2 - c)\\
%\Rightarrow 0&= -2X_1'(Y-X_1\beta_1 - X_2 \beta_2)\\
%\Rightarrow X_1'X_1\beta_1 &= X_1'(Y - X_2c)\\
%\Rightarrow \beta_1 &= (X_1'X_1)^{-1}X_1'(Y - X_2c)
%\end{align*}
\begin{align*}
\beta &= \argmin_{\beta_1 = -\beta_2}  (Y-X_1\beta_1 -X_2\beta_2)'(Y-X_1\beta_1 -X_2\beta_2)\\
\Rightarrow \mathcal{L} &= (Y-X_1\beta_1 -X_2\beta_2)'(Y-X_1\beta_1 -X_2\beta_2) + \lambda'(\beta_2 + \beta_1)\\
\Rightarrow 0&= -2X_1'(Y-X_1\beta_1 - X_2 \beta_2)+\lambda\\
\Rightarrow 0&= -2X_2'(Y-X_1\beta_1 - X_2 \beta_2)+\lambda\\
\Rightarrow 0&= (X_1 - X_2)'(Y-X_1\beta_1 + X_2 \beta_1)\\
\Rightarrow \beta_1 &= -\beta_2 =( (X_1 - X_2)'(X_1 - X_2))^{-1}(X_1 - X_2)'Y
\end{align*}
\section{8.4(a)}
Let $Z = X$
\begin{align*}
\alpha &= \argmin_{\beta = 0} (Y-X\beta - \alpha)'(Y-X\beta - \alpha)\\
\Rightarrow \mathcal{L} &=  (Y-X\beta - \alpha)'(Y-X\beta - \alpha) + \lambda'(\beta)\\
\Rightarrow 0&=-\vec{1}(Y-X\beta - \alpha) \\
\Rightarrow \alpha &= \frac{1}{n}\vec{1}'Y = \frac{1}{n}\sum_iY_i
\end{align*}
\section{8.22}
\subsection{Part A}
\begin{align*}
\tilde{\beta} &= \argmin_{2\beta_2 = \beta_1}  (Y-X_1\beta_1 -X_2\beta_2)'(Y-X_1\beta_1 -X_2\beta_2)\\
\Rightarrow \mathcal{L} &= (Y-X_1\beta_1 -X_2\beta_2)'(Y-X_1\beta_1 -X_2\beta_2) + \lambda'(2\beta_2 - \beta_1)\\
\Rightarrow 0&= -2X_1'(Y-X_1\beta_1 - X_2 \beta_2)+\lambda\\
\Rightarrow 0&= -2X_2'(Y-X_1\beta_1 - X_2 \beta_2)+2\lambda\\
\Rightarrow 0&= (2X_1 + X_2)'(Y-X_12\beta_2 - X_2 \beta_2)\\
\Rightarrow \tilde{\beta}_2&= ((2X_1 +X_2)'(2X_1 +X_2))^{-1}(2X_1 +X_2)'Y\\
\Rightarrow \tilde{\beta}_1&= 2\tilde{\beta}_2
\end{align*}
\subsection{Part B}
\begin{align*}
\sqrt{n}(\tilde{\beta}_2 - \beta_2) &= 2\sqrt{n}((2X_1 +X_2)'(2X_1 +X_2))^{-1}(2X_1 +X_2)'e\\
&=2 (\frac{1}{n}\sum_i (2X_{1,i} +X_{2,i})^{2})^{-1} \frac{1}{\sqrt{n}}\sum_i (2X_{1,i} +X_{2,i})e_i\\
&\Rightarrow N\left(0,\frac{E[ (2X_{1,i} +X_{2,i})^2e_i^2]}{E[(2X_{1,i} +X_{2,i})^{2}]^2} \right)
\end{align*}
\section{9.1}

\section{9.2}

\section{9.4}

\section{9.7}
\end{document}
