% !TEX TS-program = pdflatex
% !TEX encoding = UTF-8 Unicode

% This is a simple template for a LaTeX document using the "article" class.
% See "book", "report", "letter" for other types of document.

\documentclass[11pt]{article} % use larger type; default would be 10pt

\usepackage[utf8]{inputenc} % set input encoding (not needed with XeLaTeX)

%%% Examples of Article customizations
% These packages are optional, depending whether you want the features they provide.
% See the LaTeX Companion or other references for full information.

%%% PAGE DIMENSIONS
\usepackage{geometry} % to change the page dimensions
\geometry{a4paper} % or letterpaper (US) or a5paper or....
% \geometry{margin=2in} % for example, change the margins to 2 inches all round
% \geometry{landscape} % set up the page for landscape
%   read geometry.pdf for detailed page layout information

\usepackage{graphicx} % support the \includegraphics command and options

% \usepackage[parfill]{parskip} % Activate to begin paragraphs with an empty line rather than an indent
\usepackage{amssymb}
\usepackage{amsmath}
%%% PACKAGES
\usepackage{booktabs} % for much better looking tables
\usepackage{array} % for better arrays (eg matrices) in maths
\usepackage{paralist} % very flexible & customisable lists (eg. enumerate/itemize, etc.)
\usepackage{verbatim} % adds environment for commenting out blocks of text & for better verbatim
\usepackage{subfig} % make it possible to include more than one captioned figure/table in a single float
% These packages are all incorporated in the memoir class to one degree or another...

%%% HEADERS & FOOTERS
\usepackage{fancyhdr} % This should be set AFTER setting up the page geometry
\pagestyle{fancy} % options: empty , plain , fancy
\renewcommand{\headrulewidth}{0pt} % customise the layout...
\lhead{}\chead{}\rhead{}
\lfoot{}\cfoot{\thepage}\rfoot{}

%%% SECTION TITLE APPEARANCE
\usepackage{sectsty}
\allsectionsfont{\sffamily\mdseries\upshape} % (See the fntguide.pdf for font help)
% (This matches ConTeXt defaults)

%%% ToC (table of contents) APPEARANCE
\usepackage[nottoc,notlof,notlot]{tocbibind} % Put the bibliography in the ToC
\usepackage[titles,subfigure]{tocloft} % Alter the style of the Table of Contents

\renewcommand{\cftsecfont}{\rmfamily\mdseries\upshape}
\renewcommand{\cftsecpagefont}{\rmfamily\mdseries\upshape} % No bold!
\DeclareMathOperator*{\argmax}{arg\,max}
\DeclareMathOperator*{\argmin}{arg\,min}

\newcount\colveccount
\newcommand*\colvec[1]{
        \global\colveccount#1
        \begin{pmatrix}
        \colvecnext
}
\def\colvecnext#1{
        #1
        \global\advance\colveccount-1
        \ifnum\colveccount>0
                \\
                \expandafter\colvecnext
        \else
                \end{pmatrix}
        \fi
}

\newcommand{\norm}[1]{\left\lVert#1\right\rVert}

\title{Econometrics HW3}
\author{Michael B. Nattinger\footnote{I worked on this assignment with my study group: Alex von Hafften, Andrew Smith, and Ryan Mather. I have also discussed problem(s) with Emily Case, Sarah Bass, and Danny Edgel.}}

\begin{document}
\maketitle

\section{3.24}
\begin{center}
\begin{tabular}{ll}
& beta \\ 
\hline 
education & 0.14431 \\ 
experience & 0.042633 \\ 
experience\^{}2/100 & -0.095056 \\ 
constant & 0.53089 \\ 
\hline 
\end{tabular}\\
\begin{tabular}{ll}
& results \\ 
\hline 
R\^{}2 & 0.38932 \\ 
SSE & 82.505 \\ 
\hline 
\end{tabular}\\
\begin{tabular}{ll}
& reestimate \\ 
\hline 
coefficient estimate & 0.14431 \\ 
R\^{}2 & 0.36874 \\ 
SSE & 82.505 \\ 
\hline 
\end{tabular}
\end{center}
From the above tables, we see that we have matched the ols coefficient from equation (3.50). The $R^2$ and SSE are listed as well in the second table. In the third table, we see our re-estimated coefficient is the same as in the original regression; however, the $R^2$ is lower in the re-estimated regression as part of the informational content was already regressed out of the response variable in the first stage of the two-stage regression. The SSE are identical, however, due to the residuals from the original regression being identical to the residuals from the second stage of the re-estimated regression.
\section{3.25}
\begin{center}
\begin{tabular}{ll}
& sums \\ 
\hline 
a & 4.4187e-14 \\ 
b & -7.2209e-13 \\ 
c & -2.0606e-13 \\ 
d & 133.1331 \\ 
e & 1.5575e-11 \\ 
f & -8.249e-14 \\ 
g & 82.505 \\ 
\hline 
\end{tabular}
\end{center}
The above table yields the relevant sums. Note that $a,b,c,e$ are 0 (to computational accuracy) reflecting the fact that these sums are the inner product of one of the columns of X and the residual estimates. These inner products are 0 by construction. $f$ is also 0 by construction for similar reasons. $d,g$ are not forced to be 0 by construction, and in this case they are clearly nonzero.

\section{7.2}
\begin{align*}
\frac{1}{n}\sum_{i=1}^n X_i X_i' &\rightarrow_p E[X_iX_i']\\
%\frac{1}{n}\sum_{i=1}^n X_i Y_i &\rightarrow_p E[X_iY_i]\\
\frac{1}{n}\lambda I_k &\rightarrow_p 0\\
\hat{\beta} &= (X'X + \lambda I_k)^{-1}X'Y \\
&= (X'X + \lambda I_k)^{-1}X'(X\beta + \epsilon) \\
&= (X'X + \lambda I_k)^{-1}X'X\beta + (X'X + \lambda I_k)^{-1}X'\epsilon\\
&\rightarrow_p \left( E[X_iX_i'] + 0 \right)^{-1} E[X_iX_i']\beta + (E[X_iX_i']+0)^{-1}E[X_i\epsilon] \\
&= \left( E[X_iX_i'] \right)^{-1} E[X_iX_i']\beta + \left( E[X_iX_i'] \right)^{-1} 0\\
&= \beta
\end{align*}
Thus, $\hat{\beta}$ is consistent for $\beta$.
\section{7.3}
\begin{align*}
\frac{1}{n}\lambda I_k = \frac{1}{n}cn I_k &\rightarrow_p c I_k\\
\Rightarrow \hat{\beta} &\rightarrow_p \left( E[X_iX_i'] + c I_k \right)^{-1} E[X_iX_i']\beta + (E[X_iX_i']+CI_k)^{-1}E[X_i\epsilon] \\
&= \left( E[X_iX_i'] + c I_k \right)^{-1} E[X_iX_i']\beta 
\end{align*}
So, in this case the estimator is not consistent as $ \left( E[X_iX_i'] + c I_k \right)^{-1} E[X_iX_i'] \neq I_k$.
\section{7.4}
\begin{enumerate}
\item $E[X_1] = 1/2(1) + 1/2(-1) = 0$
\item $E[X_1]^2 = 1/2(1) + 1/2(1) = 1$
\item $E[X_1X_2] = 3/4(1) + 1/4(-1) = 1/2$
\item $E[e^2] = (5/4)(3/4) + (1/4)(1/4) = 1$
\item $E[X_1^2 e^2] = (3/4)((1)(5/4)) + (1/4)((1)(1/4)) = 1$
\item $E[X_1X_2 e^2] = (3/4)((1)(5/4)) + (1/4)((-1)(1/4)) = 7/8$
\end{enumerate}
\section{7.8}
We know from (7.18) that $\hat{\sigma}^2 \rightarrow_p \sigma^2$. Moreover, 
%\begin{align*}
%\hat{\sigma}^2 &= \frac{1}{n}\hat{\epsilon}'\hat{\epsilon}\\
% &= \frac{1}{n}(Y - X\hat{\beta})'(Y - X\hat{\beta})\\
%&= \frac{1}{n}(Y'Y - \hat{\beta}'X'Y - Y'X\hat{\beta} + \hat{\beta}'X'X\hat{\beta})\\
%\Rightarrow \frac{\partial \hat{\sigma}^2}{\partial \hat{\beta}} &= \frac{1}{n}(2X'X\hat{\beta} - X'Y - Y'X)
%\end{align*}
\begin{align*}
\sqrt{n}(\hat{\sigma}^2 - \sigma^2) &= \sqrt{n}\left(\frac{1}{n}\sum_{i=1}^n \hat{\epsilon}_i^2 - \sigma^2\right) \\
&= \sqrt{n}\left(\frac{1}{n}\sum_{i=1}^n (\epsilon_i - x_i' (\hat{\beta}- \beta) )^2 - \sigma^2\right) \\
&=  \sqrt{n}\left(\frac{1}{n}\sum_{i=1}^n \epsilon_i^2 - \sigma^2\right) - 2\left(\frac{1}{n}\sum_{i=1}^n\epsilon_ix_i' \right)\sqrt{n}(\hat{\beta} - \beta) + \sqrt{n} (\hat{\beta} - \beta)'\left(\frac{1}{n}\sum_{i=1}^nx_i x_i' \right)(\hat{\beta} - \beta)\\
&= \sqrt{n}\left(\frac{1}{n}\sum_{i=1}^n \epsilon_i^2 - \sigma^2\right) - 2o_p(1) O_p(1) + O_p(1) O_p(1) o_p(1)\\
&\rightarrow_d  N(0,V),
\end{align*}
where $V = Var(\epsilon_i^2) = E(\epsilon_i^4) - \sigma^4$. Note that we have implicitly assumed that the fourth moment of $\epsilon$ exists.
\section{7.9a}
The first estimator, $\hat{\beta}$ is the univariate version of OLS. We know that this is therefore a consistent estimator. It is less immediate that $\tilde{\beta}$ is consistent, but we will show below that this is the case.

\begin{align*}
\tilde{\beta} &= \frac{1}{n}\sum_{i=1}^n \frac{Y_i}{X_i} = \frac{1}{n}\sum_{i=1}^n \frac{X_i\beta + e_i}{X_i}\\
&= \frac{1}{n}\sum_{i=1}^n \beta + \frac{e_i}{X_i} =\beta + \frac{1}{n}\sum_{i=1}^n \frac{e_i}{X_i}\\
&\rightarrow_p \beta + E\left[ \frac{e_i}{X_i}\right] =  \beta + E\left[ \frac{E[e_i|X_i]}{X_i}\right]\\
&= \beta
\end{align*}
Therefore, $\tilde{\beta}$ is also a consistent estimator of $\beta$.
\section{7.10}
\subsection{Point forecast}
Let $\hat{Y}_{n+1} = x'\hat{\beta}$. We will show that this estimator of $Y_{n+1}$ yields, in expectation conditional on $X,x$, the expectation of $Y_{n+1}$ conditional on $x$.
\begin{align*}
\hat{Y}_{n+1} &= x'\hat{\beta} = x'((X'X)^{-1}X'Y)\\
&=  x'(X'X)^{-1}X'(X\beta + e)\\
&= x'\beta + x'(X'X)^{-1}X'e.\\
E[\hat{Y}_{n+1}|X,x] &= E[x'\beta + x'(X'X)^{-1}X'e|X,x]\\
&= x'\beta + E[ x'(X'X)^{-1}X'E[e|X]|X,x]\\
&= x'\beta\\
&= E[Y_{n+1}|x]
\end{align*}
\subsection{Variance estimator}
\begin{align*}
Var(\hat{Y}_{n+1}) &= E[\hat{e}_{n+1}^2] \\
&= E[(e_{n+1} - x'(\hat{\beta} - \beta))^2]\\
&= E[e_{n+1}^2] - 2E[e_{n+1}x'(\hat{\beta} - \beta)] + E[x'(\hat{\beta} - \beta)(\hat{\beta} - \beta)'x]\\
&= \sigma^2 + x'V_{\hat{\beta}}x
\end{align*}
These are not known, however. Yet, we do have estimates of these quantities. Therefore,
\begin{align*}
\hat{Var}(\hat{Y}_{n+1})  &= \hat{\sigma}^2 + x'\hat{V}_{\beta} x
\end{align*}
is an estimator of the variance of our forecast.
\section{7.13}
We propose $\hat{\gamma} = \frac{1}{n}\sum_{i=1}^n X_i/Y_i$. Naturally, this leads to an estimator for $\theta: \hat{\theta} = 1/\hat{\gamma}$. $Var(\hat{\gamma}) = \frac{1}{n^2} \sum_{i=1}^{n}Var\left( \frac{X_i}{Y_i} \right) = \frac{1}{n^2} \sum_{i=1}^{n}Var\left( \gamma + \frac{u_i}{Y_i} \right) = \frac{1}{n} \left( \frac{Var(u_i)}{Var(Y_i)} \right):= \frac{1}{n}V$. Therefore, $\sqrt{n}(\hat{\gamma} - \gamma) \rightarrow_d N(0,V).$ Thus, we can apply the delta method and find that $\sqrt{n}(\hat{\theta} - \theta) \rightarrow_d N(0,W)$ where $W = \frac{V}{\gamma^2} = \theta^2V$.

The asymptotic standard error for $\hat{\theta}$ is $\sqrt{W} = \theta \sqrt{V} = \theta \sqrt{\frac{Var(u_i)}{Var(Y_i)}}.$

\section{7.14}
We can retrieve OLS estimates of $\beta_1,\beta_2$ ($\hat{\beta}_1,\hat{\beta}_2$) and then define $\hat{\theta} = \hat{\beta}_1 \hat{\beta}_2$. Next, we know the asymptotic distribution for OLS: $\sqrt{n}(\hat{\beta} - \beta) \rightarrow_d N(0,V_\beta)$ where $V_{\beta} = E[x_ix_i']^{-1} E[\epsilon_i^2x_ix_i']  E[x_ix_i']^{-1} $Then, we can apply the delta method and find:

\begin{align*}
\sqrt{n}(\hat{\theta} - \theta) \rightarrow_d N(0,V),
\end{align*}
where $V = [\beta_2 \beta_1]V_{\beta} [\beta_2 \beta_1 ]'$. %Should there be a $1/4$ here?

To run a test, we would estimate $V: \hat{V} = [\hat{\beta}_2 \hat{\beta}_1]\hat{V}_{\beta}[\hat{\beta}_2 \hat{\beta}_1]'$ and calculate the 95 percent CI as $\left[\hat{\theta} - 1.96\sqrt{\hat{V}/n},\hat{\theta} + 1.96\sqrt{\hat{V}/n}\right]$.

\section{7.15}

\begin{align*}
\hat{\beta} &= \frac{\sum_{i=1}^n X_i^3 Y_i}{\sum_{i=1}^nX_i^4} \\
&= \frac{\sum_{i=1}^n X_i^3 (X_i \beta + e_i)}{\sum_{i=1}^nX_i^4} \\
&=  \frac{\sum_{i=1}^n X_i^4\beta + \sum_{i=1}^nX_i^3e_i}{\sum_{i=1}^nX_i^4}\\
&= \beta + \frac{ \sum_{i=1}^nX_i^3e_i}{\sum_{i=1}^nX_i^4}\\
\Rightarrow \sqrt{n}(\hat{\beta} - \beta) &\rightarrow_d \frac{1}{E[X_i^4]}N(0,E[X_i^6e_i^2])\\
&=  N\left(0,\frac{,E[X_i^6e_i^2]}{E[X_i^4]} \right)
\end{align*}
\section{7.17}
\subsection{Part A}
Under the null hypothesis that $\theta = 0,$ $\sqrt{n}(\hat{\theta} - \theta) \rightarrow_d N(0,Var(\hat{\theta}))= N(0,Var(\hat{\beta}_1 - \hat{\beta}_2)) = N(0,Var(\hat{\beta}_1) + Var(\hat{\beta}_2) - 2Cov(\hat{\beta}_1,\hat{\beta}_2)) \sim N(0,s(\hat{\beta}_1)^2 + s(\hat{\beta}_2)^2 - 2\hat{\rho} s(\hat{\beta}_1) s(\hat{\beta}_2)). $ Therefore, the 95\% CI for $\hat{\theta} \\= \left[\hat{\theta} - 1.96\sqrt{s(\hat{\beta}_1)^2 + s(\hat{\beta}_2)^2 - 2\hat{\rho} s(\hat{\beta}_1) s(\hat{\beta}_2)},\hat{\theta} + 1.96\sqrt{s(\hat{\beta}_1)^2 + s(\hat{\beta}_2)^2 - 2\hat{\rho} s(\hat{\beta}_1) s(\hat{\beta}_2)}\right] \\ = \left[0.2 - 1.96\sqrt{2(0.07)^2 (1-\hat{\rho})},  0.2 + 1.96\sqrt{2(0.07)^2 (1-\hat{\rho}) }\right]$.
\subsection{Part B}
We are not given the estimated covariance of $\hat{\beta}_1,\hat{\beta}_2$ so we cannot calculate the estimated correlation.
\subsection{Part C}
Correlation is in $[-1,1]$ so an upper bound for the width of the confidence interval is when the estimated correlation is $-1: [0.2 - 1.96*2*(0.07),0.2 + 1.96*2*(0.07)] =  [-0.0744,0.4744]$. This bound contains $0$ so we cannot reject the null hypothesis given the reported information.
\section{7.19}
\begin{align*}
\sqrt{n}(\hat{\beta}_1 - \hat{\beta}_2) &=\sqrt{n}(\hat{\beta}_1 - \beta) - \sqrt{n}(\hat{\beta}_2 - \beta).
%\sqrt{n}(\hat{\beta}_1 - \beta)&\rightarrow_d N(0,E(x_ix_i')^{-1}E(\epsilon_i^2 x_i x_i')E(x_ix_i')^{-1})\\
%\sqrt{n}(\hat{\beta}_2 - \beta)&\rightarrow_d N(0,E(x_ix_i')^{-1}E(\epsilon_i^2 x_i x_i')E(x_ix_i')^{-1}).
%\end{align*}
%Note also the following:
%\begin{align*}
%Cov\left( \hat{\beta}_1 - \beta, \hat{\beta_2} - \beta \right) &= E[\hat{\beta}_1\hat{\beta}_2 - \beta\hat{\beta}_1 - \beta \hat{\beta}_2 + \beta^2]\\
%&= 
\end{align*}
Let us add an indicator $d_i: 1\{ \text{is in the first split} \}$. Then, the regression equation is of the form: 
\begin{equation*}
y_i = d_ix_i'\beta + (1-d_i)x_i\beta + \epsilon_i
\end{equation*}

\begin{align*}
\sqrt{n}\left[ \colvec{2}{\hat{\beta}_1}{\hat{\beta}_2} - \colvec{2}{\beta}{\beta} \right] &= \left[ \frac{1}{2n}\sum_{i=1}^n\colvec{2}{d_ix_i}{(1-d_i)x_i}\colvec{2}{d_ix_i}{(1-d_i)x_i}'\right] \frac{1}{2\sqrt{n}} \sum_{i=1}^n \colvec{2}{d_ix_i\epsilon_i}{(1-d_i)x_i\epsilon_i}\\
&= \left[\frac{1}{2n}\begin{pmatrix} \sum_{i=1}^{\infty}d_i x_i x_i' &  \sum_{i=1}^{\infty}d_i (1-d_i) x_i x_i' \\  \sum_{i=1}^{\infty}d_i (1-d_i) x_i x_i' &  \sum_{i=1}^{\infty}(1-d_i)x_i x_i' \end{pmatrix} \right]^{-1}  \frac{1}{2\sqrt{n}} \sum_{i=1}^n \colvec{2}{d_ix_i\epsilon_i}{(1-d_i)x_i\epsilon_i}
\end{align*}
\begin{align*}
\frac{1}{2n}\begin{pmatrix} \sum_{i=1}^{\infty}d_i x_i x_i' &  \sum_{i=1}^{\infty}d_i (1-d_i) x_i x_i' \\  \sum_{i=1}^{\infty}d_i (1-d_i) x_i x_i' &  \sum_{i=1}^{\infty}(1-d_i)x_i x_i' \end{pmatrix} &\rightarrow_p \begin{pmatrix} \frac{1}{2}E[x_ix_i'] & 0 \\ 0 & \frac{1}{2}E[x_i x_i'] \end{pmatrix}\\
\frac{1}{2\sqrt{n}} \sum_{i=1}^n \colvec{2}{d_ix_i\epsilon_i}{(1-d_i)x_i\epsilon_i} &\rightarrow_d N\left(\colvec{2}{0}{0},\begin{pmatrix}\frac{1}{2} E[e_i^2 x_ix_i'] & 0 \\ 0 & \frac{1}{2} E[e_i^2 x_ix_i']\end{pmatrix}\right)
\end{align*}
\begin{align*}
\Rightarrow \sqrt{n}\left[ \colvec{2}{\hat{\beta}_1}{\hat{\beta}_2} - \colvec{2}{\beta}{\beta} \right] &= N(0,V \otimes I_2)
\end{align*}
where $I_2$ is the $2\times 2$ identity matrix, $\otimes$ is the kronecker product, and 
\begin{align*}
V = E[x_ix_i']^{-1}E[\epsilon_i^2x_ix_i']E[x_ix_i']^{-1}.
\end{align*}

Then, $\sqrt{n}(\hat{\beta}_1 - \hat{\beta}_2) =\sqrt{n}(\hat{\beta}_1 - \beta) - \sqrt{n}(\hat{\beta}_2 - \beta) \rightarrow_d N(0,2V)$.
\section{Q 9}
\subsection{Part A}
\begin{align*}
\hat{\beta} &= \left[\frac{1}{n} \sum_{i=1}^n w_i w_i' 1\{ x_i \in \{1,2\} \} \right]^{-1}\frac{1}{n}\sum_{i=1}^n w_i y_i 1\{ x_i \in \{1,2\}\}\\
&= \left[\frac{1}{n} \sum_{i=1}^n w_i w_i' 1\{ x_i \in \{1,2\} \} \right]^{-1}\frac{1}{n}\sum_{i=1}^n w_i (w_i' \beta +\epsilon_i) 1\{ x_i \in \{1,2\}\}\\
&= \beta + \left[\frac{1}{n} \sum_{i=1}^n w_i w_i' 1\{ x_i \in \{1,2\} \} \right]^{-1}\frac{1}{n}\sum_{i=1}^n w_i \epsilon_i 1\{ x_i \in \{1,2\}\} \\
&\rightarrow_p \beta + E[w_iw_i'1\{x\in \{ 1,2\}\}]^{-1} E[w_i\epsilon_i1\{x\in \{ 1,2\}]\\
&=\beta + E[w_iw_i'1\{x\in \{ 1,2\}\}]^{-1} E[w_iE[\epsilon_i|w_i]1\{x\in \{ 1,2\}]\\
&= \beta.
\end{align*}
Therefore, $\hat{\beta}\rightarrow_p \beta.$
\subsection{Part B}
\begin{align*}
\hat{\beta} &=\beta + E[w_iw_i'1\{x\in \{ 1,2\}\}]^{-1} E[w_iE[\epsilon_i|w_i]1\{x\in \{ 1,2\}]
\end{align*}
(A1') does not give us enought to deal with the indicator function inside the second expectation. So, in general, no.
\subsection{Part C}
\begin{align*}
\sqrt{n}(\hat{\beta} - \beta) &=  \left[\frac{1}{n} \sum_{i=1}^n w_i w_i' 1\{ x_i \in \{1,2\} \} \right]^{-1}\frac{1}{\sqrt{n}}\sum_{i=1}^n w_i \epsilon_i 1\{ x_i \in \{1,2\}\}\\
&\rightarrow_d E[w_i w_i' 1\{ x_i \in \{1,2\} \} ]^{-1}N(0,Var(w_i\epsilon_i 1\{ x_i \in \{1,2\} \}))\\
Var(w_i\epsilon_i 1\{ x_i \in \{1,2\} )&= E[\epsilon_i^2w_i w_i'1\{ x_i \in \{1,2\} \}]\\
&=  E[E\epsilon_i^2|w_i] w_i w_i' 1\{ x_i \in \{1,2\} \}]\\
&= \sigma^2E[w_i w_i' 1\{ x_i \in \{1,2\} \}]\\
&= \sigma^2\begin{pmatrix}1/2 & 3/4 \\ 3/4 & 5/4 \end{pmatrix}\\
\Rightarrow \sqrt{n}(\hat{\beta} - \beta) &\rightarrow_d E[w_i w_i' 1\{ x_i \in \{1,2\}  ]^{-1}N(0,Var(w_i\epsilon_i 1\{ x_i \in \{1,2\} \}))\\
&\sim N\left(0,\sigma^2\begin{pmatrix}1/2 & 3/4 \\ 3/4 & 5/4 \end{pmatrix}^{-1}\right)\\
&\sim N\left(0,\sigma^2\begin{pmatrix} 20 & -12 \\ -12 & 8 \end{pmatrix}\right)
\end{align*}
\subsection{Part D}
From identical logic to that which we used in Part A, we know that $\hat{\hat{\beta}}_2 $ is a consistent estimator for $\gamma$. As we have shown in Part A that $\hat{\beta}_2$ is also consistent, we can choose estimators by comparing asymptotic variances. We showed in Part D that this variance is $8\sigma^2$ for $\hat{\beta}_2$, while by replicating the same steps we followed in Part C with the inequality in the indicator function flipped, we find that the asymptotic variance of $\hat{\hat{\beta}}_2$ is $ 72\sigma^2>8\sigma^2$. Thus, we should use $\hat{\beta}^2$ as it yields more precise estimates of the slope coefficient.
\subsection{Part E}
\begin{align*}
\hat{\alpha} &= \left[\frac{1}{n} \sum_{i=1}^n x_i x_i' 1\{ x_i \in \{1,2\} \} \right]^{-1}\frac{1}{n}\sum_{i=1}^n x_i y_i 1\{ x_i \in \{1,2\}\}\\
%&=  \left[\frac{1}{n} \sum_{i=1}^n x_i x_i' 1\{ x_i \in \{1,2\} \} \right]^{-1}\frac{1}{n}\sum_{i=1}^n x_i (w_i'\beta + \epsilon_i) 1\{ x_i \in \{1,2\}\}
&\rightarrow_p E[x_ix_i'1\{ x_i \in \{ 1,2\}\}]^{-1} E[x_iy_i1\{ x_i \in \{ 1,2\}\}]\\
&= E[x_ix_i'1\{ x_i \in \{ 1,2\}\}]^{-1}(E[x_i1\{ x_i \in \{ 1,2\}\}] + \gamma E[x_ix_i'1\{ x_i \in \{ 1,2\}\}] + E[x_i\epsilon_i1\{ x_i \in \{ 1,2\}\}])\\
&= (5/4)^{-1}((3/4) + \gamma(5/4) + 0)\\
&= \gamma + 3/5
\end{align*}
\subsection{Part F}
\begin{align*}
\sqrt{n}(\hat{\alpha}-\alpha) &= \left[\frac{1}{n} \sum_{i=1}^n x_i x_i' 1\{ x_i \in \{1,2\} \} \right]^{-1}\frac{1}{\sqrt{n}}\sum_{i=1}^n (x_i +x_i^2(\gamma - \alpha) + x_i \epsilon_i) 1\{ x_i \in \{1,2\}\}\\
&\rightarrow_d N(0,(4/5)^2Var[(x_i +x_i^2(\gamma - \alpha) + x_i \epsilon_i) 1\{ x_i \in \{1,2\}\}])
\end{align*}
\begin{align*}
Var[(x_i +x_i^2(\gamma - \alpha) + x_i \epsilon_i) 1\{ x_i \in \{1,2\}\}]&= E[(x_i +x_i^2(\gamma - \alpha) + x_i \epsilon_i)^2 1\{ x_i \in \{1,2\}\}]
\end{align*}
\begin{align*}
&=E[x_i^21\{ x_i \in \{1,2\}\}]+(9/25)E[x_i^41\{ x_i \in \{1,2\}\}]+\sigma^2 E[x_i^21\{ x_i \in \{1,2\}\}]\\&-(6/5)E[x_i^31\{ x_i \in \{1,2\}\}] +2E[x_i^2\epsilon_i1\{ x_i \in \{1,2\}\}] - (6/5)E[x_i^3 \epsilon_i1\{ x_i \in \{1,2\}\}]\\
&=(5/4) - (6/5)(9/4) + (9/25)(17/4) + \sigma^2 (5/4)\\
\Rightarrow \sqrt{n}(\hat{\alpha}-\alpha) &\rightarrow_d N(0,(4/5)^2((5/4) - (6/5)(9/4) + (9/25)(17/4) + \sigma^2 (5/4)))\\
&\sim N(0,(16/25)(2/25 + (5/4)\sigma^2))
\end{align*}
\end{document}
