% !TEX TS-program = pdflatex
% !TEX encoding = UTF-8 Unicode

% This is a simple template for a LaTeX document using the "article" class.
% See "book", "report", "letter" for other types of document.

\documentclass[11pt]{article} % use larger type; default would be 10pt

\usepackage[utf8]{inputenc} % set input encoding (not needed with XeLaTeX)

%%% Examples of Article customizations
% These packages are optional, depending whether you want the features they provide.
% See the LaTeX Companion or other references for full information.

%%% PAGE DIMENSIONS
\usepackage{geometry} % to change the page dimensions
\geometry{a4paper} % or letterpaper (US) or a5paper or....
% \geometry{margin=2in} % for example, change the margins to 2 inches all round
% \geometry{landscape} % set up the page for landscape
%   read geometry.pdf for detailed page layout information

\usepackage{graphicx} % support the \includegraphics command and options

% \usepackage[parfill]{parskip} % Activate to begin paragraphs with an empty line rather than an indent
\usepackage{amssymb}
\usepackage{amsmath}
%%% PACKAGES
\usepackage{booktabs} % for much better looking tables
\usepackage{array} % for better arrays (eg matrices) in maths
\usepackage{paralist} % very flexible & customisable lists (eg. enumerate/itemize, etc.)
\usepackage{verbatim} % adds environment for commenting out blocks of text & for better verbatim
\usepackage{subfig} % make it possible to include more than one captioned figure/table in a single float
% These packages are all incorporated in the memoir class to one degree or another...

%%% HEADERS & FOOTERS
\usepackage{fancyhdr} % This should be set AFTER setting up the page geometry
\pagestyle{fancy} % options: empty , plain , fancy
\renewcommand{\headrulewidth}{0pt} % customise the layout...
\lhead{}\chead{}\rhead{}
\lfoot{}\cfoot{\thepage}\rfoot{}

%%% SECTION TITLE APPEARANCE
\usepackage{sectsty}
\allsectionsfont{\sffamily\mdseries\upshape} % (See the fntguide.pdf for font help)
% (This matches ConTeXt defaults)

%%% ToC (table of contents) APPEARANCE
\usepackage[nottoc,notlof,notlot]{tocbibind} % Put the bibliography in the ToC
\usepackage[titles,subfigure]{tocloft} % Alter the style of the Table of Contents

\renewcommand{\cftsecfont}{\rmfamily\mdseries\upshape}
\renewcommand{\cftsecpagefont}{\rmfamily\mdseries\upshape} % No bold!
\DeclareMathOperator*{\argmax}{arg\,max}
\DeclareMathOperator*{\argmin}{arg\,min}

\newcount\colveccount
\newcommand*\colvec[1]{
        \global\colveccount#1
        \begin{pmatrix}
        \colvecnext
}
\def\colvecnext#1{
        #1
        \global\advance\colveccount-1
        \ifnum\colveccount>0
                \\
                \expandafter\colvecnext
        \else
                \end{pmatrix}
        \fi
}

\newcommand{\norm}[1]{\left\lVert#1\right\rVert}

\title{Econometrics HW2}
\author{Michael B. Nattinger\footnote{I worked on this assignment with my study group: Alex von Hafften, Andrew Smith, and Ryan Mather. I have also discussed problem(s) with Emily Case, Sarah Bass, and Danny Edgel.}}

\begin{document}
\maketitle

\section{Question 1}
Suppose that $Y = X^{3}$ and $f_X(x) = 42x^5(1-x), x \in (0,1).$ We are asked to find the PDF of $Y$.

We will begin by finding the CDF of $Y$:
\begin{align*}
P(Y\leq y) &= P(X^{3} \leq y) = P(X \leq y^{1/3}) = \int_0^{y^{1/3}} f_X(x)dx =  \int_0^{y^{1/3}}42x^5 - 42x^6dx \\
&= (7x^6 - 6x^7)|^{y^{1/3}}_{0} = (7y^2 - 6y^{7/3}) - 0 = 7y^2 - 6y^{7/3}.
\end{align*}
Thus, $F_Y(y) =  7y^2 - 6y^{7/3}$. $f_Y(y) = \frac{d}{dy} F_Y(y) =  \frac{d}{dy} 7y^2 - 6y^{7/3} = 14y - \frac{42}{3}y^{4/3}$.

We will check that this integrates to 1:
$\int_{-\infty}^{\infty} f_Y(y)dy = \int_0^1f_Y(y)dy = F_Y(y)|^{1}_{0} = (7 - 6) - (0) = 1. $

\section{Question 2}
Let $x \in [0,1]$ and define $F_X,f_X,a$ as described in the problem. We then have 3 cases:

\begin{itemize}
\item $x<0.5$: In this case, $\int^{x}_0 f_X(t)dt = \int^{x}_01.2 dx = 1.2x.$
\item $x=0.5$: In this case,  $\int^{x}_0 f_X(t)dt = \int^{0.5}_0 1.2dt + \int^{0.5}_{0.5} a dt = 0.6 + 0 = 0.2+0.8(x).$
\item $x>0.5$:In this case,  $\int^{x}_0 f_X(t)dt = \int^{0.5}_0 1.2dt + \int^{0.5}_{0.5} a dt + \int^{x}_{0.5} 0.8 dt = 0.6 + 0 + 0.8x - 0.4 = 0.2+0.8(x).$
\end{itemize}

Thus, $F_X(x) = \int_0^x f_X(t)dt$ $\forall x \in [0,1].$

\section{Question 3}
We will begin by finding the CDF of Y. $P(Y\leq y) = P(X^2 \leq y) = P(|X| \leq \sqrt{y}) = P(-\sqrt{y}\leq X \leq \sqrt{y})$. Note: $Y$ is weakly positive. Also, $Y\leq 4$ because $|X| \leq 2. $ We then have, for $y=0, P(Y\leq 0) = F(0) = 0 \Rightarrow f_Y(0) = 0.$

For $y \in (0,1],$
\begin{align*}
P(-\sqrt{y}\leq X \leq \sqrt{y}) &= \int_{-\sqrt{y}}^{\sqrt{y}} (2/9) (x+1) dx = ((1/9)x^2  + (2/9)x)|_{-\sqrt{y}}^{\sqrt{y}}\\
 &= ((1/9)y + (2/9)\sqrt{y}) -((1/9)y - (2/9)\sqrt{y}) = (4/9) \sqrt{y} \\
\Rightarrow f_Y(y) &= \frac{d}{dy} (4/9)\sqrt{y} = (2/9)y^{-1/2}.
\end{align*}

For $y \in (1,4],$
\begin{align*}
P(-\sqrt{y}\leq X \leq \sqrt{y}) &= \int_{-1}^{\sqrt{y}} (2/9) (x+1) dx = ((1/9)x^2  + (2/9)x)|_{-1}^{\sqrt{y}}\\
 &= ((1/9)y + (2/9)\sqrt{y}) -((1/9) - (2/9)) = (1/9)y + (2/9)\sqrt{y} + (1/9) \\
\Rightarrow f_Y(y) &= \frac{d}{dy} \left((1/9)y + (2/9)\sqrt{y} + (1/9)\right)  = (1/9) +(1/9)y^{-1/2}
\end{align*}

\section{Question 4}
We will find the median of the given distribution.

\begin{align*}
P(X\leq m) &= \int_{-\infty}^m \frac{1}{\pi (1+x^2)} dx = \frac{1}{\pi}(\text{tan}^{-1} x |_{\infty}^{m}) = \frac{1}{\pi}\left(\text{tan}^{-1}(m) + \frac{\pi}{2}\right) \\
\Rightarrow P(X \leq m) &= 0.5 \rightarrow \frac{1}{\pi}\left( \text{tan}^{-1}(m) + \frac{\pi}{2}\right) = 0.5 \Rightarrow m = \text{tan}\left(\frac{\pi}{2} - \frac{\pi}{2}\right)\\
\Rightarrow m &= 0.
\end{align*}

\section{Question 5}
%We will begin by finding $E|X - a|,a\in\mathbb{R}$. We have that $E|X - a| = \int_{-\infty}^{\infty}|t-a|f_X(t)dt =  \int_{-\infty}^{a} (a-t)f_X(t)dt+ \int_{a}^{\infty} (t-a)f_X(t)dt =  \int_{-\infty}^{a} af_X(t)dt -  \int_{-\infty}^{a} t f_X(t)dt +  \int_{a}^{\infty} tf_X(t)dt - \int_{a}^{\infty} a f_X(t)dt = aF(a) - a(1-F(a)) - E[X|X<a] + E[X|X\geq a] = 2aF(a) - a + (- E[X|X<a] + E[X|X\geq a]) $.
%
%Taking the derivative with respect to a, $\frac{d}{da} E|X - a| =  2F(a) + 2af(a) -1 + \frac{d}{da} (- E[X|X<a] + E[X|X\geq a]) = 2F(a) + 2af(a) -1 + (-xf(x)|_{a} + xf(x)|_{a}) = 2F(a) + 2af(a) -1$.

We will begin by finding $E|X - a|,a\in\mathbb{R}$. We have that $E|X - a| = \int_{-\infty}^{\infty}|t-a|f_X(t)dt =  \int_{-\infty}^{a} (a-t)f_X(t)dt+ \int_{a}^{\infty} (t-a)f_X(t)dt $. Taking the derivative with respect to a, $ \frac{d}{da} E|X - a| = ((a-t)f_X(t)|_{a}) + \int_{-\infty}^{a}f_X(t)dt + ((t-a)f_X(t)|_{a}) - \int_{a}^{\infty} f_X(t)dt =  \int_{-\infty}^{a}f_X(t)dt  - \int_{a}^{\infty} f_X(t)dt $. At the minimum, the derivitave with respect to a is $0$ so $ \int_{-\infty}^{a}f_X(t)dt  = \int_{a}^{\infty} f_X(t)dt \Rightarrow P(X\leq a) = P(X \geq a) = 0.5$ so $\min\limits_{a}E|X-a| = E|X-m|$ where $m$ is the median of $X$..

%Taking the derivative with respect to a, $\frac{d}{da} E|X - a| =  2F(a) + 2af(a) -1 + \frac{d}{da} (- E[X|X<a] + E[X|X\geq a]) = 2F(a) + 2af(a) -1 + (-xf(x)|_{a} + xf(x)|_{a}) = 2F(a) + 2af(a) -1$.

\section{Question 6}
\subsection{Show that if a density function is symmetric about a point a, then $\alpha_3= 0.$}
Let $X$ be a random variable, with a density function symmetric about point $a$.  Define $Y = X-a$, a random variable. Notice that $E[Y^3] = E[(-Y)^3]$ by the symmetry of the distribution of $X$. This implies that $E[Y^3] = 0$. Also, by symmetry, $E[X] = a$ so $\mu_3 = E(X - E[X])^3 = 0.$ Thus $\alpha_3 = 0.$
\subsection{Calculate $\alpha_3$ for $f(x) = e^{-x},x\geq0$}
By the chain rule, $E[X] = \int_{0}^{\infty} te^{-t}dt = -te^{-t}  - e^{-t} |_{0}^{\infty} = 1.$

$E(X^2) = \int_{0}^{\infty} t^2 e^{-x} dt = (-t^2 e^t)|^{\infty}_{0} + 2 \int_{0}^{\infty}  te^{-t}dt = 2.$ Thus, $\mu_2 = E(X^2) - E(X)^2 = 2-1 = 1.$

$E(X - E(X))^3 =  \int_{0}^{\infty} (t-1)^3 e^{-x} dt = \int_{0}^{\infty} (t^3 - 3t^2 + 3t - 1) e^{-x} dt\\ =  \int_{0}^{\infty} t^3 e^{-x} dt -3 \int_{0}^{\infty}t^2 e^{-x}dt + 3  \int_{0}^{\infty}t e^{-x}dt - \int_{0}^{\infty}e^{-x}dt  = \int_{0}^{\infty} t^3 e^{-x} dt -3 (2) + 3  (1) - (1) \\ = (-t^3 e^{-t})|_{0}^{\infty} + 3\int_0^{\infty} t^2 e^{-t}dt -4 = 0 + 3(2) -4 = 2 = \mu_3.$

Thus, $\alpha_3 = \frac{2}{1^{3/2}} = 2.$

\subsection{Calculate $\alpha_4$ for the listed densities and comment on the peakedness of the distributions.}
\begin{itemize}
\item From lecture we have the moment generating function of $f(x)$: $M(t) = e^{t^2/2}$ and several derivatives including $M^{''}(t) = e^{t^2/2} + t^2 e^{t^2/2}, M^{'''}(t) = 3t e^{t^2/2} + t^3e^{t^2/2}.$ We take a fourth derivative of $M$: $ M^{''''}(t) = 3e^{t^2/2} + 3t^2e^{t^2/2} + 3t^2e^{t^2/2} + t^4 e^{t^2/2} =  3e^{t^2/2} +  6t^2e^{t^2/2} + t^4 e^{t^2/2}.$ Thus, $M^{''''}(0) = 3, M^{''}(0) = 1 \Rightarrow \alpha_4 = 3.$ 
\item By symmetry, $E[X] = 0.$ $E(X^2) = \int_{-1}^{1} t^2/2dt = (t^3/6)|_{-1}^{1} = (1/6) -(-1/6) = 1/3$ so the second central moment of the distribution is 1/3. Because the mean of the distribution is 0, $E(X-E[X])^4 = E[X^4] = \int_{-1}^1 t^4/2 dt = t^5/10|_{-1}^{1} = 1/5 \Rightarrow \alpha_4 = \frac{1/5}{1/3} = \frac{3}{5}.$ This distribution is, therefore, less peaked than the standard normal distribution, which is intuitive.
\item By symmetry $E[X] = 0 \Rightarrow E(X-E[X])^k = E(X^k).$ $E(X^2) = \int_{-\infty}^{\infty}t^2(1/2)e^{-|t|}dt = 2\int_{0}^{\infty}t^2(1/2)e^{-t}dt =  \int_{0}^{\infty}t^2e^{-t}dt.$ We calculated this quantity in the previous subsection to be $2$.  $E(X^4) = \int_{-\infty}^{\infty}t^4(1/2)e^{-|t|}dt = \int_{0}^{\infty}t^4e^{-t}dt = ((-t^3 e^{-t})|_{0}^{\infty}) + (1/3)\int_{0}^{\infty} t^3e^{-t}dt = 0 + (1/3)(6)  =2$. Thus, $\alpha_4 = \frac{2}{2} = 1.$ Thus, this distribution is less peaked than the standard normal distribution, but more peaked than the uniform distribution.
\end{itemize}
% use moment generating functions
\end{document}
