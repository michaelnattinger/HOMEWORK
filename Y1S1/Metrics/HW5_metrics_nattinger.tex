% !TEX TS-program = pdflatex
% !TEX encoding = UTF-8 Unicode

% This is a simple template for a LaTeX document using the "article" class.
% See "book", "report", "letter" for other types of document.

\documentclass[11pt]{article} % use larger type; default would be 10pt

\usepackage[utf8]{inputenc} % set input encoding (not needed with XeLaTeX)

%%% Examples of Article customizations
% These packages are optional, depending whether you want the features they provide.
% See the LaTeX Companion or other references for full information.

%%% PAGE DIMENSIONS
\usepackage{geometry} % to change the page dimensions
\geometry{a4paper} % or letterpaper (US) or a5paper or....
\geometry{margin=1in} % for example, change the margins to 2 inches all round
% \geometry{landscape} % set up the page for landscape
%   read geometry.pdf for detailed page layout information

\usepackage{graphicx} % support the \includegraphics command and options

% \usepackage[parfill]{parskip} % Activate to begin paragraphs with an empty line rather than an indent
\usepackage{amssymb}
\usepackage{amsmath}
%%% PACKAGES
\usepackage{booktabs} % for much better looking tables
\usepackage{array} % for better arrays (eg matrices) in maths
\usepackage{paralist} % very flexible & customisable lists (eg. enumerate/itemize, etc.)
\usepackage{verbatim} % adds environment for commenting out blocks of text & for better verbatim
\usepackage{subfig} % make it possible to include more than one captioned figure/table in a single float
% These packages are all incorporated in the memoir class to one degree or another...

%%% HEADERS & FOOTERS
\usepackage{fancyhdr} % This should be set AFTER setting up the page geometry
\pagestyle{fancy} % options: empty , plain , fancy
\renewcommand{\headrulewidth}{0pt} % customise the layout...
\lhead{}\chead{}\rhead{}
\lfoot{}\cfoot{\thepage}\rfoot{}

%%% SECTION TITLE APPEARANCE
\usepackage{sectsty}
\allsectionsfont{\sffamily\mdseries\upshape} % (See the fntguide.pdf for font help)
% (This matches ConTeXt defaults)

%%% ToC (table of contents) APPEARANCE
\usepackage[nottoc,notlof,notlot]{tocbibind} % Put the bibliography in the ToC
\usepackage[titles,subfigure]{tocloft} % Alter the style of the Table of Contents
\usepackage{bbm}

\renewcommand{\cftsecfont}{\rmfamily\mdseries\upshape}
\renewcommand{\cftsecpagefont}{\rmfamily\mdseries\upshape} % No bold!
\DeclareMathOperator*{\argmax}{arg\,max}
\DeclareMathOperator*{\argmin}{arg\,min}

\newcount\colveccount
\newcommand*\colvec[1]{
        \global\colveccount#1
        \begin{pmatrix}
        \colvecnext
}
\def\colvecnext#1{
        #1
        \global\advance\colveccount-1
        \ifnum\colveccount>0
                \\
                \expandafter\colvecnext
        \else
                \end{pmatrix}
        \fi
}

\newcommand{\norm}[1]{\left\lVert#1\right\rVert}

\title{Econometrics HW5}
\author{Michael B. Nattinger\footnote{I worked on this assignment with my study group: Alex von Hafften, Andrew Smith, and Ryan Mather. I have also discussed problem(s) with Emily Case, Sarah Bass, and Danny Edgel.}}

\begin{document}
\maketitle

\section{Question 1}
\subsection{$a_n = 1/n$}
Let $\epsilon>0$. Let $N$ be the smallest integer such that $N>1/\epsilon.$ Then, $|a_n -0| = 1/n<\epsilon $ $\forall n>N.$
\subsection{$a_n = \frac{1}{n}sin(n\pi/2)$}
Let $\epsilon>0$. Let $N$ be the smallest integer such that $N>1/\epsilon.$ Then, $|a_n -0| = |\frac{1}{n}sin(n\pi/2)|\leq \frac{1}{n}<\epsilon$ $\forall n>N$.

\section{Question 2}
\subsection{Does $X_n\rightarrow_p 0$ as $n \rightarrow 0$?}
Let $\epsilon>0$. Let $N$ be the smallest integer such that $N>\epsilon.$ Then, for $n>N,$ $P(|X_n|\geq \epsilon) = 2/n$  so $\lim\limits_{n\rightarrow \infty}P(|X_n|\geq \epsilon) = \lim\limits_{n\rightarrow \infty} 2/n =  0 $, so $X_n \rightarrow_p 0.$%$P(|X_n - 0|\geq \epsilon) = 0 as n\rightarrow \infty$
\subsection{Calculate $E(X_n)$.}
$E(X_n) = -n(1/n) +0(1-2/n) + n(1/n) = 0.$
\subsection{Calculate $Var(X_n)$.}
$Var(X_n) = E(X_n^2) - E(X_n)^2 = (n^2)(1/n) + (0^2)(1-2/n) + (n^2)(1/n) -0^2 = 2n.$
\subsection{Calculate $X_n$ for the next distribution.}
$E(X_n) = (0)(1-1/n) + (n)(1/n) =1.$
\subsection{Conclude that $\dots$.}
Note that $\lim\limits_{n\rightarrow \infty}E(X_n) = \lim\limits_{n\rightarrow \infty}1 = 1.$ Now let $\epsilon >0$. Note that $\lim\limits_{n\rightarrow \infty}P(|X_n-0|<\epsilon) = \lim\limits_{n\rightarrow \infty}1/n$ for $n>N$ where $N$ is the smallest integer such that $N>1/\epsilon$. Thus, $\lim\limits_{n\rightarrow \infty}P(|X_n-0|<\epsilon) = \lim\limits_{n\rightarrow \infty}1/n = 0$ so $X_n \rightarrow_p 0$, yet $E(X_n) \rightarrow 1.$

\section{Question 3}
\subsection{Show that $\bar{Y}^{*}$}
$E(\bar{Y}^{*}) = E(\frac{1}{n}\sum_{i=1}^nw_iY_i) = \frac{1}{n}\sum_{i=1}^nw_iE(Y_i) =  \frac{1}{n}\sum_{i=1}^nw_i\mu =  \frac{\mu}{n}\sum_{i=1}^nw_i = \mu.$
\subsection{Calculate $Var(\bar{Y}^{*})$} %Do what Alex did here
$Var(\bar{Y}^{*}) =\frac{1}{n^2}Var(\sum_{i=1}^nw_iY_i)= \frac{1}{n^2}\sum_{i=1}^nw_i^2Var(Y_i) = \sigma_Y^2\frac{1}{n^2}\sum_{i=1}^nw_i^2$ where $\sigma_Y^2$ is the variance of each draw of $Y$.% E[\frac{1}{n}\sum_{i=1}^nw_iY_i]^2 - \mu^2 = \frac{1}{n^2}E[\sum_{i=1}^nw_i^2Y_i^2 +\sum_{i\neq j}w_iY_iw_jY_j] - \mu^2$
\subsection{Show the first sufficient condition.}
Let $\frac{1}{n^2}\sum_{i=1}^n w_i^2 \rightarrow 0.$ Let $\epsilon>0$. By Chebyshev's inequality, $P(|\bar{Y}^{*}-\mu|\geq \epsilon) \leq \frac{\sigma_Y^2\sum_{i=1}^nw_i^2}{n^2\epsilon^2} \rightarrow_{n\rightarrow \infty}0 $ so $\lim_{n\rightarrow \infty} P(|\bar{Y}^{*}-\mu|\geq \epsilon) = 0.$

\subsection{Show the second sufficient condition.}
Now, let $\max_{i\leq n}w_i/n \rightarrow 0$. Let $\epsilon>0$. By Chebyshev's inequality, $P(|\bar{Y}^{*}-\mu|\geq \epsilon) \leq \frac{\sigma_Y^2\sum_{i=1}^nw_i^2}{n^2} $\\$\leq \frac{\sigma_Y^2\sum_{i=1}^nw_i \max_{j\leq n}w_j}{n^2} = \frac{\sigma_Y^2\max_{j\leq n}w_j\sum_{i=1}^nw_i}{n^2} =  \frac{\sigma_Y^2\max_{j\leq n}w_j}{n} \rightarrow_{n\rightarrow \infty} 0. $

\section{Question 4}
\subsection{$\frac{1}{n}\sum_{i=1}^n X_i^2$}
Assuming the moment exists, by the WLLN $\frac{1}{n}\sum_{i=1}^n X_i^2 \rightarrow_p E[X_i^2]$.
\subsection{$\frac{1}{n}\sum_{i=1}^n X_i^3$}
Assuming the moment exists, by the WLLN $\frac{1}{n}\sum_{i=1}^n X_i^3 \rightarrow_p E[X_i^3]$.
\subsection{$\max_{i\leq n}X_i$ }
We cannot say anything using WLLN or CMT. %In general this may not converge.
\subsection{$\frac{1}{n}\sum_{i=1}^n X_i^2 - (\frac{1}{n}\sum_{i=1}^n X_i)^2$}
Assuming the necessary moments exist, by WLLN, $\frac{1}{n}\sum_{i=1}^n X_i^2 \rightarrow_p E[X_i^2], \frac{1}{n}\sum_{i=1}^n X_i \rightarrow_p E[X_i]$ so by continuity and the CMT, $\frac{1}{n}\sum_{i=1}^n X_i^2 - (\frac{1}{n}\sum_{i=1}^n X_i)^2 \rightarrow_p Var(X_i)$.
\subsection{$\frac{\sum_{i=1}^n X_i^2 }{\sum_{i=1}^n X_i }$}
Assuming the moments exist, by WLLN, $\frac{1}{n}\sum_{i=1}^n X_i^2 \rightarrow_p E[X_i^2], \frac{1}{n}\sum_{i=1}^n X_i \rightarrow_p E[X_i]$ so by continuity and the CMT,  $\frac{\sum_{i=1}^n X_i^2 }{\sum_{i=1}^n X_i }\rightarrow_p E[X_i^2]/E[X_i]. $
\subsection{$\mathbbm{1}(\sum_{i=1}^n X_i )$}
By WLLN, $ \frac{1}{n}\sum_{i=1}^n X_i \rightarrow_p E[X_i]$ so by CMT $\mathbbm{1}(\sum_{i=1}^n X_i ) \rightarrow_p \mathbbm{1}(E[X_i]>0) $ unless $E[x_i]=0$ in which case the indicator function is not continuous at that point, and CMT cannot be applied.
\section{Question 5}
Note that $\hat{\mu} = exp(log(\hat{\mu})) = exp(log((\pi_{i=1}^n X_i)^{1/n})) = exp((1/n)\sum_{i=1}^nlog(X_i)).$ Due to continuity of log, exp on $(0,\infty)$, the WLLN and CMT gives us $\hat{\mu} =  exp((1/n)\sum_{i=1}^nlog(X_i)) \rightarrow_p exp(E(log(X_i))) = \mu$.
\section{Question 6}
\subsection{Find the natural moment estimator for $\mu_k$}
Define $\hat{\mu}_k := \frac{1}{n}\sum_{i=1}^nX_i^k.$ By WLLN this is a consistent estimator for $\mu_k$.
\subsection{Find the asymptotic distribution of $\sqrt{n}(\hat{\mu}_k - \mu)$ as $n\rightarrow \infty$.}
Assuming the necessary moments exist, by CLT $Var(X_i^k) = E(X_i^{2k}) - (E(X_i^k))^2$ so \\$\sqrt{n}(\hat{\mu}_k - \mu_k) \rightarrow_d N(0,\mu_{2k} - \mu_k^2)$.

\section{Question 7}
\subsection{Find a consistent estimator}
By continuity, assuming the moment exists, $\hat{m}_k = (\hat{\mu}_k)^{1/k}$ is a consistent estimator for $m_k$.
\subsection{Find the distribution} %delta method
Using the delta method, $\sqrt{n}(g(\hat{m}_k - m_k)) \rightarrow_{d}N(0,V)$ where $V = ((1/k)  (\mu_k)^{\frac{1-k}{k}})^2(\mu_{2k} - \mu_k^2) =\frac{1}{k^2} \mu_k^{2\frac{1-k}{k}}(\mu_{2k} - \mu_k^2 )$.
\section{Question 8}
\subsection{Use the Delta Method}
Using the Delta Method, $\sqrt{n}(\hat{\beta} - \beta) \rightarrow_d N(0,V)$ where $V= 4\mu^2 v^2$.
\subsection{What happens when $\mu$ is 0?}
If $\mu=0$ we get a degenerate normal with no variance; in the limit the distribution collapses into a unit point mass at 0.
\subsection{Improve your answer.} 
$\sqrt{n}\hat{\mu} \rightarrow_d N(0,v^2) \Rightarrow\sqrt{n}\hat{\mu}/v \rightarrow_d N(0,1)  \Rightarrow n\hat{\mu}^2/v^2 \rightarrow_d \chi^2_1  \Rightarrow n\hat{\beta} \rightarrow_d v^2\chi^2_1 $
\subsection{Why do we get this difference?}
It seems to me that, when $\beta=0$, the estimator $\hat{\beta}$ converges to $0$ at a rate of $n$ rather than a rate of $\sqrt{n}$. So, by checking the convergence at the rate of $\sqrt{n}$ we find that $\hat{\beta}$ has already converged to $0$, though if we check the convergence at the rate of $n$ we find a distribution at that rate, which takes the form of a scaled $\chi^2_1$.
\end{document}
