% !TEX TS-program = pdflatex
% !TEX encoding = UTF-8 Unicode

% This is a simple template for a LaTeX document using the "article" class.
% See "book", "report", "letter" for other types of document.

\documentclass[11pt]{article} % use larger type; default would be 10pt

\usepackage[utf8]{inputenc} % set input encoding (not needed with XeLaTeX)

%%% PAGE DIMENSIONS
\usepackage{geometry} % to change the page dimensions
\geometry{a4paper} % or letterpaper (US) or a5paper or....
\geometry{margin=1in} 
\usepackage{graphicx} % support the \includegraphics command and options

\usepackage{amssymb}
\usepackage{amsmath}
%%% PACKAGES
\usepackage{booktabs} % for much better looking tables
\usepackage{array} % for better arrays (eg matrices) in maths
\usepackage{paralist} % very flexible & customisable lists (eg. enumerate/itemize, etc.)
\usepackage{verbatim} % adds environment for commenting out blocks of text & for better verbatim
\usepackage{subfig} % make it possible to include more than one captioned figure/table in a single float
% These packages are all incorporated in the memoir class to one degree or another...

%%% HEADERS & FOOTERS
\usepackage{fancyhdr} % This should be set AFTER setting up the page geometry
\pagestyle{fancy} % options: empty , plain , fancy
\renewcommand{\headrulewidth}{0pt} % customise the layout...
\lhead{}\chead{}\rhead{}
\lfoot{}\cfoot{\thepage}\rfoot{}

%%% SECTION TITLE APPEARANCE
\usepackage{sectsty}
\allsectionsfont{\sffamily\mdseries\upshape} % (See the fntguide.pdf for font help)
% (This matches ConTeXt defaults)

%%% ToC (table of contents) APPEARANCE
\usepackage[nottoc,notlof,notlot]{tocbibind} % Put the bibliography in the ToC
\usepackage[titles,subfigure]{tocloft} % Alter the style of the Table of Contents
\renewcommand{\cftsecfont}{\rmfamily\mdseries\upshape}
\renewcommand{\cftsecpagefont}{\rmfamily\mdseries\upshape} % No bold!
\usepackage{graphicx}
\graphicspath{ {./pings/} }

\usepackage{amsmath}
\DeclareMathOperator*{\argmax}{arg\,max}
\DeclareMathOperator*{\argmin}{arg\,min}

\newcount\colveccount
\newcommand*\colvec[1]{
        \global\colveccount#1
        \begin{pmatrix}
        \colvecnext
}
\def\colvecnext#1{
        #1
        \global\advance\colveccount-1
        \ifnum\colveccount>0
                \\
                \expandafter\colvecnext
        \else
                \end{pmatrix}
        \fi
}


%%% END Article customizations

%%% The "real" document content comes below...

\title{Pset note}
\author{Michael B. Nattinger}
%\date{} % Activate to display a given date or no date (if empty),
         % otherwise the current date is printed 

\begin{document}
\maketitle

\section{Summary}
Our equilibrium consists of the following equations:
\begin{align}
Y_{t} &= A_tK_t^{\alpha}L_t^{1-\alpha} \label{F}\\
Y_{t} &= C_{t} + I_{t} + G_{t} \label{Y}\\
L_{t}^{\phi}C_t^{\sigma} &= (1-\tau_{L,t})A_t(1-\alpha) K_t^{\alpha}L_t^{-\alpha} \label{L}\\
C_t^{-\sigma}(1+\tau_{I,t}) &= \beta E_t[C_{t+1}^{-\sigma}[A_{t+1}\alpha K_{t+1}^{\alpha-1}L_{t+1}^{1-\alpha} + (1-\delta)(1+\tau_{I,t+1})]] \label{EE}
\end{align}

Log linearizing the labor supply equation we get the following:

\begin{align}
\phi l_t + \sigma c_t &= \frac{-\bar{\tau}_{L}}{1-\bar{\tau}_{L}}\hat{\tau}_{L,t}+\alpha k_t -\alpha l_t. \label{l}
\end{align}

Log linearizing the EE this is what I got:
 
\begin{align}
 \sigma(E_t[c_{t+1}]-  c_t) + \frac{\bar{\tau}_{I}}{1+\bar{\tau}_{I}}\hat{\tau}_{I,t} &= \beta E_t\left[\frac{\alpha\bar{A} \bar{K}^{\alpha - 1} \bar{L}^{1-\alpha}}{1+\bar{\tau}_I}(a_{t+1} + (1-\alpha)(- k_{t+1} + l_{t+1})) + (1-\delta)\frac{\bar{\tau}_{I}}{1+\bar{\tau}_{I}}\hat{\tau}_{I,t+1} \right] \label{ee}
\end{align}

Note that $\bar{\tau}_{I}$ being 0 makes the $\hat{\tau}_I$ term drop out, and a similar problem exists for the labor wedge. I had initially thought we could just define $\tilde{\tau}_{I,t}:= \frac{\bar{\tau}_{I}}{1+\bar{\tau}_{I}}\hat{\tau}_{I,t}$ (and a similar $\tilde{\tau}_{L,t}$), but if $\bar{\tau}_{I} = 0$ then $\tilde{\tau}_{I,t} = 0$. So this is probematic I think.

\end{document}
