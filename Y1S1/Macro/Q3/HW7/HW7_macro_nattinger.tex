% !TEX TS-program = pdflatex
% !TEX encoding = UTF-8 Unicode

% This is a simple template for a LaTeX document using the "article" class.
% See "book", "report", "letter" for other types of document.

\documentclass[11pt]{article} % use larger type; default would be 10pt

\usepackage[utf8]{inputenc} % set input encoding (not needed with XeLaTeX)

%%% PAGE DIMENSIONS
\usepackage{geometry} % to change the page dimensions
\geometry{a4paper} % or letterpaper (US) or a5paper or....
\geometry{margin=1in} 
\usepackage{graphicx} % support the \includegraphics command and options

\usepackage{amssymb}
\usepackage{amsmath}
%%% PACKAGES
\usepackage{booktabs} % for much better looking tables
\usepackage{array} % for better arrays (eg matrices) in maths
\usepackage{paralist} % very flexible & customisable lists (eg. enumerate/itemize, etc.)
\usepackage{verbatim} % adds environment for commenting out blocks of text & for better verbatim
\usepackage{subfig} % make it possible to include more than one captioned figure/table in a single float
% These packages are all incorporated in the memoir class to one degree or another...

%%% HEADERS & FOOTERS
\usepackage{fancyhdr} % This should be set AFTER setting up the page geometry
\pagestyle{fancy} % options: empty , plain , fancy
\renewcommand{\headrulewidth}{0pt} % customise the layout...
\lhead{}\chead{}\rhead{}
\lfoot{}\cfoot{\thepage}\rfoot{}

%%% SECTION TITLE APPEARANCE
\usepackage{sectsty}
\allsectionsfont{\sffamily\mdseries\upshape} % (See the fntguide.pdf for font help)
% (This matches ConTeXt defaults)

%%% ToC (table of contents) APPEARANCE
\usepackage[nottoc,notlof,notlot]{tocbibind} % Put the bibliography in the ToC
\usepackage[titles,subfigure]{tocloft} % Alter the style of the Table of Contents
\renewcommand{\cftsecfont}{\rmfamily\mdseries\upshape}
\renewcommand{\cftsecpagefont}{\rmfamily\mdseries\upshape} % No bold!
\usepackage{graphicx}
\graphicspath{ {./pings/} }

\usepackage{amsmath}
\DeclareMathOperator*{\argmax}{arg\,max}
\DeclareMathOperator*{\argmin}{arg\,min}

\newcount\colveccount
\newcommand*\colvec[1]{
        \global\colveccount#1
        \begin{pmatrix}
        \colvecnext
}
\def\colvecnext#1{
        #1
        \global\advance\colveccount-1
        \ifnum\colveccount>0
                \\
                \expandafter\colvecnext
        \else
                \end{pmatrix}
        \fi
}

%%% END Article customizations

%%% The "real" document content comes below...

\title{Macro PS7}
\author{Michael B. Nattinger}

%\date{} % Activate to display a given date or no date (if empty),
         % otherwise the current date is printed 

\begin{document}
\maketitle
\section{Question 1}
The collateral constraint is binding, i.e. $R_tB_t = Q_{t,t+1}^{min}K_t$. Our arbitrage condition is the following:
\begin{align*}
R_t = \frac{E_t[Q_{t+1}]+ \frac{1}{2}(\bar{K} - K_t) }{Q_t},
\end{align*} 
where the expectation is taken with respect to the realization of $a_{t+1}$. The budget constraint of the constrained household is the following:
\begin{align*}
Q_tK_t = (a_t + Q_t)K_{t-1} + B_t - R_{t-1}B_{t-1}\\
Q_tK_t = (a_t + Q_t)K_{t-1} + \frac{Q_{t,t+1}^{min}K_t}{R_t}- R_{t-1}B_{t-1}\\
Q_tK_t =  \frac{1}{1-\frac{Q_{t,t+1}^{min}}{R_tQ_t}}\left[ (a_t+Q_t)K_{t-1} - R_{t-1}B_{t-1} \right]
\end{align*}

Finally, optimality from the unconstrained household gives us that $R_t = \beta_2^{-1}$. Now we can rearrange to find something we can compute in Matlab:
\begin{align}
Q_t = \beta_{2}E_t[Q_{t+1}]+ \frac{\beta}{2}(\bar{K} - K_t) \label{fp}\\
Q_tK_t = \frac{1}{1-\frac{Q_{t,t+1}^{min}}{R_tQ_t}}\left[ (a_t+Q_t)K_{t-1} - R_{t-1}B_{t-1} \right] \label{Kt}
\end{align}

The state variables in this case are $K_{t-1}$ and our realization of $a$.

\section{Question 2}
Across a grid of capital and prices we can initialize our guess for a function $Q(K_{t-1},a_t,B_{t-1})$, for a grid of $K_{t-1}$ levels, and for each grid point calculate $K_t$ using (\ref{Kt}) and then update $Q(K_{t-1},a_t,B_{t-1})$ using equation (\ref{fp}) and our guess of $E[Q(K_{t},a_{t+1})]$. We continue to iterate until this function has converged.
\section{Question 3}
\section{Question 4}

\end{document}
