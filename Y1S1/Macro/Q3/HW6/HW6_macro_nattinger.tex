% !TEX TS-program = pdflatex
% !TEX encoding = UTF-8 Unicode

% This is a simple template for a LaTeX document using the "article" class.
% See "book", "report", "letter" for other types of document.

\documentclass[11pt]{article} % use larger type; default would be 10pt

\usepackage[utf8]{inputenc} % set input encoding (not needed with XeLaTeX)

%%% PAGE DIMENSIONS
\usepackage{geometry} % to change the page dimensions
\geometry{a4paper} % or letterpaper (US) or a5paper or....
\geometry{margin=1in} 
\usepackage{graphicx} % support the \includegraphics command and options

\usepackage{amssymb}
\usepackage{amsmath}
%%% PACKAGES
\usepackage{booktabs} % for much better looking tables
\usepackage{array} % for better arrays (eg matrices) in maths
\usepackage{paralist} % very flexible & customisable lists (eg. enumerate/itemize, etc.)
\usepackage{verbatim} % adds environment for commenting out blocks of text & for better verbatim
\usepackage{subfig} % make it possible to include more than one captioned figure/table in a single float
% These packages are all incorporated in the memoir class to one degree or another...

%%% HEADERS & FOOTERS
\usepackage{fancyhdr} % This should be set AFTER setting up the page geometry
\pagestyle{fancy} % options: empty , plain , fancy
\renewcommand{\headrulewidth}{0pt} % customise the layout...
\lhead{}\chead{}\rhead{}
\lfoot{}\cfoot{\thepage}\rfoot{}

%%% SECTION TITLE APPEARANCE
\usepackage{sectsty}
\allsectionsfont{\sffamily\mdseries\upshape} % (See the fntguide.pdf for font help)
% (This matches ConTeXt defaults)

%%% ToC (table of contents) APPEARANCE
\usepackage[nottoc,notlof,notlot]{tocbibind} % Put the bibliography in the ToC
\usepackage[titles,subfigure]{tocloft} % Alter the style of the Table of Contents
\renewcommand{\cftsecfont}{\rmfamily\mdseries\upshape}
\renewcommand{\cftsecpagefont}{\rmfamily\mdseries\upshape} % No bold!
\usepackage{graphicx}
\graphicspath{ {./pings/} }

\usepackage{amsmath}
\DeclareMathOperator*{\argmax}{arg\,max}
\DeclareMathOperator*{\argmin}{arg\,min}

\newcount\colveccount
\newcommand*\colvec[1]{
        \global\colveccount#1
        \begin{pmatrix}
        \colvecnext
}
\def\colvecnext#1{
        #1
        \global\advance\colveccount-1
        \ifnum\colveccount>0
                \\
                \expandafter\colvecnext
        \else
                \end{pmatrix}
        \fi
}

%%% END Article customizations

%%% The "real" document content comes below...

\title{Macro PS5}
\author{Michael B. Nattinger\footnote{I worked on this assignment with my study group: Alex von Hafften, Andrew Smith, and Ryan Mather. I have also discussed problem(s) with Emily Case, Sarah Bass, Katherine Kwok, and Danny Edgel.}}

%\date{} % Activate to display a given date or no date (if empty),
         % otherwise the current date is printed 

\begin{document}
\maketitle
\section{Question 1}
The planner solves the following optimization problem:
\begin{align*}
\max_{x_t,\pi_t,i_t} \frac{1}{2}E\sum_{t=0}^{\infty}\beta^t (x_t^2 + \alpha \pi_t^2) \\
\text{s.t.} \sigma E_t\Delta x_{t+1} = i_t - E_t\pi_{t+1} - r_t^n,\\
\text{and } \pi_t = \kappa x_t + \beta E_t\pi_{t+1} + u_t
\end{align*}

We consider the primal approach:
\begin{align*}
\max_{x_t,\pi_t} \frac{1}{2}E\sum_{t=0}^{\infty}\beta^t (x_t^2 + \alpha \pi_t^2) \\
\text{and } \pi_t = \kappa x_t + \beta E_t\pi_{t+1} + u_t\\
\mathcal{L} = E\sum_{t=0}^{\infty}\left[(1/2)\beta^t (x_t^2 + \alpha \pi_t^2) + \lambda_t(\pi_t - \kappa x_t - \beta \pi_{t+1} - u_t) \right]
\end{align*}

\begin{align*}
\beta^t x_t &= \lambda_t k\\
\beta^t\alpha\pi_t + \lambda_t - \beta \lambda_{t-1} &= 0, t\geq 1\\
\beta^t\alpha\pi_t + \lambda_t &= 0, t\geq 1
\end{align*}
\begin{align*}
\alpha \kappa \pi_t + \Delta x_t &= 0, t\geq 1\\
\alpha \kappa \pi_0 + x_0 &= 0.
\end{align*}

In class, at this point we defined $\hat{p}_t := p_t - p_{-1}$. However, in this question we are asked about commitment with the timeless perspective. Following Woodford (1999), we define $p_{-1} = 0 \Rightarrow \hat{p}_t = p_t.$ Therefore, we can proceed just as we did in class without having to carry around the hats on $p_t$. Note that now, following class, we also $x_{-1}:=0$. Then, we can combine our above two equations into one that holds for all $t$:

\begin{align*}
\alpha \kappa \pi_t + \Delta x_t &= 0.
\end{align*}

Since the above holds for all $t$, it follows that $-\alpha \kappa p_t =  x_t$ for all $t$, which can be easily shown via induction.

We can plug this into our constraint, the NKPC curve:

\begin{align*}
p_t - p_{t-1} &= - \alpha \kappa^2 p_t + \beta E_tp_{t+1} - \beta p_t + u_t\\
-\beta E_{t} p_{t+1} &= (-1 - \alpha \kappa^2 - \beta )p_t+ p_{t-1} + u_t\\
\begin{pmatrix} -\beta & 0 \\ 0 & 1 \end{pmatrix}\colvec{2}{E_tp_{t+1}}{p_t} &= \begin{pmatrix} -1 - \alpha \kappa^2 - \beta & 1 \\ 1 & 0 \end{pmatrix}\colvec{2}{p_{t}}{p_{t-1}} + \colvec{2}{1}{0} u_t\\
\colvec{2}{E_tp_{t+1}}{p_t} &= \begin{pmatrix}-1/\beta & 0 \\ 0 & 1 \end{pmatrix}\begin{pmatrix} -1 - \alpha \kappa^2 - \beta & 1 \\ 1 & 0 \end{pmatrix}\colvec{2}{p_{t}}{p_{t-1}} + \begin{pmatrix}-1/\beta & 0 \\ 0 & 1 \end{pmatrix} \colvec{2}{1}{0}u_t\\
\colvec{2}{E_tp_{t+1}}{p_t} &= \begin{pmatrix} \frac{1}{\beta} + \frac{\alpha}{\beta} \kappa^2 + 1 & -\frac{1}{\beta} \\ 1 & 0 \end{pmatrix}\colvec{2}{p_{t}}{p_{t-1}} +  \colvec{2}{-\frac{1}{\beta}}{0}u_t
\end{align*}

We can find the eigenvalues of the matrix above, that satisfy the following:
\begin{align*}
-\beta\lambda^2 + (1+\alpha\kappa^2 + \beta)\lambda - 1 = 0
\end{align*}
This has two roots, one above and one below 1 in magnitude, corresponding to the fact that we have one state and one control variable. WLOG let $\lambda_1>1$. By using the quadratic formula and multiplying the roots you can easily show that $\lambda_1\lambda_2 = \beta^{-1}$.

We can now write the equation in the following way:

\begin{align*}
-\beta(1-\lambda_1L)(1-\lambda_2L)L^{-1}p_t &= u_t \\ %  
(\beta\lambda_1 - \beta L^{-1})(1-\lambda_2L)p_t &= u_t \\
(1-\beta\lambda_2L^{-1})(1-\lambda_2L)p_t &= \lambda_2 u_t\\
(1-\lambda_2L)p_t &= \lambda_2 (1-\beta\lambda_2L^{-1})^{-1}u_t
\end{align*}
We are given that $u_t\sim \text{iid}(\bar{u},\sigma^2)$. We then rewrite the above equation as the following:

\begin{align}
p_t &= \lambda_2 p_{t-1} + \lambda_2 E_{t}\sum_{j=0}^{\infty}(\beta\lambda_2)^ju_{t+j} \nonumber \\
p_t &=\lambda_2 p_{t-1} +\lambda_2\left(u_t + \bar{u}\frac{\beta\lambda_2}{1-\beta\lambda_2}\right), \label{pt} \\
x_t &= \lambda_2 x_{t-1} - \lambda_2 \alpha \kappa \left(u_t + \bar{u}\frac{\beta\lambda_2}{1-\beta\lambda_2}\right). \label{xt}
\end{align}

The equations (\ref{pt}),(\ref{xt}) determine the dynamics of the price level and output gap.
\section{Question 2}
Under discretion, we have that $\alpha \kappa \pi_t + x_t = 0$ in each period. We also know that NKPC holds:
\begin{align*}
\pi_t &= \kappa x_t + \beta E_t \pi_{t+1} + u_t\\
&= -\alpha\kappa^2 \pi_t + \beta E_t \pi_{t+1} + u_t\\
&=  \frac{\beta}{1+\alpha\kappa^2} E_t \pi_{t+1} + \frac{1}{1+\alpha\kappa^2}u_t \\
&= \frac{1}{1+\alpha \kappa^2}E_t\sum_{j=0}^{\infty} \left( \frac{\beta}{1+\alpha\kappa^2} \right)^{j} u_{t+j}\\
&= \frac{u_t}{1+\alpha\kappa^2} +  \frac{\beta}{1+\alpha\kappa^2} \frac{\bar{u}}{1- \frac{\beta}{1+\alpha\kappa^2}}\\
\pi_t &= \frac{u_t}{1+\alpha\kappa^2} + \frac{\beta\bar{u}}{1+\alpha\kappa^2 - \beta},\\
x_t &= -\alpha \kappa \frac{u_t}{1+\alpha\kappa^2} - \alpha \kappa \frac{\beta\bar{u}}{1+\alpha\kappa^2 - \beta}.
\end{align*}
\section{Question 3}
Under the inflation targeting rule, $\pi_t = 0$ and our NKPC curve states the following:
\begin{align*}
x_t &= -\frac{u_t}{\kappa}.
\end{align*}
\section{Question 4}
Under output targeting rule, $x_t = 0$ and our NKPC curve states the following:
\begin{align*}
\pi_t &= \beta E_t\pi_{t+1} + u_t\\
&=  E_t\sum_{j=0}^{\infty}\beta^{j}u_{t+j}\\
&= u_t + \frac{\beta\bar{u}}{1-\beta}.
\end{align*}

\section{Question 5}
We can consider the welfare implications of the two regimes to determine the optimal policy. Under inflation targeting, our welfare losses are the following:
\begin{align*}
\mathcal{W}^{\pi} &= \frac{1}{2}E\sum_{t=0}^{\infty}\beta^t \frac{u_t^2}{\kappa^2}\\
&= \frac{1}{2\kappa^2}\sum_{t=0}^{\infty}\beta^t E[u_t^2]\\
&= \frac{\bar{u}^2 + \sigma^2}{2\kappa^2(1-\beta)}.
\end{align*}

Under discretion, our welfare losses are the following:
\begin{align*}
\mathcal{W}^{D} &= \frac{\alpha(1+\alpha\kappa^2)}{2}E\sum_{t=0}^{\infty}\beta^t \left(  \frac{u_t}{1+\alpha\kappa^2} + \frac{\beta\bar{u}}{1+\alpha\kappa^2 - \beta}\right)^2\\
&=  \frac{\alpha(1+\alpha\kappa^2)}{2} E\sum_{t=0}^{\infty}\beta^t \left[ \frac{u_t^2}{(1+\alpha \kappa^2)^2} + \frac{2\beta\bar{u}u_t}{(1+\alpha\kappa^2)(1+\alpha\kappa^2 - \beta)} + \frac{\beta^2\bar{u}^2}{(1+\alpha\kappa^2 - \beta)^2} \right] \\
&=  \frac{\alpha(1+\alpha\kappa^2)}{2} \sum_{t=0}^{\infty}\beta^t \left[ \frac{\bar{u}^2 + \sigma^2}{(1+\alpha \kappa^2)^2} + \frac{2\beta\bar{u}^2}{(1+\alpha\kappa^2)(1+\alpha\kappa^2 - \beta)} + \frac{\beta^2\bar{u}^2}{(1+\alpha\kappa^2 - \beta)^2} \right] \\
&=  \frac{\alpha(1+\alpha\kappa^2)}{2(1-\beta)} \left[\left( \frac{1}{(1+\alpha\kappa^2)^2} + \frac{2\beta}{(1+\alpha\kappa^2)(1+\alpha\kappa^2 - \beta)}  +  \frac{\beta^2}{(1+\alpha\kappa^2 - \beta)^2} \right)\bar{u}^2 + \frac{\sigma^2}{(1+\alpha\kappa^2)^2} \right] \\
&=  \frac{\alpha}{2(1-\beta)} \left[\left( \frac{1}{1+\alpha\kappa^2} + \frac{2\beta}{1+\alpha\kappa^2 - \beta}  +  \frac{(1+\alpha \kappa^2)\beta^2}{(1+\alpha\kappa^2 - \beta)^2} \right)\bar{u}^2 + \frac{\sigma^2}{(1+\alpha\kappa^2)} \right]
\end{align*}
\section{Question 6}
\section{Question 7}
\end{document}
