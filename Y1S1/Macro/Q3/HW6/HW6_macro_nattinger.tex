% !TEX TS-program = pdflatex
% !TEX encoding = UTF-8 Unicode

% This is a simple template for a LaTeX document using the "article" class.
% See "book", "report", "letter" for other types of document.

\documentclass[11pt]{article} % use larger type; default would be 10pt

\usepackage[utf8]{inputenc} % set input encoding (not needed with XeLaTeX)

%%% PAGE DIMENSIONS
\usepackage{geometry} % to change the page dimensions
\geometry{a4paper} % or letterpaper (US) or a5paper or....
\geometry{margin=1in} 
\usepackage{graphicx} % support the \includegraphics command and options

\usepackage{amssymb}
\usepackage{amsmath}
%%% PACKAGES
\usepackage{booktabs} % for much better looking tables
\usepackage{array} % for better arrays (eg matrices) in maths
\usepackage{paralist} % very flexible & customisable lists (eg. enumerate/itemize, etc.)
\usepackage{verbatim} % adds environment for commenting out blocks of text & for better verbatim
\usepackage{subfig} % make it possible to include more than one captioned figure/table in a single float
% These packages are all incorporated in the memoir class to one degree or another...

%%% HEADERS & FOOTERS
\usepackage{fancyhdr} % This should be set AFTER setting up the page geometry
\pagestyle{fancy} % options: empty , plain , fancy
\renewcommand{\headrulewidth}{0pt} % customise the layout...
\lhead{}\chead{}\rhead{}
\lfoot{}\cfoot{\thepage}\rfoot{}

%%% SECTION TITLE APPEARANCE
\usepackage{sectsty}
\allsectionsfont{\sffamily\mdseries\upshape} % (See the fntguide.pdf for font help)
% (This matches ConTeXt defaults)

%%% ToC (table of contents) APPEARANCE
\usepackage[nottoc,notlof,notlot]{tocbibind} % Put the bibliography in the ToC
\usepackage[titles,subfigure]{tocloft} % Alter the style of the Table of Contents
\renewcommand{\cftsecfont}{\rmfamily\mdseries\upshape}
\renewcommand{\cftsecpagefont}{\rmfamily\mdseries\upshape} % No bold!
\usepackage{graphicx}
\graphicspath{ {./pings/} }

\usepackage{amsmath}
\DeclareMathOperator*{\argmax}{arg\,max}
\DeclareMathOperator*{\argmin}{arg\,min}

\newcount\colveccount
\newcommand*\colvec[1]{
        \global\colveccount#1
        \begin{pmatrix}
        \colvecnext
}
\def\colvecnext#1{
        #1
        \global\advance\colveccount-1
        \ifnum\colveccount>0
                \\
                \expandafter\colvecnext
        \else
                \end{pmatrix}
        \fi
}

%%% END Article customizations

%%% The "real" document content comes below...

\title{Macro PS5}
\author{Michael B. Nattinger\footnote{I worked on this assignment with my study group: Alex von Hafften, Andrew Smith, and Ryan Mather. I have also discussed problem(s) with Emily Case, Sarah Bass, Katherine Kwok, and Danny Edgel.}}

%\date{} % Activate to display a given date or no date (if empty),
         % otherwise the current date is printed 

\begin{document}
\maketitle
\section{Question 1}
The planner solves the following optimization problem:
\begin{align*}
\max_{x_t,\pi_t,i_t} \frac{1}{2}E\sum_{t=0}^{\infty}\beta^t (x_t^2 + \alpha \pi_t^2) \\
\text{s.t.} \sigma E_t\Delta x_{t+1} = i_t - E_t\pi_{t+1} - r_t^n,\\
\text{and } \pi_t = \kappa x_t + \beta E_t\pi_{t+1} + u_t
\end{align*}

We consider the primal approach:
\begin{align*}
\max_{x_t,\pi_t} \frac{1}{2}E\sum_{t=0}^{\infty}\beta^t (x_t^2 + \alpha \pi_t^2) \\
\text{and } \pi_t = \kappa x_t + \beta E_t\pi_{t+1} + u_t\\
\mathcal{L} = E\sum_{t=0}^{\infty}\left[(1/2)\beta^t (x_t^2 + \alpha \pi_t^2) + \lambda_t(\pi_t - \kappa x_t - \beta \pi_{t+1} - u_t) \right]
\end{align*}

\begin{align*}
\beta^t x_t &= \lambda_t k\\
\beta^t\alpha\pi_t + \lambda_t - \beta \lambda_{t-1} &= 0, t\geq 1\\
\beta^t\alpha\pi_t + \lambda_t &= 0, t\geq 1
\end{align*}
\begin{align*}
\alpha \kappa \pi_t + \Delta x_t &= 0, t\geq 1\\
\alpha \kappa \pi_0 + x_0 &= 0.
\end{align*}

In class, at this point we defined $\hat{p}_t := p_t - p_{-1}$. However, in this question we are asked about commitment with the timeless perspective. Following Woodford (1999), we define $p_{-1} = 0 \Rightarrow \hat{p}_t = p_t.$ Therefore, we can proceed just as we did in class without having to carry around the hats on $p_t$. Note that now, following class, we also $x_{-1}:=0$. Then, we can combine our above two equations into one that holds for all $t$:

\begin{align*}
\alpha \kappa \pi_t + \Delta x_t &= 0.
\end{align*}

Since the above holds for all $t$, it follows that $-\alpha \kappa p_t =  x_t$ for all $t$, which can be easily shown via induction.

We can plug this into our constraint, the NKPC curve:

\begin{align*}
p_t - p_{t-1} &= - \alpha \kappa^2 p_t + \beta E_tp_{t+1} - \beta p_t + u_t\\
-\beta E_{t} p_{t+1} &= (-1 - \alpha \kappa^2 - \beta )p_t+ p_{t-1} + u_t\\
\begin{pmatrix} -\beta & 0 \\ 0 & 1 \end{pmatrix}\colvec{2}{E_tp_{t+1}}{p_t} &= \begin{pmatrix} -1 - \alpha \kappa^2 - \beta & 1 \\ 1 & 0 \end{pmatrix}\colvec{2}{p_{t}}{p_{t-1}} + \colvec{2}{1}{0} u_t\\
\colvec{2}{E_tp_{t+1}}{p_t} &= \begin{pmatrix}-1/\beta & 0 \\ 0 & 1 \end{pmatrix}\begin{pmatrix} -1 - \alpha \kappa^2 - \beta & 1 \\ 1 & 0 \end{pmatrix}\colvec{2}{p_{t}}{p_{t-1}} + \begin{pmatrix}-1/\beta & 0 \\ 0 & 1 \end{pmatrix} \colvec{2}{1}{0}u_t\\
\colvec{2}{E_tp_{t+1}}{p_t} &= \begin{pmatrix} \frac{1}{\beta} + \frac{\alpha}{\beta} \kappa^2 + 1 & -\frac{1}{\beta} \\ 1 & 0 \end{pmatrix}\colvec{2}{p_{t}}{p_{t-1}} +  \colvec{2}{-\frac{1}{\beta}}{0}u_t
\end{align*}

We can find the eigenvalues of the matrix above, that satisfy the following:
\begin{align*}
-\beta\lambda^2 + (1+\alpha\kappa^2 + \beta)\lambda - 1 = 0
\end{align*}
This has two roots, one above and one below 1 in magnitude, corresponding to the fact that we have one state and one control variable. WLOG let $\lambda_1>1$. By using the quadratic formula and multiplying the roots you can easily show that $\lambda_1\lambda_2 = \beta^{-1}$.

We can now write the equation in the following way:

\begin{align*}
-\beta(1-\lambda_1L)(1-\lambda_2L)L^{-1}p_t &= u_t \\ %  
(\beta\lambda_1 - \beta L^{-1})(1-\lambda_2L)p_t &= u_t \\
(1-\beta\lambda_2L^{-1})(1-\lambda_2L)p_t &= \lambda_2 u_t\\
(1-\lambda_2L)p_t &= \lambda_2 (1-\beta\lambda_2L^{-1})^{-1}u_t
\end{align*}
We are given that $u_t\sim \text{iid}(\bar{u},\sigma^2)$. We then rewrite the above equation as the following:

\begin{align}
p_t &= \lambda_2 p_{t-1} + \lambda_2 E_{t}\sum_{j=0}^{\infty}(\beta\lambda_2)^ju_{t+j} \nonumber \\
p_t &=\lambda_2 p_{t-1} +\lambda_2\left(u_t + \bar{u}\frac{\beta\lambda_2}{1-\beta\lambda_2}\right), \label{pt} \\
x_t &= \lambda_2 x_{t-1} - \lambda_2 \alpha \kappa \left(u_t + \bar{u}\frac{\beta\lambda_2}{1-\beta\lambda_2}\right). \label{xt}
\end{align}

The equations (\ref{pt}),(\ref{xt}) determine the dynamics of the price level and output gap.
\section{Question 2}
Under discretion, we have that $\alpha \kappa \pi_t + x_t = 0$ in each period. We also know that NKPC holds:
\begin{align*}
\pi_t &= \kappa x_t + \beta E_t \pi_{t+1} + u_t\\
&= -\alpha\kappa^2 \pi_t + \beta E_t \pi_{t+1} + u_t\\
&=  \frac{\beta}{1+\alpha\kappa^2} E_t \pi_{t+1} + \frac{1}{1+\alpha\kappa^2}u_t \\
&= \frac{1}{1+\alpha \kappa^2}E_t\sum_{j=0}^{\infty} \left( \frac{\beta}{1+\alpha\kappa^2} \right)^{j} u_{t+j}\\
&= \frac{u_t}{1+\alpha\kappa^2} +  \frac{\beta}{(1+\alpha\kappa^2)^2} \frac{\bar{u}}{1- \frac{\beta}{1+\alpha\kappa^2}}\\
\pi_t &= \frac{u_t}{1+\alpha\kappa^2} + \frac{\beta\bar{u}}{(1+\alpha\kappa^2)(1+\alpha\kappa^2 - \beta)},\\
x_t &= -\alpha \kappa \frac{u_t}{1+\alpha\kappa^2} - \alpha \kappa \frac{\beta\bar{u}}{(1+\alpha\kappa^2)(1+\alpha\kappa^2 - \beta)}.
\end{align*}
\section{Question 3}
Under the inflation targeting rule, $\pi_t = 0$ and our NKPC curve states the following:
\begin{align*}
x_t &= -\frac{u_t}{\kappa}.
\end{align*}
\section{Question 4}
Under output targeting rule, $x_t = 0$ and our NKPC curve states the following:
\begin{align*}
\pi_t &= \beta E_t\pi_{t+1} + u_t\\
&=  E_t\sum_{j=0}^{\infty}\beta^{j}u_{t+j}\\
&= u_t + \frac{\beta\bar{u}}{1-\beta}.
\end{align*}

\section{Question 5}
We can consider the welfare implications of the two regimes to determine the optimal policy. Under inflation targeting, our welfare losses are the following:
\begin{align*}
\mathcal{W}^{\pi} &= \frac{1}{2}E\sum_{t=0}^{\infty}\beta^t \frac{u_t^2}{\kappa^2}\\
&= \frac{1}{2\kappa^2}\sum_{t=0}^{\infty}\beta^t E[u_t^2]\\
&= \frac{\bar{u}^2 + \sigma^2}{2\kappa^2(1-\beta)}.
\end{align*}

Under discretion, our welfare losses are the following:
\begin{align*}
\mathcal{W}^{D} &= \frac{\alpha(1+\alpha\kappa^2)}{2}E\sum_{t=0}^{\infty}\beta^t \left(  \frac{u_t}{1+\alpha\kappa^2} + \frac{\beta\bar{u}}{(1+\alpha\kappa^2)(1+\alpha\kappa^2 - \beta)}\right)^2\\
&=  \frac{\alpha(1+\alpha\kappa^2)}{2} E\sum_{t=0}^{\infty}\beta^t \left[ \frac{u_t^2}{(1+\alpha \kappa^2)^2} + \frac{2\beta\bar{u}u_t}{(1+\alpha\kappa^2)^2(1+\alpha\kappa^2 - \beta)} + \frac{\beta^2\bar{u}^2}{(1+\alpha\kappa^2)^2(1+\alpha\kappa^2 - \beta)^2} \right] \\
&=  \frac{\alpha(1+\alpha\kappa^2)}{2} \sum_{t=0}^{\infty}\beta^t \left[ \frac{\bar{u}^2 + \sigma^2}{(1+\alpha \kappa^2)^2} + \frac{2\beta\bar{u}^2}{(1+\alpha\kappa^2)^2(1+\alpha\kappa^2 - \beta)} + \frac{\beta^2\bar{u}^2}{(1+\alpha\kappa^2)^2(1+\alpha\kappa^2 - \beta)^2} \right] \\
&=  \frac{\alpha}{2(1-\beta)(1+\alpha\kappa^2)} \left[\left( 1 + \frac{2\beta}{(1+\alpha\kappa^2 - \beta)}  +  \frac{\beta^2}{(1+\alpha\kappa^2 - \beta)^2} \right)\bar{u}^2 + \sigma^2 \right] \\
&=  \frac{\alpha}{2(1-\beta)(1+\alpha\kappa^2)} \left[\left(   \frac{1 + 2\alpha\kappa^2 + \alpha^2\kappa^4 }{(1+\alpha\kappa^2 - \beta)^2} \right)\bar{u}^2 + \sigma^2 \right] \\
\end{align*}

It is optimal to adopt inflation targeting instead of discretionary policy if the welfare losses under inflation targeting are less than the welfare losses under discretion:
\begin{align*}
\mathcal{W}^{\pi} &< \mathcal{W}^{D} \\
\frac{\bar{u}^2 + \sigma^2}{2\kappa^2(1-\beta)}&< \frac{\alpha}{2(1-\beta)(1+\alpha\kappa^2)} \left[\left(   \frac{1 + 2\alpha\kappa^2 + \alpha^2\kappa^4 }{(1+\alpha\kappa^2 - \beta)^2} \right)\bar{u}^2 + \sigma^2 \right]\\
\frac{\bar{u}^2 + \sigma^2}{2\kappa^2}&< \frac{\alpha}{2(1+\alpha\kappa^2)} \left[\left(   \frac{1 + 2\alpha\kappa^2 + \alpha^2\kappa^4 }{(1+\alpha\kappa^2 - \beta)^2} \right)\bar{u}^2 + \sigma^2 \right]
\end{align*}

Taking the limit that $\beta \approx 1$, $\mathcal{W}^{\pi} < \mathcal{W}^{D} $ leads to the following:
\begin{align*}
\frac{\bar{u}^2 + \sigma^2}{2\kappa^2}&< \frac{\alpha}{2(1+\alpha\kappa^2)} \left[\left(   \frac{1 + 2\alpha\kappa^2 + \alpha^2\kappa^4 }{(\alpha\kappa^2 )^2} \right)\bar{u}^2 + \sigma^2 \right]\\
\left(1+\alpha\kappa^2 - \alpha\kappa^2  \right)\sigma^2 &< \left( \frac{1 + 2\alpha\kappa^2 + \alpha^2\kappa^4 }{\alpha\kappa^2 }  - 1-\alpha\kappa^2\right)\bar{u}^2 \\
\sigma^2 &< \left( \frac{1 + \alpha\kappa^2 }{\alpha\kappa^2 } \right)\bar{u}^2
%\frac{\bar{u}^2 + \sigma^2}{2\kappa^2}&< \frac{\alpha}{2} \left[\left( \frac{1}{1+\alpha\kappa^2} +  \frac{1 + 3\alpha\kappa^2}{\alpha^2\kappa^4} \right)\bar{u}^2 + \frac{\sigma^2}{(1+\alpha\kappa^2)} \right] \\
%\left[ \frac{1}{2\kappa^2} - \frac{\alpha}{2(1+\alpha\kappa^2)} - \frac{1+3\alpha\kappa^2}{\alpha^2\kappa^4}\right] \bar{u}^2 &< \left[ \frac{\alpha}{2(1+\alpha\kappa)} - \frac{1}{2\kappa^2} \right] \sigma^2 \\
%\left[ \frac{\alpha^2\kappa^2 -2-6\alpha\kappa^2}{2\alpha^2\kappa^4} - \frac{\alpha}{2(1+\alpha\kappa^2)} \right] \bar{u}^2 &< \left[ \frac{\alpha\kappa^2 - 2-2\alpha\kappa}{2(1+\alpha\kappa)\kappa^2} \right] \sigma^2 \\
%\left[ \frac{\alpha^2\kappa^2 -2-8\alpha\kappa^2 - 6\alpha^2\kappa^4}{2(1+\alpha\kappa^2)\alpha^2\kappa^4} \right] \bar{u}^2 &< \left[ \frac{\alpha\kappa^2 - 2-2\alpha\kappa}{2(1+\alpha\kappa)\kappa^2} \right] \sigma^2 \\
%\left[ \frac{\alpha^2\kappa^2 -2-8\alpha\kappa^2 - 6\alpha^2\kappa^4}{(1+\alpha\kappa^2)\alpha^2\kappa^2} \right] \bar{u}^2 &< \left[ \frac{\alpha\kappa^2 - 2-2\alpha\kappa}{(1+\alpha\kappa)} \right] \sigma^2
\end{align*}
%\begin{align*}
%\left[ \alpha^2\kappa^2 -2 - 8\alpha\kappa^2 - 6\alpha^2\kappa^4 + \alpha^3 \kappa^3 - 2\alpha\kappa - 8\alpha^2\kappa^3 - 6\alpha^3\kappa^5 \right]\bar{u}^2 &<\left[ -2\alpha^2\kappa^4 - 3\alpha^3\kappa^4 - \alpha^4\kappa^6 \right]\\
%\left[ -\alpha^2\kappa^2 +2 + 8\alpha\kappa^2 + 6\alpha^2\kappa^4 - \alpha^3 \kappa^3 + 2\alpha\kappa + 8\alpha^2\kappa^3 + 6\alpha^3\kappa^5 \right]\bar{u}^2 &>\left[ 2\alpha^2\kappa^4 + 3\alpha^3\kappa^4 + \alpha^4\kappa^6 \right]\sigma^2\\
%\mathcal{K} \bar{u}^2 > \sigma^2
%\end{align*}
So, inflation targeting is optimal relative to discretionary policy if the variance of the markup shocks is less than some (positive) constant multiplied by the square of their expectation.

Inflation targeting kills off the welfare losses from the misallocation effects of the markup shocks, which come from prices changing (i.e. inflation). If the mean of these markup shocks are far away from 0, relative to their variance,  the distortionary effects of the markup shocks are large and a persistent problem for the economy, and by killing off this effect of the shocks the planner makes the households better off than if the planner tried to simultaneously address both the welfare losses from working suboptimally and the misallocation effects under discretion. Under discretion, markup shocks that are far away from 0 cause persistent inflation, and this persistent inflation is the cause of the misallocation effects. This is why inflation targeting is optimal in this case. If instead the shocks were approximately mean $0$ then the misallocation effects are less important (because inflation would approximately average out to 0 over time under discretion) and so the planner would not be better off by targeting inflation in this case. % maybe rewrite this

\section{Question 6}
Our welfare losses from output targeting are as follows:
\begin{align*}
\mathcal{W}^{x} &= \frac{1}{2}E\sum_{t=0}^{\infty} \beta^t \alpha\left( u_t + \frac{\beta\bar{u}}{1-\beta} \right)^2\\
&= \frac{\alpha}{2}E\sum_{t=0}^{\infty} \beta^t \left( u_t^2 +2 \frac{\beta\bar{u}u_t}{1-\beta} + \frac{\beta^2\bar{u}^2}{1-\beta + \beta^2} \right) \\
&= \frac{\alpha}{2}\sum_{t=0}^{\infty} \beta^t \left( \bar{u}^2 + \sigma^2 +2 \frac{\beta\bar{u}^2}{1-\beta} + \frac{\beta^2\bar{u}^2}{1-\beta + \beta^2} \right) \\
&=\frac{\alpha}{2(1-\beta)} \left( \bar{u}^2 + \sigma^2 +2 \frac{\beta\bar{u}^2}{1-\beta} + \frac{\beta^2\bar{u}^2}{1-\beta + \beta^2} \right) \\
&=\frac{\alpha}{2(1-\beta)}\sigma^2 + \frac{\alpha}{2}\frac{1 + \beta }{(1-\beta)^3}\bar{u}^2.
\end{align*}

Output targeting is strictly preferred to inflation targeting when $\mathcal{W}^{x} < \mathcal{W}^{\pi}$:
\begin{align*}
\mathcal{W}^{x} &< \mathcal{W}^{\pi}\\
\frac{\alpha}{2(1-\beta)}\sigma^2 + \frac{\alpha}{2}\frac{1 + \beta }{(1-\beta)^3}\bar{u}^2 &< \frac{\bar{u}^2 + \sigma^2}{2\kappa^2(1-\beta)}\\
\frac{\alpha}{2}\sigma^2 + \frac{\alpha}{2}\frac{1 + \beta }{(1-\beta)^2}\bar{u}^2 &< \frac{\bar{u}^2 + \sigma^2}{2\kappa^2} \\
\frac{\alpha \kappa^2 - 1}{2 \kappa^2}\sigma^2 &< \frac{(1-\beta)^2-\alpha(1+\beta)\kappa^2 }{2(1-\beta)^2\kappa^2} \bar{u}^2\\
(1-\beta)^2(\alpha \kappa^2 - 1)\sigma^2 &< (1-\beta)^2-\alpha(1+\beta)\kappa^2  \bar{u}^2
\end{align*}

Making the assumption that $\beta \approx 1 $, $\mathcal{W}^{x} < \mathcal{W}^{\pi} \leadsto 0< -2\alpha\kappa^2  \bar{u}^2 \leq0$. This implies that $0<0$ which is a contradiction. Therefore, if $\beta \approx 1$, output targeting is not strictly preferred to inflation targeting.
\section{Question 7}
We now assume no markup shocks, and a Taylor rule for monetary policy. The NKIS and NKPC curves are thus the following:
\begin{align*}
\sigma E_t \Delta x_{t+1} &= \phi \pi_t - E_t\pi_{t+1} - r_t^n\\
\pi_t &= \kappa x_t + \beta E_t\pi_{t+1}
\end{align*}

From the NKPC curve, we have that $E_{t}\pi_{t+1} = \frac{\pi_t-\kappa x_t }{\beta}$. Plugging this into our NKIS curve,
\begin{align*}
\sigma E_t \Delta x_{t+1} &= \phi \pi_t -\frac{\pi_t-\kappa x_t }{\beta} - r_t^n\\
\pi_t &= \frac{\sigma E_t\Delta x_{t+1} + \kappa x_t/\beta - r_t^n}{\phi - \beta^{-1}}\\
&\rightarrow_{\phi \rightarrow \infty} 0
\end{align*}
 Plugging this back into our NKPC curve, we similarly find that $x_t = 0$. This matches our first best allocation under no markup shocks, as equations (\ref{pt}),(\ref{xt}), along with our initial conditions $x_{-1} = p_{-1} = 0$ and no markup shocks, show that $x_t = p_t = 0 \forall t$.
%Note that $i_t = r_t^n + \phi \pi_t$. Note also that as $\phi \rightarrow \infty, i_t \rightarrow r_t^n$ so $\pi_t \rightarrow 0 \Rightarrow x_t,p_t \rightarrow 0$. This matches our first-best allocation where, by equations (\ref{pt}),(\ref{xt}), our initial conditions $x_{-1} = 0 = p_{-1}$, and that $u_t = \bar{u} = 0$, $x_t = 0 = p_t = \pi_t \forall t$.

In class, Dima asked us to solve this using Blanchard-Kahn. Rewriting, and setting up for Blanchard-Kahn, we write the following:
\begin{align*}
\begin{pmatrix} 1 & \sigma \\ \beta  & 0 \end{pmatrix} \colvec{2}{E_t\pi_{t+1}}{E_tx_{t+1}} &=  \begin{pmatrix} \phi & \sigma \\ 1 & -\kappa \end{pmatrix}\colvec{2}{\pi_{t}}{x_{t}} + \colvec{2}{-1}{0}r_{t}^n \\
 \colvec{2}{E_t\pi_{t+1}}{E_tx_{t+1}} &=  \begin{pmatrix} 0 & 1/\beta \\ 1/\sigma  & -1/(\beta\sigma) \end{pmatrix} \begin{pmatrix} \phi & \sigma \\ 1 & -\kappa \end{pmatrix}\colvec{2}{\pi_{t}}{x_{t}} +\begin{pmatrix} 0 & 1/\beta \\ 1/\sigma  & -1/(\beta\sigma) \end{pmatrix} \colvec{2}{-1}{0}r_{t}^n \\
 \colvec{2}{E_t\pi_{t+1}}{E_tx_{t+1}} &=  \begin{pmatrix} 1/\beta & -\kappa/\beta \\ \phi/\sigma - 1/(\beta\sigma) & 1+\kappa/(\beta\sigma) \end{pmatrix}\colvec{2}{\pi_{t}}{x_{t}} +\colvec{2}{0}{-1/\sigma}r_{t}^n
\end{align*}

We can find the eigenvalues of the square matrix above:
\begin{align*}
0 &=\lambda^2 - \left( \frac{1}{\beta} + 1 + \frac{\kappa}{\beta\sigma}\right)\lambda +\frac{\phi\kappa}{\beta\sigma} + \frac{1}{\beta}\\
\lambda &= \frac{1}{2}\left( \frac{\sigma+\beta\sigma + \kappa}{\beta\sigma}\right) \pm \frac{1}{2}\sqrt{\left( \frac{\sigma+\beta\sigma + \kappa}{\beta\sigma}\right)^2 - 4\left( \frac{\sigma + \phi\kappa}{\beta\sigma}  \right)} \\
&=  \frac{1}{2}\left( \frac{\sigma+\beta\sigma + \kappa}{\beta\sigma}\right) \pm \frac{1}{2}\sqrt{\frac{\sigma^2+ 2\kappa\sigma +\beta^2\sigma^2 +2 \kappa\beta\sigma  + \kappa^2 - 2\beta\sigma^2 - 4\phi\kappa\beta\sigma}{\beta^2\sigma^2}   }
\end{align*}

Letting $\beta \approx 1$,
\begin{align*}
\lambda &= 1+\frac{ \kappa}{2\sigma} \pm \frac{1}{2\sigma}\sqrt{ 4\kappa\sigma   + \kappa^2  - 4\phi\kappa\sigma }
\end{align*}

With our eigenvalues $\Lambda$ and eigenvector matrix $Q$, which I compute in Matlab due to time constraints,  we can write the system as follows:
\begin{align*}
E_t\colvec{2}{\pi_{t+1}}{x_{t+1}} &= Q\Lambda Q^{-1}\colvec{2}{\pi_{t}}{x_{t}} + \colvec{2}{0}{-1/\sigma}r_t^n\\
E_tY_{t+1} &= \Lambda Y_t + C r_{t}^n 
\end{align*}
We have two control variables, and thus two eigenvalues greater than one, and two equations which must be satisfied. For each row $i$ of $Y$:
\begin{align*}
E_tY_{i,t+1} &= \lambda_i Y_{i,t} + C_ir_t^n\\
Y_{i,t} &= \lambda_i^{-1}E_tY_{i,t+1} - \lambda_i^{-1}C_ir_t^n\\
Y_{i,t} &= -\lambda_i^{-1}C_i E_t\sum_{j=1}^{\infty}\lambda_i^{-j}r_{t+j}^n
\end{align*}

We now have two ways to proceed: bound each $r_{t+j}^n$ (as we know it is a linearized variable) and show that the infinite sum is finite, and achieve bounds of $|Y_{i,t}|$ which go to $0$ in the limit or find an explicit form for $Y_{i,t}$ by making an assumption on $a_t$ and showing that $Y_{i,t} \rightarrow 0$ in the limit. I do the former here and the latter in a footnote.\footnote{Under flexible prices, the euler equation holds: $\sigma E_t \Delta c_{t+1} = i_t - E_t \pi_{t+1} \Rightarrow \sigma E_t\Delta a_{t+1} = r_t^n $. Assuming $a_t$ follow ar(1) processes,  $a_{t+1} = \rho a_t + \epsilon_{t+1}$ where $\epsilon_{t+1} \sim_{iid} N(0,\sigma_a^2), \sigma E_t\Delta a_{t+1} = \sigma(\rho - 1)a_t$. Plugging this into our equation above we get: $Y_{i,t} = -\lambda_i^{-1}C_i \sum_{j=1}^{\infty}\lambda_i^{-j} \sigma(\rho - 1)a_{t+j} = -\lambda_i^{-1}C_i  \sigma(\rho - 1)(1-\lambda_i^{-1}\rho)^{-1}a_{t}$. Noting that the $|\lambda_i^{-1}C_i|$ terms go to infinity as $\phi \rightarrow \infty$, $Y_{it}\rightarrow 0$ as $\phi \rightarrow \infty \forall i$ so $\pi_t,x_t \rightarrow 0$ in the limit. This matches our first best allocation under no markup shocks, as equations (\ref{pt}),(\ref{xt}), along with our initial conditions $x_{-1} = p_{-1} = 0$ and no markup shocks, show that $x_t = p_t = 0 \forall t$.}

As $r_t^n$ is linearized, we can safely assume $|r_t^n|<1$. Therefore,
\begin{align*}
|Y_{i,t}| &< \frac{|\lambda_i^{-1}C_i|}{1-|\lambda_i|^{-1}}.
\end{align*}

From Matlab, we have the following values for $\lambda_i^{-1}C_i$:
\begin{align*}
\lambda_1^{-1}C_1 &= -(\kappa + \sqrt{(\kappa (-4\phi\sigma^2 + 4\sigma + \kappa))})/((\sqrt{\kappa(-4\phi\sigma^2 + 4\sigma +\kappa)}(\kappa + 2\sigma + \sqrt{\kappa(-4\phi\sigma^2 +4\sigma + \kappa)}))) \\
\lambda_2^{-1}C_2 &= (\kappa - \sqrt{(\kappa (-4\phi\sigma^2 + 4\sigma + \kappa))})/((\sqrt{\kappa(-4\phi\sigma^2 + 4\sigma +\kappa)}(\kappa + 2\sigma - \sqrt{\kappa(-4\phi\sigma^2 +4\sigma + \kappa)})))
\end{align*}


Noting that the $|\lambda_i^{-1}C_i|$ terms go to zero as $\phi \rightarrow \infty$, $Y_{it}\rightarrow 0$ as $\phi \rightarrow \infty \forall i$ so $\pi_t,x_t \rightarrow 0$ in the limit.\footnote{This is because $Q^{-1}$ does not go to zero in the limit. From Matlab,\\ $Q^{-1} = \begin{pmatrix}(\phi\sigma - 1)/\sqrt{\kappa(-4\phi\sigma^2 + 4\sigma + \kappa)} & (\kappa + \sqrt{\kappa(-4\phi\sigma^2 + 4\sigma + \kappa)})/(2\sqrt{\kappa(-4\phi\sigma^2 + 4\sigma + \kappa)}) \\ -(\phi\sigma - 1)/\sqrt{\kappa(-4\phi\sigma^2 + 4\sigma + \kappa)} & -(\kappa - \sqrt{\kappa(-4\phi\sigma^2 + 4\sigma + \kappa)})/(2\sqrt{\kappa(-4\phi\sigma^2 + 4\sigma + \kappa)}) \end{pmatrix}\\ \rightarrow_{\phi\rightarrow\infty} \begin{pmatrix}0 &1/2 \\0 & 1/2  \end{pmatrix}$. Therefore, $x_t = 0 \Rightarrow \pi_t = 0.$} This matches our first best allocation under no markup shocks, as equations (\ref{pt}),(\ref{xt}), along with our initial conditions $x_{-1} = p_{-1} = 0$ and no markup shocks, show that $x_t = p_t = 0 \forall t$.
\end{document}
