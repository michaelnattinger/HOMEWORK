% !TEX TS-program = pdflatex
% !TEX encoding = UTF-8 Unicode

% This is a simple template for a LaTeX document using the "article" class.
% See "book", "report", "letter" for other types of document.

\documentclass[11pt]{article} % use larger type; default would be 10pt

\usepackage[utf8]{inputenc} % set input encoding (not needed with XeLaTeX)

%%% PAGE DIMENSIONS
\usepackage{geometry} % to change the page dimensions
\geometry{a4paper} % or letterpaper (US) or a5paper or....

\usepackage{graphicx} % support the \includegraphics command and options

\usepackage{amssymb}
\usepackage{amsmath}
%%% PACKAGES
\usepackage{booktabs} % for much better looking tables
\usepackage{array} % for better arrays (eg matrices) in maths
\usepackage{paralist} % very flexible & customisable lists (eg. enumerate/itemize, etc.)
\usepackage{verbatim} % adds environment for commenting out blocks of text & for better verbatim
\usepackage{subfig} % make it possible to include more than one captioned figure/table in a single float
% These packages are all incorporated in the memoir class to one degree or another...

%%% HEADERS & FOOTERS
\usepackage{fancyhdr} % This should be set AFTER setting up the page geometry
\pagestyle{fancy} % options: empty , plain , fancy
\renewcommand{\headrulewidth}{0pt} % customise the layout...
\lhead{}\chead{}\rhead{}
\lfoot{}\cfoot{\thepage}\rfoot{}

%%% SECTION TITLE APPEARANCE
\usepackage{sectsty}
\allsectionsfont{\sffamily\mdseries\upshape} % (See the fntguide.pdf for font help)
% (This matches ConTeXt defaults)

%%% ToC (table of contents) APPEARANCE
\usepackage[nottoc,notlof,notlot]{tocbibind} % Put the bibliography in the ToC
\usepackage[titles,subfigure]{tocloft} % Alter the style of the Table of Contents
\renewcommand{\cftsecfont}{\rmfamily\mdseries\upshape}
\renewcommand{\cftsecpagefont}{\rmfamily\mdseries\upshape} % No bold!
\usepackage{graphicx}
\graphicspath{ {./pings/} }

\usepackage{amsmath}
\DeclareMathOperator*{\argmax}{arg\,max}
\DeclareMathOperator*{\argmin}{arg\,min}

\newcount\colveccount
\newcommand*\colvec[1]{
        \global\colveccount#1
        \begin{pmatrix}
        \colvecnext
}
\def\colvecnext#1{
        #1
        \global\advance\colveccount-1
        \ifnum\colveccount>0
                \\
                \expandafter\colvecnext
        \else
                \end{pmatrix}
        \fi
}

%%% END Article customizations

%%% The "real" document content comes below...

\title{Macro PS2}
\author{Michael B. Nattinger\footnote{I worked on this assignment with my study group: Alex von Hafften, Andrew Smith, and Ryan Mather. I have also discussed problem(s) with Emily Case, Sarah Bass, Katherine Kwok, and Danny Edgel.}}

%\date{} % Activate to display a given date or no date (if empty),
         % otherwise the current date is printed 

\begin{document}
\maketitle
\section{Question 1}
The planner solves the following maximization problem subject to the capital law of motion and the resource constraint:
\begin{align*}
\max_{\{ C_t,I_t, K_t\}_{t=1}^{\infty}} \sum_{t=0}^{\infty} \beta^t log C_t\\
\text{s.t. } K_{t+1} = K_t^{1-\delta}I_t^{\delta}\\
\text{and } AK_t^{\alpha} = C_t + I_t
\end{align*}
We can solve the resource constraint for $I_t$ and plug it into the capital law of motion. Using this simplification, we can write down our Lagrangian:
\begin{align*}
\mathcal{L} &= \sum_{t=0}^{\infty} \beta^t log C_t + \lambda_t\left(-K_{t+1}+ K_t^{1-\delta}\left( AK_t^{\alpha}  - C_t \right)^{\delta}\right)
\end{align*}
Taking first order conditions with respect to $C_t,K_{t+1}$ we find the following:
\begin{align*}
\frac{\beta^t}{C_t} &= \lambda_t \delta K_t^{1-\delta}(AK_t^{\alpha}  - C_t )^{\delta - 1}\\
\lambda_t &= \lambda_{t+1}(K_{t+1}^{1-\delta}\delta(AK_{t+1}^{\alpha}  - C_{t+1} )^{\delta - 1}A\alpha K_{t+1}^{\alpha - 1} + (1-\delta)K_{t+1}^{-\delta} \left( AK_{t+1}^{\alpha}  - C_{t+1} \right)^{\delta})\\
\Rightarrow \lambda_t &= \frac{\beta^t}{\delta C_tK_t^{1-\delta}I_t^{\delta - 1}} \\
\Rightarrow  \frac{1}{C_tK_t^{1-\delta}I_t^{\delta - 1}}  &=  \frac{\beta}{C_{t+1}K_{t+1}^{1-\delta}I_{t+1}^{\delta - 1}}(A\alpha \delta K_{t+1}^{\alpha - \delta}I_{t+1}^{\delta - 1} + (1-\delta)K_{t+1}^{-\delta}I_{t+1}^{\delta})
\end{align*}

\begin{align}
\frac{1}{C_tK_t^{1-\delta}I_t^{\delta - 1}}  &= \frac{\beta}{C_{t+1}}(A\alpha \delta K_{t+1}^{\alpha - 1} + (1-\delta)K_{t+1}^{-1}I_{t+1}) \label{eul}
\end{align}
The above equation forms our Euler equation.
%We can solve the capital law of motion for $I_t$ and plug it into the resource constraint. Using this simplification, we can write down our Lagrangian:
%\begin{align*}
%\mathcal{L} &= \sum_{t=0}^{\infty} \beta^t log C_t + \lambda_t\left(AK_t^{\alpha} - \left(\frac{K_{t+1}}{K_t^{1-\delta}}\right)^{1/\delta} - C_t\right)
%\end{align*}
%Taking first order conditions with respect to $C_t,K_{t+1}$ we find the following:
%\begin{align*}
%\frac{\beta^t}{C_t} &= \lambda_t\\
%\lambda_{t+1}A\alpha K_{t+1}^{\alpha - 1} &= \lambda_{t}\frac{}{}
%\end{align*}

Assume we are on the optimal trajectory at time $t$, and consider a one-period deviation in consumption by an amount $D$. Our resource constraint implies that this results in a decrease in $I_t$ by an equal amount, $D$. Then, our $K_{t+1}$ is reduced (to first order approximation) by $ -\delta D K_t^{1-\delta}I_t^{\delta-1}$. Then, our consumption in the second equation is reduced by two effects: reduced $K_{t+1}$ leads to less production at time $t+1$, and a larger gap to make up via $I_{t+1}$ to get back onto the optimal trajectory at time $t+2$. The net effect of the first of these terms, to first order expansion, is $-(\delta D  K_t^{1-\delta}I_t^{\delta-1})(A\alpha K_{t+1}^{\alpha - 1} ) $, in other words, the reduction in $C_{t+1}$ from the (first order approximation of the) decrease in production in period (t+1). Now we must address the second of these turns. $K_{t+2} = K_{t+1}^{1-\delta}I_{t+1}^{\delta}$ is fixed and we know the value of $K_{t+1}$ so we can determine the value of $I_{t+1}.$ To first order approximation, small deviations of capital and investment $(\Delta K_{t+1}),(\Delta I_{t+1})$ satisfy $(1-\delta)((\Delta K_{t+1}))(K_{t+1}^{-\delta}I_{t+1}^{\delta}) = -\delta (\Delta I_{t+1})(K_{t+1}^{1-\delta}I_{t+1}^{\delta - 1}) \Rightarrow (\Delta I_{t+1}) =- \frac{1-\delta}{\delta}(I_{t+1}K_{t+1}^{-1})(\Delta K_{t+1})$. This is taken away from $C_{t+1}.$ Therefore, our second effect of the reduction in $K_{t+1}$ on $C_{t+1}$ is $-(\delta \Delta  K_t^{1-\delta}I_t^{\delta-1})\frac{1-\delta}{\delta}\frac{I_{t+1}}{K_{t+1}}$.

Our marginal utility by making this move is thus 
\begin{align*}
dU &=  \beta^tC_t^{-1}D - \beta^{t+1} C_{t+1}^{-1}\left((\delta K_t^{1-\delta}I_t^{\delta-1})\left(A\alpha K_{t+1}^{\alpha - 1} + \frac{1-\delta}{\delta}\frac{I_{t+1}}{K_{t+1}} \right) \right)D\\
dU = 0 \Rightarrow C_t^{-1} &= \beta C_{t+1}^{-1}( K_t^{1-\delta}I_t^{\delta-1})\left(A\alpha \delta K_{t+1}^{\alpha - 1} + (1-\delta)\frac{I_{t+1}}{K_{t+1}} \right)
\end{align*}
This yields (\ref{eul}), our euler condition. Therefore, the euler condition represents a no-profitable-deviation condition.
\section{Question 2}
The system of equations that pins down the law of motion for the system are the following:
\begin{align*}
\frac{1}{C_tK_t^{1-\delta}I_t^{\delta - 1}}  &= \frac{\beta}{C_{t+1}}(A\alpha \delta K_{t+1}^{\alpha - 1} + (1-\delta)K_{t+1}^{-1}I_{t+1})\\
AK_t^{\alpha} &= C_t + I_t \\
K_{t+1} &= K_t^{1-\delta}I_t^{\delta}
\end{align*}
We can use the resource constraint to rewrite the system of equations without $I_t$:
\begin{align}
%\frac{1}{C_tK_t^{1-\delta}(AK_t^{\alpha} - C_t)^{\delta - 1}}  &= \frac{\beta}{C_{t+1}}(A\alpha \delta K_{t+1}^{\alpha - 1} + (1-\delta)K_{t+1}^{-1}(AK_t^{\alpha} - C_t)) \\
C_{t+1} &= \beta C_tK_t^{1-\delta}(AK_t^{\alpha} - C_t)^{\delta - 1}(A\alpha \delta K_{t+1}^{\alpha - 1}  + (1-\delta)K_{t+1}^{-1}(AK_{t+1}^{\alpha} - C_{t+1}) ) \label{lom1}\\
K_{t+1} &= K_t^{1-\delta}(AK_t^{\alpha} - C_t)^{\delta} \label{lom2}
\end{align}

Equations (\ref{lom1}) and (\ref{lom2}) determine the law of motion of the system. We can use these equations and impose stationarity $(K_t = K_{t+1} = \bar{K}, C_t = C_{t+1} = \bar{C})$ to determine the steady state:
\begin{align*}
1 &= \beta \bar{K}^{1-\delta}(A\bar{K}^{\alpha} - \bar{C})^{\delta - 1}(A\alpha \delta \bar{K}^{\alpha - 1} + (1 - \delta) \bar{K}^{-1}(A\bar{K}^{\alpha} - \bar{C})) \\
1 &= \bar{K}^{-\delta}(A\bar{K}^{\alpha} - \bar{C})^\delta
\end{align*}
The above equations pin down the steady state of the model.
\section{Question 3}
We will log linearize about the steady state defined in Question 2. We first will define $I = AK^{\alpha} - C.$ Log linearizing I we get:
\begin{align*}
\bar{I}(1+i_t) &= A\bar{K}^{\alpha}(1+\alpha k_t) - \bar{C}(1+c_t)\\
\Rightarrow i_t &= A\frac{\bar{K}^{\alpha}}{\bar{I}}k_t - \frac{\bar{C}}{\bar{I}}c_t \\
\Rightarrow i_t &= A\alpha\bar{K}^{\alpha - 1}k_t - \frac{\bar{C}}{\bar{I}}c_t ,
\end{align*}
where we have used the fact that equation (\ref{lom2}) implies that $\bar{K} = \bar{I}$.

Using this we can log linearize equation (\ref{lom2}):
\begin{align*}
\bar{K}(1+k_{t+1}) &= \bar{K}^{1-\delta}(1+(1-\delta)k_t)\bar{I}^{\delta}(1+\delta i_t)\\
\Rightarrow k_{t+1} &= (1-\delta)k_t + \delta i_t\\
&= (1-\delta)k_t + \delta \left( A\alpha\bar{K}^{\alpha - 1}k_t - \frac{\bar{C}}{\bar{I}}c_t \right)\\
&= (1-\delta + A\alpha\bar{K}^{\alpha - 1}\delta) k_t - \delta\frac{\bar{C}}{\bar{I}}c_t
\end{align*}

Now we can log linearize equation (\ref{lom1}):
\begin{align*}
(1+c_{t+1}) &= \beta (1+c_{t}) (1+(1-\delta)k_t) (1+(\delta - 1)i_t)\\ &*(A\alpha \delta \bar{K}^{\alpha - 1}(1+(\alpha -1)k_{t+1}) + (1-\delta)(1-k_{t+1})(1+i_{t+1}))\\
c_{t+1} &= \beta (A\alpha\delta\bar{K}^{\alpha-1}(\alpha - 1)k_{t+1} + (1-\delta)(i_{t+1} - k_{t+1})\\ &+ (A\alpha\delta\bar{K}^{\alpha-1} + (1-\delta))(c_t + (1-\delta)k_t + (\delta - 1)i_t))\\
c_{t+1} &= \frac{A\alpha \delta \bar{K}^{\alpha-1}(\alpha -1)}{(A\alpha\delta\bar{K}^{\alpha-1} + (1-\delta))}k_{t+1} +  \frac{ (1-\delta)}{(A\alpha\delta\bar{K}^{\alpha-1} + (1-\delta))} i_{t+1} \\
&-   \frac{ (1-\delta)}{(A\alpha\delta\bar{K}^{\alpha-1} + (1-\delta))} k_{t+1} + c_t + (1-\delta)k_t + (\delta - 1)i_t
\end{align*}

\section{Question 4}
Define, for sake of convenience, $ \phi = A\bar{K}^{\alpha - 1}$. Note that the euler equation steady state yields $1-\delta + \phi \alpha \delta = 1/\beta$, and the investment steady state yields $\bar{C}/\bar{K} = \phi - 1$. Then, we can reduce the above equation to the following:
\begin{align*}
c_{t+1} &= \beta( \phi \alpha \delta (\alpha - 1) - 1+\delta)k_{t+1} + \beta(1-\delta) i_{t+1} + c_t + (1-\delta) k_t - (1-\delta) i_t,\\
&= \beta(\phi \alpha \delta  (\alpha - 1) - 1+\delta)k_{t+1} + \beta(1-\delta)\left(\phi \alpha k_{t+1} - (\phi - 1)c_{t+1} \right)  \\&+ c_t + (1-\delta) k_t - (1-\delta) \left( \phi \alpha k_t - (\phi - 1)c_t \right)
\end{align*}
\begin{align*}
\Rightarrow c_{t+1}\left( 1 + \beta(1-\delta) (\phi - 1)\right) &=  \beta(\phi \alpha \delta  (\alpha - 1) + (1-\delta)(\phi \alpha - 1))k_{t+1}   \\&+(\delta  +(1-\delta) \phi) c_t +  (1-\delta)\left( 1-\phi \alpha \right)k_t 
\end{align*}
\begin{align*}
c_{t+1}\left( 1 + \beta(1-\delta) (\phi - 1)\right)  &=  \beta(\phi \alpha \delta  (\alpha - 1) + (1-\delta)(\phi \alpha  - 1))\left( \beta^{-1}k_t - \delta(\phi - 1)c_t \right)\\ &+(\delta  +(1-\delta) \phi) c_t +  (1-\delta)\left( 1-\phi \alpha  \right)k_t \\
&= ((\phi \alpha \delta  (\alpha - 1) + (1-\delta)(\phi \alpha  - 1)) + (1-\delta) - (1-\delta)\phi\alpha)k_t \\&+ ((\delta  +(1-\delta) \phi) - \delta(\phi - 1)\beta (\phi \alpha \delta  (\alpha - 1) + (1-\delta)(\phi \alpha  - 1)))c_t\\
&= \phi \alpha \delta (\alpha - 1)k_t \\&+ ((\delta  +(1-\delta) \phi)- \delta(\phi - 1)\beta (\phi \alpha \delta  (\alpha - 1) + (1-\delta)(\phi \alpha  - 1)))c_t
\end{align*}
\begin{align*}
c_{t+1} &= \frac{\phi \alpha \delta (\alpha - 1)}{1 + \beta(1-\delta) (\phi - 1)}k_{t}\\ &+ \frac{(\delta  +(1-\delta) \phi) - \delta(\phi - 1)\beta (\phi \alpha \delta  (\alpha - 1) + (1-\delta)(\phi \alpha  - 1))}{1 + \beta(1-\delta) (\phi - 1)}c_t
\end{align*}

Define $\theta:= (\delta  +(1-\delta) \phi) - \delta(\phi - 1)\beta (\phi \alpha \delta  (\alpha - 1) + (1-\delta)(\phi \alpha  - 1))$. Then, we can write our log linearized law of motion as the following:

\begin{align}
\colvec{2}{k_{t+1}}{c_{t+1}} &= X_{t+1} = \begin{pmatrix} \beta^{-1} & -\delta(\phi - 1) \\ \frac{\phi \alpha \delta (\alpha - 1)}{1 + \beta(1-\delta) (\phi - 1)} & \frac{\theta}{1 + \beta(1-\delta) (\phi - 1)} \end{pmatrix} \colvec{2}{k_{t}}{c_{t}} = AX_t. \label{loml}
\end{align}

We now must decompose $A = \Gamma \Omega \Gamma^{-1}.$ We will solve for the eigenvectors of A, which form the column vector $(1-\delta)$:

\begin{align*}
det(A-\lambda I_2) &= det\begin{pmatrix} \beta^{-1} - \lambda & -\delta(\phi - 1) \\ \frac{\phi \alpha \delta (\alpha - 1)}{1 + \beta(1-\delta) (\phi - 1)} & \frac{\theta}{1 + \beta(1-\delta) (\phi - 1)}  - \lambda\end{pmatrix} \\
&= (\beta^{-1} - \lambda)\left(  \frac{\theta}{1 + \beta(1-\delta) (\phi - 1)}  - \lambda\right) + \frac{\phi \alpha \delta (\alpha - 1)\delta(\phi - 1)}{1 + \beta(1-\delta) (\phi - 1)} \\
&= \frac{\theta \beta^{-1} + \phi \alpha \delta (\alpha - 1)\delta(\phi - 1)}{1 + \beta(1-\delta) (\phi - 1)} - \lambda\left(\beta^{-1} +\frac{\theta}{1 + \beta(1-\delta) (\phi - 1)}\right) + \lambda^2.
\end{align*}

The eigenvalues are the roots of the above expression. We can solve by applying the quadratic formula:

\begin{align*}
\lambda &= (1/2)\left( \beta^{-1} +\frac{\theta}{1 + \beta(1-\delta) (\phi - 1)} \pm \sqrt{\left( \beta^{-1} +\frac{\theta}{1 + \beta(1-\delta) (\phi - 1)}\right)^2 - 4\frac{\theta \beta^{-1} + \phi \alpha \delta (\alpha - 1)\delta(\phi - 1)}{1 + \beta(1-\delta) (\phi - 1)}} \right)
\end{align*}
Notice that $\beta^{-1} +\frac{\theta}{1 + \beta(1-\delta) (\phi - 1)}>2$. Therefore, clearly the root associated with addition is above 1 in magnitude. This is the eigenvalue corresponding to the explosive eigenvector of our system. We now make the following changes to our system:

\begin{align*}
\Gamma^{-1} X_{t+1} = Y_{t+1} &= \Omega\Gamma^{-1} X_{t} = \Omega Y_{t}
\end{align*}

As $A$ is diagonal with the explosive eigenvalue in the upper left entry, we know that $Y_{1,t} = 0 \forall t$. This defines our saddle path. Equivalently, the saddle path is determined by the second column of the eigenvector matrix $\Gamma$. Therefore, there exists some $z$ such that $c_t = z k_t$, which defines the Blanchard-Kahn first order approximation to the saddle path.

\section{Question 5}
%Outline: From (4) find $\eta$ such that $c_t = \eta k_t \Rightarrow C_t = ZK_t$ for some $Z$. Plug into euler and LOM and solve to show that euler is satisfied.
We will guess that the solution to the euler equation is of the form $C_t = ZK_t$:

\begin{align*}
ZK_{t+1}^z&= \beta  ZK_t^{z} K_t^{1-\delta}(AK_t^{\alpha} -  ZK_t^{z})^{\delta - 1}(A\alpha \delta K_{t+1}^{\alpha - 1}  + (1-\delta)K_{t+1}^{-1}(AK_{t+1}^{\alpha} -  ZK_{t+1}^{z}) )\\
K_{t+1} &= K_t^{1-\delta}(AK_t^{\alpha} -  ZK_t^{z})^{\delta}
\end{align*}

The above system has several possible solutions. The first such solution is the "eat everything" option where $I_t = 0 \Rightarrow AK_{t}^{\alpha} = C_t \Rightarrow K_{t+1} = 0$. This is a possible solution but not the only solution, and in general it is not the solution corresponding to the saddle path.

To find the other solutions for $Z,z$ we simplify the above expressions:
\begin{align*}
Z(K_t^{1-\delta}(AK_t^{\alpha} -  ZK_t^{z})^{\delta})^z&= \beta  ZK_t^{z} K_t^{1-\delta}(AK_t^{\alpha} -  ZK_t^{z})^{\delta - 1}(A\alpha \delta K_{t+1}^{\alpha - 1}  + (1-\delta)K_{t+1}^{-1}(AK_{t+1}^{\alpha} -  ZK_{t+1}^{z}) ) \\
K_t^{z}(AK_t^{\alpha - 1} -  ZK_t^{z - 1})^{z\delta}&= \beta  K_t^{z} K_t^{1-\delta}(AK_t^{\alpha} -  ZK_t^{z})^{\delta - 1}(A\alpha \delta K_{t+1}^{\alpha - 1}  + (1-\delta)K_{t+1}^{-1}(AK_{t+1}^{\alpha} -  ZK_{t+1}^{z}) ) \\
(AK_t^{\alpha - 1} -  ZK_t^{z - 1})^{z\delta}&= \beta  (AK_t^{\alpha-1} -  ZK_t^{z-1})^{\delta - 1}(A\alpha \delta K_{t+1}^{\alpha - 1}  + (1-\delta)(AK_{t+1}^{\alpha-1} -  ZK_{t+1}^{z-1}) )
\end{align*}
Note that if $z=\alpha$ this collapses to the following:
\begin{align*}
((A-Z)K_t^{\alpha - 1})^{\alpha\delta}&= \beta  ((A-Z)K_t^{\alpha-1} )^{\delta - 1}(A\alpha \delta K_{t+1}^{\alpha - 1}  + (1-\delta)((A-Z)K_{t+1}^{\alpha-1} ) )\\
((A-Z)K_t^{\alpha - 1})^{\alpha\delta}&= \beta  ((A-Z)K_t^{\alpha-1} )^{\delta - 1}(A\alpha \delta  + (1-\delta)((A-Z)) )K_{t+1}^{\alpha-1} \\
((A-Z)K_t^{\alpha - 1})^{\alpha\delta}&= \beta  ((A-Z)K_t^{\alpha-1} )^{\delta - 1}(A\alpha \delta  + (1-\delta)((A-Z)) )K_t^{\alpha - 1} ((A-Z)K_t^{\alpha - 1})^{\delta(\alpha - 1)}\\
((A-Z)K_t^{\alpha - 1})^{\delta}&= \beta  ((A-Z)K_t^{\alpha-1} )^{\delta - 1}(A\alpha \delta  + (1-\delta)(A-Z) )K_t^{\alpha - 1} \\
((A-Z)K_t^{\alpha - 1})&= \beta (A\alpha \delta  + (1-\delta)(A-Z) )K_t^{\alpha - 1} \\
(A-Z)&= \beta (A\alpha \delta  + (1-\delta)(A-Z) )\\
A - Z &= \beta A\alpha \delta + (1-\delta)\beta A - (1-\delta)\beta Z\\
Z(1 - \beta + \beta \delta) &= A - \beta A \alpha \delta + (1-\delta)\beta A\\
\Rightarrow Z&= \frac{A(1 - \beta  \alpha \delta + (1-\delta)\beta) }{(1-\beta + \beta\delta)}.
\end{align*}

Therefore, $C_t = ZK_t^{z}$ satisfies the euler conditions for $z = \alpha, Z = \frac{A(1 - \beta  \alpha \delta + (1-\delta)\beta) }{(1-\beta + \beta\delta)}$. It defines, therefore, the saddle path. Note that $C_t = ZK_t^z \Rightarrow \frac{C_t}{\bar{C}} = \frac{ZK_t^z}{Z\bar{K}^z} \Rightarrow c_t = zk_t$. Therefore, the saddle path, written in terms of log-deviation from steady state, is exactly linear. The Blanchard-Kahn approximation is, therefore, linearly approximating a linear function to the first order, and thus it must yield the exact solution to the social planner's problem.

\section{Question 6}
Let us first write the planner's problem. We will jump immediately to the lagrangian formulation:
\begin{align*}
\mathcal{L} &= E_t\sum_{t=0}^{\infty} \beta^t log C_t + \lambda_t\left(-K_{t+1}+ K_t^{1-\delta}\left( A_tK_t^{\alpha}  - C_t \right)^{\delta}\right)
\end{align*}

This yields the following first order conditions:
\begin{align*}
\frac{\beta^t}{C_t} &= \lambda_t \delta K_t^{1-\delta}(A_tK_t^{\alpha}  - C_t )^{\delta - 1}\\
\lambda_t &= E_t\lambda_{t+1}(K_{t+1}^{1-\delta}\delta(A_{t+1}K_{t+1}^{\alpha}  - C_{t+1} )^{\delta - 1}A_{t+1}\alpha K_{t+1}^{\alpha - 1} + (1-\delta)K_{t+1}^{-\delta} \left( A_{t+1}K_{t+1}^{\alpha}  - C_{t+1} \right)^{\delta})\\
\Rightarrow \lambda_t &= \frac{\beta^t}{\delta C_tK_t^{1-\delta}I_t^{\delta - 1}} \\
\Rightarrow  \frac{1}{C_tK_t^{1-\delta}I_t^{\delta - 1}}  &=  E_t\left[\frac{\beta}{C_{t+1}}(A_{t+1}\alpha \delta K_{t+1}^{\alpha - 1} + (1-\delta)K_{t+1}^{-1}I_{t+1})\right]
\end{align*}
The above expression forms our euler condition for the stochastic case. 
%Outline: Guess small change for solution to (6) adjusting for stochastic case and show that this guess satisfies the euler condition above.
We will assume that $E[A_t] = A \forall t$. Inspired by our solution to question (5) we will guess the solution to the euler equation in the stochastic case takes the form $C_{t} = ZK_t^{z}$, and solve for $Z,z$:

\begin{align*}
E_t[ZK_{t+1}^z]&= \beta  E_t[ZK_t^{z} K_t^{1-\delta}(A_tK_t^{\alpha} -  ZK_t^{z})^{\delta - 1}(A_{t+1}\alpha \delta K_{t+1}^{\alpha - 1}  + (1-\delta)K_{t+1}^{-1}(A_{t+1}K_{t+1}^{\alpha} -  ZK_{t+1}^{z}) )]\\
K_{t+1} &= K_t^{1-\delta}(A_tK_t^{\alpha} -  ZK_t^{z})^{\delta}
\end{align*}

As before we still have the 'eat everything' solution which will trivially satisfy the euler condition. We will solve for an additional solution:
\begin{align*}
E_t[Z(K_t^{1-\delta}(A_tK_t^{\alpha} -  ZK_t^{z})^{\delta})^z]&= \beta E_t [ ZK_t^{z} K_t^{1-\delta}(A_tK_t^{\alpha} -  ZK_t^{z})^{\delta - 1}\\&*(A_{t+1}\alpha \delta  K_{t+1}^{\alpha - 1}  + (1-\delta)K_{t+1}^{-1}(A_{t+1}K_{t+1}^{\alpha} -  ZK_{t+1}^{z}) ) ]\\
E_t[K_t^{z}(A_tK_t^{\alpha - 1} -  ZK_t^{z - 1})^{z\delta}]&= \beta E_t [K_t^{z} K_t^{1-\delta}(A_tK_t^{\alpha} -  ZK_t^{z})^{\delta - 1}\\&*(A_{t+1}\alpha \delta K_{t+1}^{\alpha - 1}  + (1-\delta)K_{t+1}^{-1}(A_{t+1}K_{t+1}^{\alpha} -  ZK_{t+1}^{z}) ) ] \\
E_t[(A_tK_t^{\alpha - 1} -  ZK_t^{z - 1})^{z\delta}]&= \beta E_t[ (A_tK_t^{\alpha-1} -  ZK_t^{z-1})^{\delta - 1}\\&*(A_{t+1}\alpha \delta K_{t+1}^{\alpha - 1}  + (1-\delta)(A_{t+1}K_{t+1}^{\alpha-1} -  ZK_{t+1}^{z-1}) )]
\end{align*}
We guess and verify that $z = \alpha$ is a solution.

\begin{align*}
((A_t-Z)K_t^{\alpha - 1})^{\alpha\delta}&= \beta E_t[ ((A_t-Z)K_t^{\alpha-1} )^{\delta - 1}(A_{t+1}\alpha \delta K_{t+1}^{\alpha - 1}  + (1-\delta)((A_{t+1}-Z)K_{t+1}^{\alpha-1} ) )]\\
((A_t-Z)K_t^{\alpha - 1})^{\alpha\delta}&= \beta  E[((A_t-Z)K_t^{\alpha-1} )^{\delta - 1}(A_{t+1}\alpha \delta  + (1-\delta)((A_{t+1}-Z)) )K_{t+1}^{\alpha-1}] \\
((A_t-Z)K_t^{\alpha - 1})^{\alpha\delta}&= \beta E_t[ ((A_t-Z)K_t^{\alpha-1} )^{\delta - 1}(A_{t+1}\alpha \delta  + (1-\delta)((A_{t+1}-Z)) )K_t^{\alpha - 1} ((A_t-Z)K_t^{\alpha - 1})^{\delta(\alpha - 1)}]\\
((A_t-Z)K_t^{\alpha - 1})^{\delta}&= \beta  E_t[((A_t-Z)K_t^{\alpha-1} )^{\delta - 1}(A_{t+1}\alpha \delta  + (1-\delta)(A_{t+1}-Z) )K_t^{\alpha - 1}] \\
((A_t-Z)K_t^{\alpha - 1})&= \beta E_t[(A_{t+1}\alpha \delta  + (1-\delta)(A_{t+1}-Z) )K_t^{\alpha - 1} ]\\
(A_t-Z)&= \beta E_t[ (A_{t+1}\alpha \delta  + (1-\delta)(A_{t+1}-Z) )\\
A_t - Z &= \beta E_t[ A_{t+1}\alpha \delta + (1-\delta)\beta A_{t+1} - (1-\delta)\beta Z]\\
Z(1 - \beta + \beta \delta) &= A_t - \beta E_t[A_{t+1}] \alpha \delta + (1-\delta)\beta E_t[A_{t+1}]\\
\Rightarrow Z&= \frac{A_t - \beta E_t[A_{t+1}] \alpha \delta + (1-\delta)\beta E_t[A_{t+1}]}{(1-\beta + \beta\delta)}.
\end{align*}

\section{Question 7}
Our solutions for Question 5 and Question 6 show that consumption and capital deviations from the steady state are perfectly correlated. This comes from the formulation for capital law of motion, which ensures that investment is perfectly correlated with capital deviations, and therefore consumption will also be perfectly correlated with capital levels via the resource constraint.
\end{document}
