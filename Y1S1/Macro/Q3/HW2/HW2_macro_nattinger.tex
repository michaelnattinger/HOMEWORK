% !TEX TS-program = pdflatex
% !TEX encoding = UTF-8 Unicode

% This is a simple template for a LaTeX document using the "article" class.
% See "book", "report", "letter" for other types of document.

\documentclass[11pt]{article} % use larger type; default would be 10pt

\usepackage[utf8]{inputenc} % set input encoding (not needed with XeLaTeX)

%%% PAGE DIMENSIONS
\usepackage{geometry} % to change the page dimensions
\geometry{a4paper} % or letterpaper (US) or a5paper or....

\usepackage{graphicx} % support the \includegraphics command and options

\usepackage{amssymb}
\usepackage{amsmath}
%%% PACKAGES
\usepackage{booktabs} % for much better looking tables
\usepackage{array} % for better arrays (eg matrices) in maths
\usepackage{paralist} % very flexible & customisable lists (eg. enumerate/itemize, etc.)
\usepackage{verbatim} % adds environment for commenting out blocks of text & for better verbatim
\usepackage{subfig} % make it possible to include more than one captioned figure/table in a single float
% These packages are all incorporated in the memoir class to one degree or another...

%%% HEADERS & FOOTERS
\usepackage{fancyhdr} % This should be set AFTER setting up the page geometry
\pagestyle{fancy} % options: empty , plain , fancy
\renewcommand{\headrulewidth}{0pt} % customise the layout...
\lhead{}\chead{}\rhead{}
\lfoot{}\cfoot{\thepage}\rfoot{}

%%% SECTION TITLE APPEARANCE
\usepackage{sectsty}
\allsectionsfont{\sffamily\mdseries\upshape} % (See the fntguide.pdf for font help)
% (This matches ConTeXt defaults)

%%% ToC (table of contents) APPEARANCE
\usepackage[nottoc,notlof,notlot]{tocbibind} % Put the bibliography in the ToC
\usepackage[titles,subfigure]{tocloft} % Alter the style of the Table of Contents
\renewcommand{\cftsecfont}{\rmfamily\mdseries\upshape}
\renewcommand{\cftsecpagefont}{\rmfamily\mdseries\upshape} % No bold!
\usepackage{graphicx}
\graphicspath{ {./pings/} }

\usepackage{amsmath}
\DeclareMathOperator*{\argmax}{arg\,max}
\DeclareMathOperator*{\argmin}{arg\,min}

\newcount\colveccount
\newcommand*\colvec[1]{
        \global\colveccount#1
        \begin{pmatrix}
        \colvecnext
}
\def\colvecnext#1{
        #1
        \global\advance\colveccount-1
        \ifnum\colveccount>0
                \\
                \expandafter\colvecnext
        \else
                \end{pmatrix}
        \fi
}

%%% END Article customizations

%%% The "real" document content comes below...

\title{Macro PS2}
\author{Michael B. Nattinger\footnote{I worked on this assignment with my study group: Alex von Hafften, Andrew Smith, and Ryan Mather. I have also discussed problem(s) with Emily Case, Sarah Bass, Katherine Kwok, and Danny Edgel.}}

%\date{} % Activate to display a given date or no date (if empty),
         % otherwise the current date is printed 

\begin{document}
\maketitle
\section{Question 1}
The planner solves the following maximization problem subject to the capital law of motion and the resource constraint:
\begin{align*}
\max_{\{ C_t,I_t, K_t\}_{t=1}^{\infty}} \sum_{t=0}^{\infty} \beta^t log C_t\\
\text{s.t. } K_{t+1} = K_t^{1-\delta}I_t^{\delta}\\
\text{and } AK_t^{\alpha} = C_t + I_t
\end{align*}
We can solve the resource constraint for $I_t$ and plug it into the capital law of motion. Using this simplification, we can write down our Lagrangian:
\begin{align*}
\mathcal{L} &= \sum_{t=0}^{\infty} \beta^t log C_t + \lambda_t\left(-K_{t+1}+ K_t^{1-\delta}\left( AK_t^{\alpha}  - C_t \right)^{\delta}\right)
\end{align*}
Taking first order conditions with respect to $C_t,K_{t+1}$ we find the following:
\begin{align*}
\frac{\beta^t}{C_t} &= \lambda_t \delta K_t^{1-\delta}(AK_t^{\alpha}  - C_t )^{\delta - 1}\\
\lambda_t &= \lambda_{t+1}(K_{t+1}^{1-\delta}\delta(AK_{t+1}^{\alpha}  - C_{t+1} )^{\delta - 1}A\alpha K_{t+1}^{\alpha - 1} + (1-\delta)K_{t+1}^{-\delta} \left( AK_{t+1}^{\alpha}  - C_{t+1} \right)^{\delta})\\
\Rightarrow \lambda_t &= \frac{\beta^t}{\delta C_tK_t^{1-\delta}I_t^{\delta - 1}} \\
\Rightarrow  \frac{1}{C_tK_t^{1-\delta}I_t^{\delta - 1}}  &=  \frac{\beta}{C_{t+1}K_{t+1}^{1-\delta}I_{t+1}^{\delta - 1}}(A\alpha \delta K_{t+1}^{\alpha - \delta}I_{t+1}^{\delta - 1} + (1-\delta)K_{t+1}^{-\delta}I_{t+1}^{\delta})\\
\end{align*}

\begin{align}
\frac{1}{C_tK_t^{1-\delta}I_t^{\delta - 1}}  &= \frac{\beta}{C_{t+1}}(A\alpha \delta K_{t+1}^{\alpha - 1} + (1-\delta)K_{t+1}^{-1}I_{t+1}) \label{eul}
\end{align}
The above equation forms our Euler equation.
%We can solve the capital law of motion for $I_t$ and plug it into the resource constraint. Using this simplification, we can write down our Lagrangian:
%\begin{align*}
%\mathcal{L} &= \sum_{t=0}^{\infty} \beta^t log C_t + \lambda_t\left(AK_t^{\alpha} - \left(\frac{K_{t+1}}{K_t^{1-\delta}}\right)^{1/\delta} - C_t\right)
%\end{align*}
%Taking first order conditions with respect to $C_t,K_{t+1}$ we find the following:
%\begin{align*}
%\frac{\beta^t}{C_t} &= \lambda_t\\
%\lambda_{t+1}A\alpha K_{t+1}^{\alpha - 1} &= \lambda_{t}\frac{}{}
%\end{align*}

Assume we are on the optimal trajectory at time $t$, and consider a one-period deviation in consumption by an amount $D$. Our resource constraint implies that this results in a decrease in $I_t$ by an equal amount, $D$. Then, our $K_{t+1}$ is reduced (to first order approximation) by $ -\delta D K_t^{1-\delta}I_t^{\delta-1}$. Then, our consumption in the second equation is reduced by two effects: reduced $K_{t+1}$ leads to less production at time $t+1$, and a larger gap to make up via $I_{t+1}$ to get back onto the optimal trajectory at time $t+2$. The net effect of the first of these terms, to first order expansion, is $-(\delta D  K_t^{1-\delta}I_t^{\delta-1})(A\alpha K_{t+1}^{\alpha - 1} ) $, in other words, the reduction in $C_{t+1}$ from the (first order approximation of the) decrease in production in period (t+1). Now we must address the second of these turns. $K_{t+2} = K_{t+1}^{1-\delta}I_{t+1}^{\delta}$ is fixed and we know the value of $K_{t+1}$ so we can determine the value of $I_{t+1}.$ To first order approximation, small deviations of capital and investment $(\Delta K_{t+1}),(\Delta I_{t+1})$ satisfy $(1-\delta)((\Delta K_{t+1}))(K_{t+1}^{-\delta}I_{t+1}^{\delta}) = -\delta (\Delta I_{t+1})(K_{t+1}^{1-\delta}I_{t+1}^{\delta - 1}) \Rightarrow (\Delta I_{t+1}) =- \frac{1-\delta}{\delta}(I_{t+1}K_{t+1}^{-1})(\Delta K_{t+1})$. This is taken away from $C_{t+1}.$ Therefore, our second effect of the reduction in $K_{t+1}$ on $C_{t+1}$ is $-(\delta \Delta  K_t^{1-\delta}I_t^{\delta-1})\frac{1-\delta}{\delta}\frac{I_{t+1}}{K_{t+1}}$.

Our marginal utility by making this move is thus 
\begin{align*}
dU &=  \beta^tC_t^{-1}D - \beta^{t+1} C_{t+1}^{-1}\left((\delta K_t^{1-\delta}I_t^{\delta-1})\left(A\alpha K_{t+1}^{\alpha - 1} + \frac{1-\delta}{\delta}\frac{I_{t+1}}{K_{t+1}} \right) \right)D\\
dU = 0 \Rightarrow C_t^{-1} &= \beta C_{t+1}^{-1}( K_t^{1-\delta}I_t^{\delta-1})\left(A\alpha \delta K_{t+1}^{\alpha - 1} + (1-\delta)\frac{I_{t+1}}{K_{t+1}} \right)
\end{align*}
This yields (\ref{eul}), our euler condition. Therefore, the euler condition represents a no-profitable-deviation condition.
\section{Question 2}
The system of equations that pins down the law of motion for the system are the following:
\begin{align*}
\frac{1}{C_tK_t^{1-\delta}I_t^{\delta - 1}}  &= \frac{\beta}{C_{t+1}}(A\alpha \delta K_{t+1}^{\alpha - 1} + (1-\delta)K_{t+1}^{-1}I_{t+1})\\
AK_t^{\alpha} &= C_t + I_t \\
K_{t+1} &= K_t^{1-\delta}I_t^{\delta}
\end{align*}
We can use the resource constraint to rewrite the system of equations without $I_t$:
\begin{align}
%\frac{1}{C_tK_t^{1-\delta}(AK_t^{\alpha} - C_t)^{\delta - 1}}  &= \frac{\beta}{C_{t+1}}(A\alpha \delta K_{t+1}^{\alpha - 1} + (1-\delta)K_{t+1}^{-1}(AK_t^{\alpha} - C_t)) \\
C_{t+1} &= \beta C_tK_t^{1-\delta}(AK_t^{\alpha} - C_t)^{\delta - 1}(A\alpha \delta K_{t+1}^{\alpha - 1}  + (1-\delta)K_{t+1}^{-1}(AK_{t+1}^{\alpha} - C_{t+1}) ) \label{lom1}\\
K_{t+1} &= K_t^{1-\delta}(AK_t^{\alpha} - C_t)^{\delta} \label{lom2}
\end{align}

Equations (\ref{lom1}) and (\ref{lom2}) determine the law of motion of the system. We can use these equations and impose stationarity $(K_t = K_{t+1} = \bar{K}, C_t = C_{t+1} = \bar{C})$ to determine the steady state:
\begin{align*}
1 &= \beta \bar{K}^{1-\delta}(A\bar{K}^{\alpha} - \bar{C})^{\delta - 1}(A\alpha \delta \bar{K}^{\alpha - 1} + (1 - \delta) \bar{K}^{-1}(A\bar{K}^{\alpha} - \bar{C})) \\
1 &= \bar{K}^{-\delta}(A\bar{K}^{\alpha} - \bar{C})^\delta
\end{align*}
The above equations pin down the steady state of the model.
\section{Question 3}

\end{document}
