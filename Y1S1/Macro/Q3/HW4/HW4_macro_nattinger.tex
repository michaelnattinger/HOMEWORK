% !TEX TS-program = pdflatex
% !TEX encoding = UTF-8 Unicode

% This is a simple template for a LaTeX document using the "article" class.
% See "book", "report", "letter" for other types of document.

\documentclass[11pt]{article} % use larger type; default would be 10pt

\usepackage[utf8]{inputenc} % set input encoding (not needed with XeLaTeX)

%%% PAGE DIMENSIONS
\usepackage{geometry} % to change the page dimensions
\geometry{a4paper} % or letterpaper (US) or a5paper or....
\geometry{margin=1in} 
\usepackage{graphicx} % support the \includegraphics command and options

\usepackage{amssymb}
\usepackage{amsmath}
%%% PACKAGES
\usepackage{booktabs} % for much better looking tables
\usepackage{array} % for better arrays (eg matrices) in maths
\usepackage{paralist} % very flexible & customisable lists (eg. enumerate/itemize, etc.)
\usepackage{verbatim} % adds environment for commenting out blocks of text & for better verbatim
\usepackage{subfig} % make it possible to include more than one captioned figure/table in a single float
% These packages are all incorporated in the memoir class to one degree or another...

%%% HEADERS & FOOTERS
\usepackage{fancyhdr} % This should be set AFTER setting up the page geometry
\pagestyle{fancy} % options: empty , plain , fancy
\renewcommand{\headrulewidth}{0pt} % customise the layout...
\lhead{}\chead{}\rhead{}
\lfoot{}\cfoot{\thepage}\rfoot{}

%%% SECTION TITLE APPEARANCE
\usepackage{sectsty}
\allsectionsfont{\sffamily\mdseries\upshape} % (See the fntguide.pdf for font help)
% (This matches ConTeXt defaults)

%%% ToC (table of contents) APPEARANCE
\usepackage[nottoc,notlof,notlot]{tocbibind} % Put the bibliography in the ToC
\usepackage[titles,subfigure]{tocloft} % Alter the style of the Table of Contents
\renewcommand{\cftsecfont}{\rmfamily\mdseries\upshape}
\renewcommand{\cftsecpagefont}{\rmfamily\mdseries\upshape} % No bold!
\usepackage{graphicx}
\graphicspath{ {./pings/} }

\usepackage{amsmath}
\DeclareMathOperator*{\argmax}{arg\,max}
\DeclareMathOperator*{\argmin}{arg\,min}

\newcount\colveccount
\newcommand*\colvec[1]{
        \global\colveccount#1
        \begin{pmatrix}
        \colvecnext
}
\def\colvecnext#1{
        #1
        \global\advance\colveccount-1
        \ifnum\colveccount>0
                \\
                \expandafter\colvecnext
        \else
                \end{pmatrix}
        \fi
}

%%% END Article customizations

%%% The "real" document content comes below...

\title{Macro PS4}
\author{Michael B. Nattinger\footnote{I worked on this assignment with my study group: Alex von Hafften, Andrew Smith, and Ryan Mather. I have also discussed problem(s) with Emily Case, Sarah Bass, Katherine Kwok, and Danny Edgel.}}

%\date{} % Activate to display a given date or no date (if empty),
         % otherwise the current date is printed 

\begin{document}
\maketitle
\section{Question 1}
The household maximizes their utility subject to their budget constraint. Equivalently, households minimize their costs subject to their utility constraint:
\begin{align*}
&\min_{C_{ik}} \int \sum_i P_{ik} C_{ik} dk\\
&\text{s.t. } \left( \int C_k^{\frac{\rho - 1}{\rho}} dk \right)^{\frac{\rho}{\rho - 1}} = C,\\
&\text{where } \left( \sum_{i=1}^{N_k} C_{ik}^{\frac{\theta - 1}{\theta}} \right)^{\frac{\theta}{\theta - 1}} = C_k.
\end{align*}

We will then write the Lagrangian as follows:

\begin{align*}
\mathcal{L} &=\int \sum_i P_{ik} C_{ik} dk - P\left(  \left( \int C_k^{\frac{\rho - 1}{\rho}} dk\right)^{\frac{\rho}{\rho - 1}} - C \right) + \int P_k\left[C_k - \left( \sum_i C_{ik}^{\frac{\theta - 1}{\theta}}\right)^{\frac{\theta}{\theta - 1}} \right] dk
\end{align*}

We solve this maximization problem by taking first order conditions with respect to our choice variables, in this case $C_{ik},C_k$:
\begin{align*}
P_{k}  &= \frac{\rho}{\rho - 1} \left( \int C_k^{\frac{\rho - 1}{\rho}} dk \right)^{\frac{1}{\rho - 1}}\frac{\rho - 1}{\rho}C_k^{\frac{-1}{\rho}}\\
\Rightarrow C_k &= \left(\frac{P_k}{P}\right)^{-\rho} C. \\
P_{ik} &= P_k  \frac{\theta}{\theta - 1}\left( \sum_{i=1}^{N_k} C_{ik}^{\frac{\theta - 1}{\theta}} \right)^{\frac{1}{\theta - 1}} \frac{\theta - 1}{\theta}C_{ik}^{-\frac{1}{\theta}}\\
\Rightarrow C_{ik} &= \left(\frac{P_{ik}}{P_{k}}\right)^{-\theta}C_{k}
\end{align*}
%Note that:
%\begin{align*}
%\frac{\partial C_k}{\partial C_{ik}} &= \frac{\theta}{\theta - 1}\left( \sum_{i=1}^{N_k} C_{ik}^{\frac{\theta - 1}{\theta}} \right)^{\frac{1}{\theta - 1}} \frac{\theta - 1}{\theta}C_{ik}^{-\frac{1}{\theta}}\\
%&= \left(\frac{C_{ik}}{C_k}\right)^{\frac{1}{\theta}}
%\end{align*}
%We then can simplify our consumption first order condition to the following:
%\begin{align*}
%P_k  \left(\frac{C_{ik}}{C_k}\right)^{\frac{1}{\theta}}  &= P C^{\frac{1}{\rho}}C_k^{\frac{-1}{\rho}}\left(\frac{C_{ik}}{C_k}\right)^{\frac{1}{\theta}}\\
%\Rightarrow C_k &= \left( \frac{P_k}{P} \right)^{-\rho}C,
%\end{align*}
%
%a familiar expression.

We can substitute in our expressions into the definitions of $C,C_k$:
\begin{align*}
\left( \int \left[ \left( \frac{P_k}{P} \right)^{-\rho}C  \right]^{\frac{\rho - 1}{\rho}} dk \right)^{\frac{\rho}{\rho - 1}} &= C\\
\Rightarrow \left( \int  \left( \frac{P_k}{P} \right)^{1-\rho} dk \right)^{\frac{\rho}{\rho - 1}}  &= 1\\
\Rightarrow \left( \int  P_k^{1-\rho} dk \right)^{\frac{1}{ 1-\rho}}  &= P, \\
\left( \sum_i \left[ \left( \frac{P_{ik}}{P_k} \right)^{-\theta} C_k \right]^{\frac{\theta - 1}{\theta}}\right)^{\frac{\theta}{\theta - 1}} &= C_k \\
\Rightarrow \left( \sum_i P_{ik}^{1-\theta} \right)^{\frac{1}{1-\theta}} &= P_k
\end{align*}

To summarize, we have the following:
\begin{align}
P_k &=  \left( \sum_i P_{ik}^{1-\theta} \right)^{\frac{1}{1-\theta}}\\
P &=  \left( \int  \left[  \left( \sum_i P_{ik}^{1-\theta} \right)^{\frac{1}{1-\theta}} \right]^{1-\rho} dk \right)^{\frac{1}{1-\rho}} \\
C_{ik} &= P_{ik}^{-\theta}\left( \sum_i P_{ik}^{1-\theta }\right)^{\frac{ \theta-\rho}{1-\theta }} P^{\rho} C
\end{align}

\section{Question 2}
The firms compete a la Bertrand:

\begin{align*}
&\max_{P_{ik}} P_{ik}C_{ik} - WL_{ik}\\
&\text{s.t.} C_{ik} = P_{ik}^{-\theta}\left( \sum_i P_{ik}^{1-\theta }\right)^{\frac{ \theta-\rho}{1-\theta }} P^{\rho} C\\
&\text{and } C_{ik} = A_{ik}L_{ik}
\end{align*}
Substituting, we form the following objective function:

\begin{align*}
&\max_{P_{ik}}P_{ik}^{1-\theta}\left( \sum_i P_{ik}^{1-\theta }\right)^{\frac{ \theta-\rho}{1-\theta}} P^{\rho} C - WA_{ik}^{-1}P_{ik}^{-\theta}\left( \sum_i P_{ik}^{1-\theta }\right)^{\frac{ \theta-\rho}{1-\theta}} P^{\rho} C \\
\Rightarrow &\max_{P_{ik}}P_{ik}^{1-\theta}\left( \sum_i P_{ik}^{1-\theta}\right)^{\frac{ \theta-\rho}{1-\theta}}  - WA_{ik}^{-1}P_{ik}^{-\theta}\left( \sum_i P_{ik}^{1-\theta}\right)^{\frac{ \theta-\rho}{1-\theta }} 
\end{align*}

We take first order conditions:

\begin{align*}
(1-\theta)P_{ik}^{-\theta}P_k^{\theta - \rho} + P_{ik}^{1-\theta} \frac{\theta - \rho}{1-\theta}P_k^{2\theta - \rho - 1}(1-\theta)P_{ik}^{-\theta} &= \frac{W}{A_{ik}}\left[ (-\theta)P_{ik}^{-\theta-1}P_k^{\theta-\rho} + P_{ik}^{-\theta} \frac{\theta - \rho}{1-\theta} P_k^{2\theta - \rho - 1}(1-\theta)P_{ik}^{-\theta}\right]\\
(1-\theta) + P_{ik}^{1-\theta} (\theta - \rho)P_k^{\theta  - 1} &= \frac{W}{A_{ik}}\left[ (-\theta)P_{ik}^{-1} + P_{ik}^{-\theta} (\theta - \rho) P_k^{\theta - 1}\right]
\end{align*}
Denote the weighted price ratio $s_{ik}:= \left(\frac{P_{ik}}{P_{k}}\right)^{1-\theta}$:

\begin{align*}
P_{ik}[(1-\theta) +  s_{ik} (\theta - \rho)]&= \frac{W}{A_{ik}}\left[ (-\theta) + s_{ik} (\theta - \rho) \right]\\
\Rightarrow P_{ik} &= \frac{W}{A_{ik}}\left[ 1 - \frac{1}{(1-\theta) +  s_{ik} (\theta - \rho)} \right]
\end{align*}

This yields a recursive expression for $P_{ik}$. In other words, we have a set of $i \times k$ nonlinear equations in $i \times k$ unknowns.
%\begin{align*}
%(1-\theta)P_{ik}^{-\theta} P_k^{\theta - \rho} + \frac{ \theta-\rho}{1-\theta } P_{ik}^{1-\theta} P_k^{1 - \rho} &= wA_{ik}^{-1}\left[ (-\theta) P_{ik}^{-\theta - 1} P_k^{\theta - \rho} +\frac{ \theta-\rho}{\theta - 1}  P_{ik}^{-\theta} P_k^{1-\rho} \right]\\
%(1-\theta) P_k^{\theta } + \frac{ \theta-\rho}{\theta - 1} P_{ik} P_k &= wA_{ik}^{-1}\left[ (-\theta) P_{ik}^{ - 1} P_k^{\theta} +\frac{ \theta-\rho}{\theta - 1}   P_k \right] \\
%(1-\theta)P_{ik}  + \frac{ \theta-\rho}{\theta - 1} P_{ik}^2 P_k^{1-\theta} &= wA_{ik}^{-1}\left[ (-\theta)  +\frac{ \theta-\rho}{\theta - 1}   P_k^{1-\theta}P_{ik} \right] \\
%0 &= \frac{ \theta-\rho}{\theta - 1} P_{ik}^2 P_k^{1-\theta} + \left(1-\theta  - wA_{ik}^{-1}\frac{ \theta-\rho}{\theta - 1}   P_k^{1-\theta} \right)P_{ik} + \theta wA_{ik}^{-1}\\
%0 &= P_{ik}^2  + \left(-\frac{(\theta-1)^2}{\theta - \rho}P_k^{\theta - 1}  - wA_{ik}^{-1} \right)P_{ik} + \frac{\theta(\theta - 1) P_k^{\theta - 1} wA_{ik}^{-1}}{\theta - \rho}
%\end{align*}
%\begin{align*}
%\Rightarrow P_{ik} &=  \frac{w}{2A_{ik}}  +\frac{(\theta-1)^2}{2\left(\theta - \rho\right)}P_k^{\theta - 1} \pm \sqrt{\left(\frac{(\theta-1)^2}{2(\theta - \rho)}P_k^{\theta - 1}  + \frac{w}{2A_{ik}} \right)^2 - \frac{\theta(\theta - 1) P_k^{\theta - 1} wA_{ik}^{-1}}{\theta - \rho}}
%\end{align*}

We can derive demand elasticities $\frac{P_{ik}\partial C_{ik}}{C_{ik}\partial P_{ik}}$:

\begin{align*}
\frac{P_{ik}\partial C_{ik}}{C_{ik}\partial P_{ik}} &= \frac{P_{ik}}{C_{ik}}\left( (-\theta)P_{ik}^{-1-\theta}P_{k}^{\theta - \rho} + P_{ik}^{-\theta} \frac{ \theta-\rho}{1-\theta } P_{k}^{2\theta - \rho - 1}(1-\theta)P_{ik}^{-\theta}  \right) P^{\rho} C\\
&=\left(P_{ik}^{1+\theta} P_{k}^{\rho - \theta} P^{-\rho} C^{-1} \right)\left( (-\theta)P_{ik}^{-1-\theta}P_{k}^{\theta - \rho} + P_{ik}^{-2\theta} (\theta-\rho) P_{k}^{2\theta - \rho - 1}  \right) P^{\rho} C\\
&=\left( (-\theta) + P_{ik}^{1-\theta} (\theta-\rho) P_{k}^{\theta  - 1}  \right) \\
&= (\theta - \rho)s_{ik} - \theta.
\end{align*}

\section{Question 3}

The markup of firm $i$ in industry $k$, $\mu_{ik}$, with marginal cost $M_{ik}$  is the following:

\begin{align*}
\mu_{ik} &= P_{ik} / M_{ik}\\
&=  \frac{W}{A_{ik}}\left[ 1 - \frac{1}{(1-\theta) +  s_{ik} (\theta - \rho)} \right] / \frac{W}{A_{ik}}\\
&=  \left[1 - \frac{1}{(1-\theta) +  s_{ik} (\theta - \rho)}\right]
\end{align*}

Taking the derivative with respect to $A_{ik}$:
\begin{align*}
\frac{\partial \mu_{ik}}{\partial A_{ik}} &= -\frac{\partial }{\partial A_{ik}}\left( \frac{1}{(1-\theta)+(\theta - \rho)s_{ik}} \right)\\
&= \left( \frac{1}{(1-\theta)+(\theta - \rho)s_{ik}} \right)^{2}(\theta - \rho)\frac{\partial s_{ik}}{\partial A_{ik}}.
\end{align*}

Note that:
\begin{align*}
\frac{\partial s_{ik}}{\partial A_{ik}} &= (1-\theta)P_{k}^{\theta - 1}P_{ik}^{-\theta}\frac{\partial P_{ik}}{\partial A_{ik}}\\
\Rightarrow \frac{\partial \mu_{ik}}{\partial A_{ik}} &= \left( \frac{1}{(1-\theta)+(\theta - \rho)s_{ik}} \right)^{2}(\theta - \rho) (1-\theta)P_{k}^{1-\theta}P_{ik}^{-\theta}\frac{\partial P_{ik}}{\partial A_{ik}}\\
&>0,
\end{align*}
where we have concluded that this term is positive by noting that the squared fraction, $(\theta - \rho)$, and price terms are positive and the $(1-\theta)$ and $\frac{\partial P_{ik}}{\partial A_{ik}}$ terms are negative.
\section{Question 4}
It is relatively straightforward to code the problem numerically as a fixed point problem. Given a tolerance, draws of $A_{ik}$, tuning parameter $\gamma \in [0,1)$ and starting guess $s_{ik}^{0,1}$, one can proceed using the following algorithm:
\begin{enumerate}
\item For all $ik$, calculate $P_{ik}^{n} = (W/A_{ik})\left[1 - \frac{1}{(1-\theta) +  s_{ik}^{n-1,1} (\theta - \rho)} \right]$. \label{step}
\item For all $k$, calculate $P_{k}^{n} = \left( \sum_i (P_{ik}^{n})^{1-\theta} \right)^{\frac{1}{1-\theta}} $.
\item For all $ik$, calculate $s_{ik}^{n,0} = \left(\frac{P_{ik}^{n}}{P_{k}^n}\right)^{1-\theta}$.
\item Check for convergence: if $\sum_{k}\sum_{i}|s_{ik}^{n,0} - s_{ik}^{n-1,1}|$ is less than the tolerance, stop. Otherwise, set $s_{ik}^{n,1} = (1-\gamma) s_{ik}^{n,0} +  \gamma s_{ik}^{n-1,1}$ and return to step (\ref{step}).
\end{enumerate}

I note a couple of details here. First, note that the higher the tuning parameter $\gamma$, the slower $s_{ik}$ will move in-between iterations. This is particularly important when $\rho$ is near $1$. For $\gamma = 2$, the algorithm converges rapidly with a tuning parameter near 0, but when $\gamma = 1.1$ then the algorithm will converge many orders of magnitude slower with a tuning parameter near 0 than with an appropriately chosen tuning parameter, such as $\gamma = 0.5$. For this exercise, we were told to use $\rho = 1.$ This causes numerical issues as we end up dividing by $0$ in our algorithm above. To get around this issue, we instead use $\rho = 1+ \epsilon$, with $\epsilon<<1$. Specifically, I use $\epsilon = 10^{-9}$, which converges in under fifteen seconds using a tuning parameter of $\gamma = 0.6.$ The numerical results change very little as $\epsilon$ decreases further towards $0$, with no change to any reported digits in moving from $\epsilon = 10^{-6}$ to $\epsilon = 10^{-9}$.

\section{Question 5}
With our solution, computed in Question 4, for all $P_{ik},$ we can calculate $P$ and use the fact that $W = PC$ to solve for $C = WP^{-1}$. Given my set of productivity draws, I calculated that $C = 4.6062 $ (note: here, C is equal to the real wage).

In the first-best outcome, firms charge their marginal cost: $P_{ik} = W/A_{ik}$. Given my set of draws, I calculated that $C_{SPP} = 7.2417$. Unlike the original Dixit-Stiglitz model where the result of the competitive equilibrium is equivalent to the first-best outcome, here there is a significant wedge owing to the market power firms have within their sectors.
\section{Question 6}
In the limit as $\theta \rightarrow \infty,$ goods within a sector become infinitely substitutable. Therefore, in the limit, consumers only buy goods from the cheapest (highest productivity) firm in the sector. The intra-sectoral dynamic then drops out, and the only competition is inter-sectoral. In other words, when $\theta \rightarrow \infty$, the problem  collapses to the one under Bertrand competition with homogeneous goods.

Note that there is a subtle difference between these models, in that the models are equivalent assuming the draws of the highest-productivity firms within each sector of this sectoral model are equal to the productivity draws of the firms in the original Dixit-Stiglitz case. In other words, if each firm in the sectoral model have productivities drawn from a distribution $\mathcal{F}$, then the distribution of productivities of the highest-productivity firms within each sector, $\mathcal{G}$, is defined as follows:
\begin{align*}
\mathcal{G} \sim \max \{ A_{1k},\dots,A_{N_k k}: A_{1k},\dots, A_{N_k k} \sim \mathcal{F} \}
\end{align*}
Then, the sectoral model with firm productivities drawn from the $\mathcal{F}$ in the limit as $\theta \rightarrow \infty$ converges to the original Dixit-Stiglitz model with firm productivities drawn from the $\mathcal{G}$ distribution.
\end{document}
