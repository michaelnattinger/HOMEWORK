% !TEX TS-program = pdflatex
% !TEX encoding = UTF-8 Unicode

% This is a simple template for a LaTeX document using the "article" class.
% See "book", "report", "letter" for other types of document.

\documentclass[11pt]{article} % use larger type; default would be 10pt

\usepackage[utf8]{inputenc} % set input encoding (not needed with XeLaTeX)

%%% PAGE DIMENSIONS
\usepackage{geometry} % to change the page dimensions
\geometry{a4paper} % or letterpaper (US) or a5paper or....
\geometry{margin=1in} 
\usepackage{graphicx} % support the \includegraphics command and options

\usepackage{amssymb}
\usepackage{amsmath}
%%% PACKAGES
\usepackage{booktabs} % for much better looking tables
\usepackage{array} % for better arrays (eg matrices) in maths
\usepackage{paralist} % very flexible & customisable lists (eg. enumerate/itemize, etc.)
\usepackage{verbatim} % adds environment for commenting out blocks of text & for better verbatim
\usepackage{subfig} % make it possible to include more than one captioned figure/table in a single float
% These packages are all incorporated in the memoir class to one degree or another...

%%% HEADERS & FOOTERS
\usepackage{fancyhdr} % This should be set AFTER setting up the page geometry
\pagestyle{fancy} % options: empty , plain , fancy
\renewcommand{\headrulewidth}{0pt} % customise the layout...
\lhead{}\chead{}\rhead{}
\lfoot{}\cfoot{\thepage}\rfoot{}

%%% SECTION TITLE APPEARANCE
\usepackage{sectsty}
\allsectionsfont{\sffamily\mdseries\upshape} % (See the fntguide.pdf for font help)
% (This matches ConTeXt defaults)

%%% ToC (table of contents) APPEARANCE
\usepackage[nottoc,notlof,notlot]{tocbibind} % Put the bibliography in the ToC
\usepackage[titles,subfigure]{tocloft} % Alter the style of the Table of Contents
\renewcommand{\cftsecfont}{\rmfamily\mdseries\upshape}
\renewcommand{\cftsecpagefont}{\rmfamily\mdseries\upshape} % No bold!
\usepackage{graphicx}
\graphicspath{ {./pings/} }

\usepackage{amsmath}
\DeclareMathOperator*{\argmax}{arg\,max}
\DeclareMathOperator*{\argmin}{arg\,min}

\newcount\colveccount
\newcommand*\colvec[1]{
        \global\colveccount#1
        \begin{pmatrix}
        \colvecnext
}
\def\colvecnext#1{
        #1
        \global\advance\colveccount-1
        \ifnum\colveccount>0
                \\
                \expandafter\colvecnext
        \else
                \end{pmatrix}
        \fi
}

%%% END Article customizations

%%% The "real" document content comes below...

\title{Macro PS4}
\author{Michael B. Nattinger\footnote{I worked on this assignment with my study group: Alex von Hafften, Andrew Smith, and Ryan Mather. I have also discussed problem(s) with Emily Case, Sarah Bass, Katherine Kwok, and Danny Edgel.}}

%\date{} % Activate to display a given date or no date (if empty),
         % otherwise the current date is printed 

\begin{document}
\maketitle
\section{Question 1}
The household maximizes their utility subject to their budget constraint. Equivalently, households minimize their costs subject to their utility constraint:
\begin{align*}
&\min_{C_{ik}} \int \sum_i P_{ik} C_{ik} dk\\
&\text{s.t. } \left( \int C_k^{\frac{\rho - 1}{\rho}} dk \right)^{\frac{\rho}{\rho - 1}} = C,\\
&\text{where } \left( \sum_{i=1}^{N_k} C_{ik}^{\frac{\theta - 1}{\theta}} \right)^{\frac{\theta}{\theta - 1}} = C_k.
\end{align*}

We will then write the Lagrangian as follows:

\begin{align*}
\mathcal{L} &=\int \sum_i P_{ik} C_{ik} dk - P\left(  \left( \int C_k^{\frac{\rho - 1}{\rho}} dk\right)^{\frac{\rho}{\rho - 1}} - C \right) + \int P_k\left[C_k - \left( \sum_i C_{ik}^{\frac{\theta - 1}{\theta}}\right)^{\frac{\theta}{\theta - 1}} \right] dk
\end{align*}

We solve this maximization problem by taking first order conditions with respect to our choice variables, in this case $C_{ik},C_k$:
\begin{align*}
P_{k}  &= \frac{\rho}{\rho - 1} \left( \int C_k^{\frac{\rho - 1}{\rho}} dk \right)^{\frac{1}{\rho - 1}}\frac{\rho - 1}{\rho}C_k^{\frac{-1}{\rho}}\\
\Rightarrow C_k &= \left(\frac{P_k}{P}\right)^{\rho} C. \\
P_{ik} &= P_k  \frac{\theta}{\theta - 1}\left( \sum_{i=1}^{N_k} C_{ik}^{\frac{\theta - 1}{\theta}} \right)^{\frac{1}{\theta - 1}} \frac{\theta - 1}{\theta}C_{ik}^{-\frac{1}{\theta}}\\
\Rightarrow C_{ik} &= \left(\frac{P_{ik}}{P_{k}}\right)^{\theta}C_{k}
\end{align*}
%Note that:
%\begin{align*}
%\frac{\partial C_k}{\partial C_{ik}} &= \frac{\theta}{\theta - 1}\left( \sum_{i=1}^{N_k} C_{ik}^{\frac{\theta - 1}{\theta}} \right)^{\frac{1}{\theta - 1}} \frac{\theta - 1}{\theta}C_{ik}^{-\frac{1}{\theta}}\\
%&= \left(\frac{C_{ik}}{C_k}\right)^{\frac{1}{\theta}}
%\end{align*}
We then can simplify our consumption first order condition to the following:
\begin{align*}
P_k  \left(\frac{C_{ik}}{C_k}\right)^{\frac{1}{\theta}}  &= P C^{\frac{1}{\rho}}C_k^{\frac{-1}{\rho}}\left(\frac{C_{ik}}{C_k}\right)^{\frac{1}{\theta}}\\
\Rightarrow C_k &= \left( \frac{P_k}{P} \right)^{-\rho}C,
\end{align*}

a familiar expression to what we found in lecture.

We can substitute in our expressions into the definitions of $C,C_k$:
\begin{align*}
\left( \int \left[ \left( \frac{P_k}{P} \right)^{-\rho}C  \right]^{\frac{\rho - 1}{\rho}} dk \right)^{\frac{\rho}{\rho - 1}} &= C\\
\Rightarrow \left( \int  \left( \frac{P_k}{P} \right)^{1-\rho} dk \right)^{\frac{\rho}{\rho - 1}}  &= 1\\
\Rightarrow \left( \int  P_k^{1-\rho} dk \right)^{\frac{1}{\rho - 1}}  &= P, \\
\left( \sum_i \left[ \left( \frac{P_{ik}}{P_k} \right)^{\theta} C_k \right]^{\frac{\theta - 1}{\theta}}\right)^{\frac{\theta}{\theta - 1}} &= C_k \\
\Rightarrow \left( \sum_i P_{ik}^{\theta - 1} \right)^{\frac{1}{\theta - 1}} &= P_k
\end{align*}

To summarize, we have the following:
\begin{align}
P_k &=  \left( \sum_i P_{ik}^{\theta - 1} \right)^{\frac{1}{\theta - 1}}\\
P &=  \left( \int  \left[  \left( \sum_i P_{ik}^{\theta - 1} \right)^{\frac{1}{\theta - 1}} \right]^{1-\rho} dk \right)^{\frac{1}{\rho - 1}} \\
C_{ik} &= P_{ik}^{-\theta}\left( \sum_i P_{ik}^{\theta - 1}\right)^{\frac{ \theta-\rho}{\theta - 1}} P^{\rho} C
\end{align}

\section{Question 2}
The firms compete a la Cournot:

\begin{align*}
&\max_{P_{ik}} P_{ik}C_{ik} - WL_{ik}\\
&\text{s.t.} C_{ik} = \left( \frac{P_{ik}}{P_k} \right)^{\theta}C_k\\
&\text{and } C_{ik} = A_{ik}L_{ik}
\end{align*}
Substituting, we form the following objective function:

\begin{align*}
\max_{P_{ik}} P_{ik}C_{ik} - WL_{ik}
\end{align*}
\section{Question 3}
\section{Question 4}
\section{Question 5}
\section{Question 6}
\end{document}
