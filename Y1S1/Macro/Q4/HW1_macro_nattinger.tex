% !TEX TS-program = pdflatex
% !TEX encoding = UTF-8 Unicode

% This is a simple template for a LaTeX document using the "article" class.
% See "book", "report", "letter" for other types of document.

\documentclass[11pt]{article} % use larger type; default would be 10pt

\usepackage[utf8]{inputenc} % set input encoding (not needed with XeLaTeX)

%%% PAGE DIMENSIONS
\usepackage{geometry} % to change the page dimensions
\geometry{a4paper} % or letterpaper (US) or a5paper or....
\geometry{margin=1in} 
\usepackage{graphicx} % support the \includegraphics command and options

\usepackage{amssymb}
\usepackage{amsmath}
%%% PACKAGES
\usepackage{booktabs} % for much better looking tables
\usepackage{array} % for better arrays (eg matrices) in maths
\usepackage{paralist} % very flexible & customisable lists (eg. enumerate/itemize, etc.)
\usepackage{verbatim} % adds environment for commenting out blocks of text & for better verbatim
\usepackage{subfig} % make it possible to include more than one captioned figure/table in a single float
% These packages are all incorporated in the memoir class to one degree or another...

%%% HEADERS & FOOTERS
\usepackage{fancyhdr} % This should be set AFTER setting up the page geometry
\pagestyle{fancy} % options: empty , plain , fancy
\renewcommand{\headrulewidth}{0pt} % customise the layout...
\lhead{}\chead{}\rhead{}
\lfoot{}\cfoot{\thepage}\rfoot{}

%%% SECTION TITLE APPEARANCE
\usepackage{sectsty}
\allsectionsfont{\sffamily\mdseries\upshape} % (See the fntguide.pdf for font help)
% (This matches ConTeXt defaults)

%%% ToC (table of contents) APPEARANCE
\usepackage[nottoc,notlof,notlot]{tocbibind} % Put the bibliography in the ToC
\usepackage[titles,subfigure]{tocloft} % Alter the style of the Table of Contents
\renewcommand{\cftsecfont}{\rmfamily\mdseries\upshape}
\renewcommand{\cftsecpagefont}{\rmfamily\mdseries\upshape} % No bold!
\usepackage{graphicx}
\graphicspath{ {./pings/} }

\usepackage{amsmath}
\DeclareMathOperator*{\argmax}{arg\,max}
\DeclareMathOperator*{\argmin}{arg\,min}

\newcount\colveccount
\newcommand*\colvec[1]{
        \global\colveccount#1
        \begin{pmatrix}
        \colvecnext
}
\def\colvecnext#1{
        #1
        \global\advance\colveccount-1
        \ifnum\colveccount>0
                \\
                \expandafter\colvecnext
        \else
                \end{pmatrix}
        \fi
}

%%% END Article customizations

%%% The "real" document content comes below...

\title{Macro PS1}
\author{Michael B. Nattinger}

%\date{} % Activate to display a given date or no date (if empty),
         % otherwise the current date is printed 

\begin{document}
\maketitle
\section{Question 1: Exercise 8.1}
The first order conditions to the Pareto problem is the following:
\begin{align*}
\theta u'(c^1) &= \lambda\\
(1-\theta)w'(c^2) &= \lambda\\
\Rightarrow \theta u'(c^1) &= (1-\theta)w'(c^2) .
\end{align*}

From envelope conditions, we get the following:
\begin{align*}
v'_{\theta}(c) &= \theta u'(c^1) \frac{\partial  c^1}{\partial c} + (1-\theta)w'(c^2)  \frac{\partial  c^2}{\partial c} \\
&= \theta u'(c^1) \frac{\partial  (c^1+c^2)}{\partial c}\\
&= \theta u'(c^1) = (1-\theta)w'(c^2) .
\end{align*}

Now we will think about concavity. This is slightly more involved as the envelope theorem holds only for the first derivative.

Define the compact set $B(c) = \{x= (c_1,c_2) \in \mathbb{R}^2: c_1+c_2\leq c, c_1\geq 0, c_2\geq 0\}$. Define on this set the function $V(x) = \theta u(c^1) + (1-\theta)c^2$. Then, observe that $v(c) = \max_{x\in B(c)}V(x)$. Since $u,w$ are continuous, $V$ is continuous so it achieves its maximum on the compact set $B(c)$. Define $X(c)$ as the corresponding argmax - since $V$ is strictly concave, it achieves its max at a unique point. Now, let $c,C \geq 0,\lambda \in [0,1]$. Then,
\begin{align*}
\lambda v(c) + (1-\lambda)v(C) &= \lambda V(X(c)) + (1-\lambda) V(X(C))\\
&\leq V(\lambda X(c) + (1-\lambda) X(C))\\
&\leq v(\lambda c + (1-\lambda) C).
\end{align*}
Therefore, $v(c)$ is concave.
%Since $u,w$ are twice differentiable, from the envelope theorem $v_{\theta}$ is twice differentiable and $v''_{\theta} = \theta u''(c^1) + (1-\theta) u''(c^2)<0$ since $u,v$ are concave. Thus, $v_{\theta}$ is concave.

%Now we will show $v_{\theta}(c)$ is concave.
%
%Let $c,q \in \mathbb{R}^+$. Define $c^1,c^2,q^1,q^2$ be the related arg max's of $c,q$. Let $\lambda \in [0,1]$ Then,
%
%\begin{align*}
%v_{\theta}(c) = \theta u(c^1) + (1-\theta)w(c^2)
%\end{align*}
\section{Question 2: Exercise 8.3}
\subsection{Part A}
A competitive equilibrium is a set of prices $\{Q_t\}_{t=0}^{\infty}$ and allocations $\{ c_t^1,c_t^2 \}_{t=0}^{\infty}$ such that both consumers optimize (maximize the sum of discounted utility) and markets clear ($c^1_t + c^2_t = y_t^1 + y_t^2 = 1 \forall t$).
\subsection{Part B}
Agent $i$ solves the following optimization problem:
\begin{align*}
&\max_{\{ c_{t}^i\}_{t=0}^{\infty}} \sum_{i=1}^{\infty}\beta^t u(c_{t}^i)\\
&\text{s.t.} \sum_{t=0}^{\infty}Q_tC_{t}^i \leq \sum_{t=0}^{\infty} Q_t y_{t}^i
\end{align*}

Denoting the Lagrange multiplier of agent $i$ as $\mu_i$, first order conditions take the following form:

\begin{align*}
\beta u'(c_t^i) &= \mu_i Q_t\\
\Rightarrow \frac{u'(c_t^1)}{u'(c_t^1)} &= \mu_i/\mu_j
\end{align*}

Note that the right hand side is independent of $t$, and since the total endowment of the economy is also constant (1), the consumption of each agent must also be constant for all time, i.e. $c_t^1 = c^1, c_t^2 = c^2.$ Market clearing also implies $c^1+c^2 = 1.$

Moreover, the first order conditions also yield the following:
\begin{align*}
\beta\frac{u'(c_{t+1}^1)}{u'(c_{t}^1)} &= \frac{Q_{t+1}}{Q_t}
\end{align*}

Constant consumption implies that $Q_{t+1} = \beta Q_t$. We can normalize $Q_0 = 1$ and then we have that $Q_{t} = \beta^t$. Now we have the following:
\begin{align*}
\sum_{t=0}^{\infty}\beta^t c^1 &= \sum_{t=0}^{\infty}\beta y_{t}^i\\
\frac{c^1}{1-\beta} &= \frac{1}{1-\beta^3}\\
\Rightarrow c^1 &= \frac{1-\beta}{1-\beta^3},\\
c^2 &= \frac{\beta - \beta^3}{1-\beta^3}.
\end{align*}

\subsection{Part C}

We can price the asset $p^A$ using $Q_t = \beta^t$:
\begin{align*}
p^A &= \sum_{i=0}^{\infty}\frac{\beta^t}{20}\\
&= \frac{1}{20(1-\beta)}.
\end{align*}

\section{Question 3: Exercise 8.4}
\subsection{Part I}
\subsubsection{Part A}
A competitive equilibrium is a set of prices $\{Q_t(s^t)\}_{t=0}^{\infty}$ and allocations $\{c_t(s^t)\}_{t=0}^{\infty}$ such that agents optimize and markets clear ($c_t(s^t) =d_t(s^t)$).


I will quickly derive first order conditions that will help us later.

The agent maximizes:
\begin{align*}
\max E_0 \sum_{t=0}^{\infty}\frac{c_t^{1-\gamma}}{1-\gamma}\\
\text{s.t.} \sum_{t=0}^{\infty}\sum_{s^t}Q_t(s^t)c_t(s^t) \leq \sum_{t=0}^{\infty}Q_t(s^t)d_t(s^t)
\end{align*}

FOC (lagrange multiplier $\mu$):
\begin{align*}
\beta^t\pi_t(s^t)u'(c_t(s^t)) &= \mu Q_t(s^t)\\
\Rightarrow \frac{\beta^t \pi_t(s^t)u'(c_t(s^t))}{u'(c_0(s_0))} &= Q_{t}(s^t)
\end{align*}

\begin{align}
(0.95)^t \pi_t(s^t)(d_t(s^t))^{-2} &=  Q_{t}(s^t) \label{Q}
\end{align}

We can use the above expression to price claims in the sections that follow. First, note that $c_t\leq d_t \Rightarrow c_t = d_t$ will maximize utility. Also, note that we are interested in prices in terms of the period $0$ good, i.e. we have assumed $Q_0(s_0) = 0.$ Note finally that $u'(c_0) = u'(d_0) = u'(1) = 1.$
\subsubsection{Part B}
Using equation (\ref{Q}), we can price the claim. $c_5 = d_5 = 0.97*0.97*1.03*0.97*1.03 = 0.968$. $\beta^5 = 0.774$. $\pi_t(s^t) = 0.8*0.8*0.2*0.1*0.2 = 0.00256$. Therefore, $Q_5 = ( 0.774)(0.00256)(0.968)^{-2} = 0.00211$.
\subsubsection{Part C}
Using equation (\ref{Q}), we can price the claim. $c_5 = d_5 = 1.03*1.03*1.03*1.03*0.97 = 1.092$. $\beta^5 = 0.774$. $\pi_t(s^t) = 0.2*0.9*0.9*0.9*0.1 = 0.01458$. Therefore, $Q_5 =  ( 0.774)(0.01458)(1.092)^{-2} = 0.00946$.
\subsubsection{Part D}
The price is the sum of the prices and endowments across states and time:
\begin{align*}
P^e &= \sum_{t=0}^{\infty}\sum_{s^t} d_t(s^t)Q_{t}(s^t)\\
&= \sum_{t=0}^{\infty}\sum_{s^t} (0.95)^t \pi_t(s^t)(d_t(s^t))^{-1}
\end{align*} 
\subsubsection{Part E}
The price is the sum of the prices and endowments across state histories at time $5$, conditional on the state at time $t=5$ being $\lambda_5 = 0.97$:
\begin{align*}
P^5 &= \sum_{s^5|s_5= 0.97} (0.95)^5 \pi_5(s^5)(d_t(s^5))^{-1}
\end{align*}
\subsection{Part II}
\subsubsection{Part F}
A recursive competitive equilibrium is a pricing kernel $\{q_t(s^t|s_{t+1})\}_{t=0}^{\infty}$ and decision rules $ c(s_t,a_{t}),a_{t+1}(s_t,a_{t}) $ such that agents optimize ($v(s_t,a_t) = \max_{c,a_{t+1}} u(c) + \beta E[v(s_{t+1},a_{t+1})]$) and markets clear $c_t  = d_t, a_{t} = 0 \forall t.$
\subsubsection{Part G}
The natural debt limit for a state in the future $A_{t+1}(s^t,s_{t+1})$ is the maximum amount one can repay eventually, i.e. present discounted value of future income. It takes a recursive form:
\begin{align*}
A(s_t) = d_{t} + \beta\sum_{s_{t+1}}Q(s_{t+1}|s_t)A(s_{t+1})
\end{align*}
\subsubsection{Part H}
In each period, the agent solves the following maximization problem: 
\begin{align*}
&\max_{c_t(s^t),\{a_{t+1}(s^t,s_{t+1})\}} E_0\sum_{t=0}^{\infty} \beta^t u(c_t(s^t))\\
&\text{s.t.} c_t(s^t) + \sum_{s_t}a_{t+1}(s^t,s_{t+1})q_t(s^t,s_{t+1}) \leq d_{t}(s^t) + a_t(s^t)
\end{align*}
Taking first order conditions, we have the following:
\begin{align*}
q_t(s^t,s_{t+1}) &= \beta \frac{u'(c_{t+1}(s^t,s_{t+1}))}{u'(c_t(s^t))}\pi(s_{t+1}|s^t)
\end{align*}

Since the endowments are governed by a Markov process, and since we know that the feasible allocations satisfy $c_t \leq d_t \Rightarrow c_t = d_t$ optimizes utility, we can rewrite the first order conditions as follows:
\begin{align*}
q_t(s^t,s_{t+1}) &= \beta \frac{u'(d_{t+1}(s^t,s_{t+1}))}{u'(d_t(s^t))}\pi(s_{t+1}|s_t)\\
&= \beta \left(\frac{d_t(s^t)}{d_{t+1}(s^t,s_{t+1})}\right)^2 \pi(s_{t+1}|s_t)\\
&= \beta (\lambda_{t+1}(s_{t+1}))^{-2}\pi(s_{t+1}|s_t)\\
&= q_t(s_t,s_{t+1}).
\end{align*}
The above expression is our pricing kernel.

Finally, since we know $c_t(s^t) = d_t(s^t)$, it must immediately hold by induction that $a_t(s^t) = 0.$

\subsubsection{Part I}
We can use our pricing kernel to price this bond.

\begin{align*}
p^b(s_t) &= \sum_{s^{t+1}}\sum_{s^{t+2}} \beta^2  (\lambda_{t+1}(s_{t+1}))^{-2}\pi(s_{t+1}|s_t)(\lambda_{t+2}(s_{t+2}))^{-2}\pi(s_{t+2}|s_{t+1})  \\
\Rightarrow p^b(\lambda_t) &= \begin{cases} (0.95)^2 ((0.97)^{-2}(0.8)(0.97^{-2}(0.8) + (1.03)^{-2}(0.2)) + (1.03)^{-2}(0.2)((0.9)(1.03)^{-2}+(0.1)(0.97)^{-2}) ) , \lambda_t = 0.97 \\(0.95)^2 ((0.97)^{-2}(0.1)(0.97^{-2}(0.8) + (1.03)^{-2}(0.2)) + (1.03)^{-2}(0.9)((0.9)(1.03)^{-2}+(0.1)(0.97)^{-2}) )  , \lambda_t = 1.03\end{cases},\\
p^b(\lambda_t) &= \begin{cases} 0.96 , \lambda_t = 0.97 \\ 0.83 , \lambda_t = 1.03\end{cases}.
\end{align*}
\end{document}
