% !TEX TS-program = pdflatex
% !TEX encoding = UTF-8 Unicode

% This is a simple template for a LaTeX document using the "article" class.
% See "book", "report", "letter" for other types of document.

\documentclass[11pt]{article} % use larger type; default would be 10pt

\usepackage[utf8]{inputenc} % set input encoding (not needed with XeLaTeX)

%%% PAGE DIMENSIONS
\usepackage{geometry} % to change the page dimensions
\geometry{a4paper} % or letterpaper (US) or a5paper or....
\geometry{margin=1in} 
\usepackage{graphicx} % support the \includegraphics command and options

\usepackage{amssymb}
\usepackage{amsmath}
%%% PACKAGES
\usepackage{booktabs} % for much better looking tables
\usepackage{array} % for better arrays (eg matrices) in maths
\usepackage{paralist} % very flexible & customisable lists (eg. enumerate/itemize, etc.)
\usepackage{verbatim} % adds environment for commenting out blocks of text & for better verbatim
\usepackage{subfig} % make it possible to include more than one captioned figure/table in a single float
% These packages are all incorporated in the memoir class to one degree or another...

%%% HEADERS & FOOTERS
\usepackage{fancyhdr} % This should be set AFTER setting up the page geometry
\pagestyle{fancy} % options: empty , plain , fancy
\renewcommand{\headrulewidth}{0pt} % customise the layout...
\lhead{}\chead{}\rhead{}
\lfoot{}\cfoot{\thepage}\rfoot{}

%%% SECTION TITLE APPEARANCE
\usepackage{sectsty}
\allsectionsfont{\sffamily\mdseries\upshape} % (See the fntguide.pdf for font help)
% (This matches ConTeXt defaults)

%%% ToC (table of contents) APPEARANCE
\usepackage[nottoc,notlof,notlot]{tocbibind} % Put the bibliography in the ToC
\usepackage[titles,subfigure]{tocloft} % Alter the style of the Table of Contents
\renewcommand{\cftsecfont}{\rmfamily\mdseries\upshape}
\renewcommand{\cftsecpagefont}{\rmfamily\mdseries\upshape} % No bold!
\usepackage{graphicx}
\graphicspath{ {./pings/} }

\usepackage{amsmath}
\DeclareMathOperator*{\argmax}{arg\,max}
\DeclareMathOperator*{\argmin}{arg\,min}

\newcount\colveccount
\newcommand*\colvec[1]{
        \global\colveccount#1
        \begin{pmatrix}
        \colvecnext
}
\def\colvecnext#1{
        #1
        \global\advance\colveccount-1
        \ifnum\colveccount>0
                \\
                \expandafter\colvecnext
        \else
                \end{pmatrix}
        \fi
}

%%% END Article customizations

%%% The "real" document content comes below...

\title{Macro PS2}
\author{Michael B. Nattinger}

%\date{} % Activate to display a given date or no date (if empty),
         % otherwise the current date is printed 

\begin{document}
\maketitle
\section{Question 1}
\subsection{Part A}
%Let us assume we are in the constrained efficient allocation and we will show that this holds as an equilibrium of an environment in which agents trade a risk free bond subject to endogenous debt constraints. We proceed in almost exactly the same fashion as section 8.8.5 in Lundqvist and Sargent.
%
%Take $q^0_t(s^t)$ as given from the A-D eqm and suppose the pricing kernel makes the following true:
%\begin{align*}
%q^0_{t+1}(s^{t+1}) &= Q_t(s_{t+1}|s^t)q_t^0(s^t),\\
%\text{or} Q_t(s_{t+1}|s^t) &= q_{t+1}^t(s^{t+1}).
%\end{align*}
%Let $\{c_t^i(s^t) \}$ be our A-D equilibrium consumption stream. 
The equilibrium is a set of prices $R$ and allocations $c^h,c^l$ such that the allocations solve the agents' problem and markets clear.

Agents maximize utility subject to the budget constraint and the endogenous debt constraint:
\begin{align*}
&\max \sum_{t=0}^{\infty} \beta^t u(c_{it})\\
&\text{s.t.} c_{it} + b_{it+1} = e_{it} + R_tb_{it}\\
&\text{and } b_{it+1} \geq -\phi.
\end{align*}
This constraint will bind on the low types, which will determine their consumption via their budget constraint. The high types will follow their Euler equations.

Market clearing is the following: $b_{it} + b_{jt} = 0$. If we let $R = \frac{1}{\beta}\frac{9}{10},\phi = \frac{5}{1+R}$ then the agents will choose the efficient allocation.

The Euler for the high type is the following:
\begin{align*}
u'(c^h) &= \beta R u'(c^l)\\
\frac{1}{c^h} &= \frac{\beta R}{c^l}
\end{align*}

Notice that $c^h = 10,c^l = 9 $ satisfies the Euler under the interest rate given above.

We also need to check feasibility: From market clearing $c_h = 15-\phi(1+R) = 10,c_l = 4 + \phi(1+R) = 9.$ Thus, this satisfies the budget constraints with equality. By Walras' law the aggregate resource constraint also clears. Therefore, the constrained efficient allocation holds as an equilibrium of the environment described.

\subsection{Part B}
A second outcome arises from this setup: autarky. If $\phi = 0$, $c^h = e^h,c^l = e^l,R = \frac{e^l}{\beta e^h}$ then the high type's Euler is still satisfied, and budget constraints are satisfied trivially, and finally the resource constraint clears by Walras' law.

\section{Question 2}
\subsection{Part A}
In this scenario, if the agent defaults then they can still save. They will choose the optimal savings:
\begin{align*}
&\max_s  log(e^h - s) + \beta log(e^l + Rs) + \beta^2log(e^h - s) + \beta^3log(e^l + Rs) + \dots\\
&\max_s \frac{log(e^h - s) + \beta log(e^l + Rs)}{1-\beta^2}\\
&\frac{1}{e^h - s} =\frac{\beta R}{e^l + Rs}\\
&e^l + Rs = \beta R e^h - \beta R s\\
& s =\frac{\beta R e^h - e^l}{R+\beta R} \\
&\Rightarrow V^d(h) = \frac{log\left(\frac{Re^h + e^l}{R+\beta R}\right) + \beta log\left( \frac{e^l \beta R + \beta R e^h }{1+\beta }\right)}{1-\beta^2}
\end{align*}
\subsection{Part B}
A competitive equilibrium with not-too-tight constraints is an allocation $c^l,c^h $ and set of prices $R$ and constraint $\phi$ such that agents optimize, markets clear, and the not-too-tight constraint is satisfied. Agents solve the following:
\begin{align*}
&\max_{c^i_t,B^i_t} \sum_{t=1}^{\infty} \beta^t log(c_t^i)\\
&\text{s.t.} c_t^i + B^i_{t+1} = e_t^i + RB^i_t\\
&\text{and } B^{i}_{t+1} \geq -\phi.
\end{align*}
The not-too-tight constraint is the following:
\begin{align*}
\frac{log(c^h)+\beta log(c^l)}{1-\beta^2} = V^d(h) 
\end{align*}

Market clearing is the following:
\begin{align*}
c^h + c^l &= e^h + e^l\\
B^h + B^l &= 0.
\end{align*}
\subsection{Part C}
The constraint on borrowing is going to bind on the low type, i.e. $c^l= \phi(1+R) + e^l \Rightarrow c^h = e^h - \phi(1+R)$. We can plug this into our not-too-tight constraint:
\begin{align}
\frac{log(e^h - \phi(1+R))+\beta log(\phi(1+R) + e^l)}{1-\beta^2} &= V^d(h) \label{con}
\end{align}
The high type is not constrained so their Euler equation must hold:
\begin{align}
\frac{1}{e^h - \phi(1+R)} &= \frac{\beta R}{e^l + \phi(1+R)} \label{ee}
\end{align}
Our constraint equation (\ref{con}) and Euler (\ref{ee}) yield 2 equations in 2 unknowns which we can solve for $R,\phi$ which yield the equilibrium.
\subsection{Part D}
We can use (\ref{ee}) to solve for $\phi$:
\begin{align*}
e^l + \phi(1+R)&= \beta R e^h - \beta R \phi(1+R)\\
\phi &= \frac{\beta R e^h - e^l}{(1+\beta R)(1+R)}
\end{align*}
Rewriting (\ref{con}) we get the following:
\begin{align*}
log(e^h - \phi(1+R))+\beta log(\phi(1+R) + e^l) &= log\left(\frac{Re^h + e^l}{R+\beta R}\right) + \beta log\left( \frac{e^l \beta R + \beta R e^h }{1+\beta }\right) \\
log\left(e^h - \frac{\beta R e^h - e^l}{(1+\beta R)}\right)+\beta log\left( \frac{\beta R e^h - e^l}{(1+\beta R)} + e^l\right) &= log\left(\frac{Re^h + e^l}{R+\beta R}\right) + \beta log\left( \frac{e^l \beta R + \beta R e^h }{1+\beta }\right) \\
log\left(\frac{e^h + e^l}{(1+\beta R)}\right)+\beta log\left( \frac{\beta R e^h + \beta Re^l}{(1+\beta R)} \right) &= log\left(\frac{Re^h + e^l}{R+\beta R}\right) + \beta log\left( \frac{e^l \beta R + \beta R e^h }{1+\beta }\right) 
\end{align*}
It is clear that $R=1$ satisfies the above equation. Then, $\phi = \frac{\beta e^h - e^l}{2(1+\beta )} $. Note that autarky, $\phi =0,R=\frac{e^l}{\beta e^h}$ also satisfies the equilibrium.
\subsection{Part E}
In class we solved the autarky case so I will jump to the solution in that case: $(c^h,c^l) = (10,9)$. For the same calibrations for the model in this question we have $ c^h = e^h - \phi(1+R) = 15 - \frac{0.5(15) - 4}{1.5} = 12 +\frac{2}{3}, c^l = 19-c^h = 6 + \frac{1}{3}$. This model yields less consumption smoothing compared to autarky.
\subsection{Part F}
The larger the punishment, the more consumption smoothing we can sustain.
\end{document}
