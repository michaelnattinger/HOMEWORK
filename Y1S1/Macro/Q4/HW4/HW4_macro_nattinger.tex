% !TEX TS-program = pdflatex
% !TEX encoding = UTF-8 Unicode

% This is a simple template for a LaTeX document using the "article" class.
% See "book", "report", "letter" for other types of document.

\documentclass[11pt]{article} % use larger type; default would be 10pt

\usepackage[utf8]{inputenc} % set input encoding (not needed with XeLaTeX)

%%% PAGE DIMENSIONS
\usepackage{geometry} % to change the page dimensions
\geometry{a4paper} % or letterpaper (US) or a5paper or....
\geometry{margin=1in} 
\usepackage{graphicx} % support the \includegraphics command and options

\usepackage{amssymb}
\usepackage{amsmath}
%%% PACKAGES
\usepackage{booktabs} % for much better looking tables
\usepackage{array} % for better arrays (eg matrices) in maths
\usepackage{paralist} % very flexible & customisable lists (eg. enumerate/itemize, etc.)
\usepackage{verbatim} % adds environment for commenting out blocks of text & for better verbatim
\usepackage{subfig} % make it possible to include more than one captioned figure/table in a single float
% These packages are all incorporated in the memoir class to one degree or another...

%%% HEADERS & FOOTERS
\usepackage{fancyhdr} % This should be set AFTER setting up the page geometry
\pagestyle{fancy} % options: empty , plain , fancy
\renewcommand{\headrulewidth}{0pt} % customise the layout...
\lhead{}\chead{}\rhead{}
\lfoot{}\cfoot{\thepage}\rfoot{}

%%% SECTION TITLE APPEARANCE
\usepackage{sectsty}
\allsectionsfont{\sffamily\mdseries\upshape} % (See the fntguide.pdf for font help)
% (This matches ConTeXt defaults)

%%% ToC (table of contents) APPEARANCE
\usepackage[nottoc,notlof,notlot]{tocbibind} % Put the bibliography in the ToC
\usepackage[titles,subfigure]{tocloft} % Alter the style of the Table of Contents
\renewcommand{\cftsecfont}{\rmfamily\mdseries\upshape}
\renewcommand{\cftsecpagefont}{\rmfamily\mdseries\upshape} % No bold!
\usepackage{graphicx}
\graphicspath{ {./pings/} }

\usepackage{amsmath}
\DeclareMathOperator*{\argmax}{arg\,max}
\DeclareMathOperator*{\argmin}{arg\,min}

\newcount\colveccount
\newcommand*\colvec[1]{
        \global\colveccount#1
        \begin{pmatrix}
        \colvecnext
}
\def\colvecnext#1{
        #1
        \global\advance\colveccount-1
        \ifnum\colveccount>0
                \\
                \expandafter\colvecnext
        \else
                \end{pmatrix}
        \fi
}

%%% END Article customizations

%%% The "real" document content comes below...

\title{Macro PS4}
\author{Michael B. Nattinger}

%\date{} % Activate to display a given date or no date (if empty),
         % otherwise the current date is printed 

\begin{document}
\maketitle
\section{Question 1}
The household solves the following problem:
\begin{align*}
&\max_{\{ c_t,l_t \}_{t=0}^{\infty}} \beta^t\left[ \frac{c_t^{1-\sigma}}{1-\sigma} + \nu(l_t) \right] \\
&\text{s.t. } (1+\tau_{ct})c_t +k_{t+1} + b_{t+1} = (1-\delta + r_t)k_t + R_tb_t +w_t(1-l_t)
\end{align*}

Taking FOCs (Lagrange multiplier $\lambda_t$) yields the following:
\begin{align*}
\beta^t c_t^{-\sigma} &= \lambda_t(1+\tau_{ct})\\
\beta^t \nu'(l_t) &=  \lambda_t w_t\\
\lambda_t &=  (1-\delta + r_{t+1}) \lambda_{t+1} \\
\lambda_t &=  R_{t+1}\lambda_{t+1}.
\end{align*}

Simplifying,
\begin{align*}
w_t &= (1+\tau_{ct})\nu'(l_t)c_t^{\sigma} \\
1&=\beta\left(\frac{c_t}{c_{t+1}}\right)^{\sigma}\frac{1+\tau_{ct+1}}{1+\tau_{ct}}R_{t+1}\\
R_{t} &= 1-\delta + r_t.
\end{align*}

The above formulas represent the solution to the household's problem. Now we can set up the Ramsey problem. The resource constraint is the following:
\begin{align*}
c_t + k_{t+1} &= (1-\delta)k_t + F(k_t,(1-l_t)).
\end{align*}

Our implementability constraint takes the following form:
\begin{align*}
\sum_{t=0}^{\infty}\beta^t [c_t^{1-\sigma} - (1-l_t)\nu'(l_t) ] &= \frac{c_0^{-\sigma}}{1+\tau_{c0}}[(1-\delta + r_0)k_{-1} + R_0b_{-1}],
\end{align*}
where we make standard assumptions on $\tau_{c0}$ to rule out the effective lump-sum tax solution.

Defining $W(c_t,l_t,\lambda) =  \frac{c_t^{1-\sigma}}{1-\sigma} + \nu(l_t) + \lambda[c_t^{1-\sigma} - (1-l_t)\nu'(l_t)] $, then the Ramsey problem consists of solving the following maximization problem:

\begin{align*}
\max \sum_t \beta^t[W(c_t,l_t,\lambda) - \lambda \frac{c_0^{-\sigma}}{1+\tau_{c0}}[(1-\delta+r_0)k_{-1} + R_{0}b_{-1}]
\end{align*}

The intertemporal first order condition is the following:
\begin{align*}
W_{ct} &= \beta W_{ct+1} R_{t+1}
\end{align*}

We are almost to our solution. We just need to look at $W_{ct}$:
\begin{align*}
c_t^{-\sigma} +\lambda c_t^{-\sigma}(1-\sigma) &= c_t^{-\sigma}(1+\lambda(1-\sigma))\\
\Rightarrow \frac{W_{ct+1}}{W_{ct}} &= \left(\frac{c_t}{c_{t+1}}\right)^{\sigma}
\end{align*}

Coming back to our intertemporal FOC for the Ramsey problem,
\begin{align*}
W_{ct} &= \beta W_{ct+1} R_{t+1} \\
1  &=\beta\left(\frac{c_t}{c_{t+1}}\right)^{\sigma}R_{t+1}
\end{align*}

Comparing to our first order conditions for the HH problem we can see immediately that $\tau_{ct} = \tau_{ct+1} \forall t$.
\section{Question 2}
\subsection{Part A}
A competitive equilibrium is a set of prices $\{ p_t,w_t,R_t \}$ and allocations $\{ c_{1t},c_{2t},n_t,M_t,B_t,T_t \}$ such that agents optimize, markets clear, and the government budget constraint holds.

Agents solve the following:
\begin{align*}
&\max_{c_{1t},c_{2t},n_t}\sum_{t=0}^{\infty}\beta^t( log(c_{1t}) + \alpha log(c_{2t}) + \gamma log(1-n_t))\\
&\text{s.t. } p_{t}c_{1t} \leq M_t\\
&\text{and }M_t + B_t \leq (M_{t-1} - p_{t-1}c_{1t-1}) - p_{t-1}c_{2t-1} + w_{t-1}n_{t-1}  + R_{t-1}B_{t-1} - T_{t}
\end{align*}

Market clearing is the following:
\begin{align*}
c_{1t} + c_{2t} &= n_t.
\end{align*}

GBC:
\begin{align*}
M_t - M_{t-1} + B_t &= R_{t-1}B_{t-1} - T_t
\end{align*}

Finally, we are given that monetary policy acts to make $R_t = R$ $\forall t.$
\subsection{Part B}
The consumer's FOCs are the following:
\begin{align*}
\beta^t c_{1t}^{-1} &= \lambda_{t+1} p_{t} + \gamma_t p_t\\
\beta^t \alpha c_{2t}^{-1} &= \lambda_{t+1}p_{t}\\
\beta^t \gamma (1-n_t)^{-1} &= \lambda_{t+1}w_t\\
\lambda_t &= \gamma_t + \lambda_{t+1} \\
\lambda_{t} &= \lambda_{t+1} R_{t}.
\end{align*}

Simplifying,
\begin{align*}
\beta^t c_{1t}^{-1} &= p_t\lambda_t\\
\Rightarrow \frac{c_{2t}}{\alpha c_{1t}} &= R_t = R, \\
\frac{\gamma c_{2t}}{\alpha(1-n_t)} &= \frac{w_t}{p_t}
\end{align*}

Note that the real wage $\frac{w_t}{p_t}$ must be one because that is the marginal productivity of labor from the firm side. We now have 3 equations in 3 unkowns we can solve for allocations:
\begin{align}
\frac{c_{2t}}{\alpha c_{1t}} &= R, \\
\frac{\gamma c_{2t}}{\alpha(1-n_t)} &= 1, \\
c_{1t} + c_{2t} &= n_t.
\end{align}

\begin{align*}
n_t &= 1 - \frac{\gamma c_{2t}}{\alpha},\\
c_{2t} &= \alpha Rc_{1t},\\
c_{1t}(1+\alpha R) &= 1 - \gamma  Rc_{1t} \\
\Rightarrow c_{1t} &= \frac{1}{1+(\alpha+\gamma) R} \\
\Rightarrow c_{2t} &= \frac{\alpha R}{1+(\alpha+\gamma) R} \\
\Rightarrow n_t &= 1 - \frac{\gamma R}{1+(\alpha+\gamma) R} \\
&= \frac{1+\alpha R}{1+(\alpha+\gamma) R}
\end{align*}
From Wolfram Alpha,
\begin{align*}
\frac{\partial n_t}{\partial R} &= - \frac{\gamma}{(1+(\alpha+\gamma)R)^2}\\
&<0.
\end{align*}
Therefore, $n_t$ is decreasing in $R$.
\end{document}
