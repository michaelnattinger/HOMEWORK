% !TEX TS-program = pdflatex
% !TEX encoding = UTF-8 Unicode

% This is a simple template for a LaTeX document using the "article" class.
% See "book", "report", "letter" for other types of document.

\documentclass[11pt]{article} % use larger type; default would be 10pt

\usepackage[utf8]{inputenc} % set input encoding (not needed with XeLaTeX)

%%% PAGE DIMENSIONS
\usepackage{geometry} % to change the page dimensions
\geometry{a4paper} % or letterpaper (US) or a5paper or....

\usepackage{graphicx} % support the \includegraphics command and options

\usepackage{amssymb}
\usepackage{amsmath}
%%% PACKAGES
\usepackage{booktabs} % for much better looking tables
\usepackage{array} % for better arrays (eg matrices) in maths
\usepackage{paralist} % very flexible & customisable lists (eg. enumerate/itemize, etc.)
\usepackage{verbatim} % adds environment for commenting out blocks of text & for better verbatim
\usepackage{subfig} % make it possible to include more than one captioned figure/table in a single float
% These packages are all incorporated in the memoir class to one degree or another...

%%% HEADERS & FOOTERS
\usepackage{fancyhdr} % This should be set AFTER setting up the page geometry
\pagestyle{fancy} % options: empty , plain , fancy
\renewcommand{\headrulewidth}{0pt} % customise the layout...
\lhead{}\chead{}\rhead{}
\lfoot{}\cfoot{\thepage}\rfoot{}

%%% SECTION TITLE APPEARANCE
\usepackage{sectsty}
\allsectionsfont{\sffamily\mdseries\upshape} % (See the fntguide.pdf for font help)
% (This matches ConTeXt defaults)

%%% ToC (table of contents) APPEARANCE
\usepackage[nottoc,notlof,notlot]{tocbibind} % Put the bibliography in the ToC
\usepackage[titles,subfigure]{tocloft} % Alter the style of the Table of Contents
\renewcommand{\cftsecfont}{\rmfamily\mdseries\upshape}
\renewcommand{\cftsecpagefont}{\rmfamily\mdseries\upshape} % No bold!
\usepackage{graphicx}
\graphicspath{ {./pings/} }

\usepackage{amsmath}
\DeclareMathOperator*{\argmax}{arg\,max}
\DeclareMathOperator*{\argmin}{arg\,min}

\newcount\colveccount
\newcommand*\colvec[1]{
        \global\colveccount#1
        \begin{pmatrix}
        \colvecnext
}
\def\colvecnext#1{
        #1
        \global\advance\colveccount-1
        \ifnum\colveccount>0
                \\
                \expandafter\colvecnext
        \else
                \end{pmatrix}
        \fi
}

%%% END Article customizations

%%% The "real" document content comes below...

\title{Macro PS2}
\author{Michael B. Nattinger\footnote{I worked on this assignment with my study group: Alex von Hafften, Andrew Smith, and Ryan Mather. I have also discussed problem(s) with Emily Case, Sarah Bass, and Danny Edgel.}}

%\date{} % Activate to display a given date or no date (if empty),
         % otherwise the current date is printed 

\begin{document}
\maketitle
\section{Question 1}
\subsection{Part A}
%First we can solve the household side. 
%
%We can write our Bellman equation as follows, and then divide capital variables by A to normalize, yielding lowercase variables:
%\begin{align*}
%V(K) = &\max_{C,K',N} u(c,1-N) + \beta V(K')\\
%&\text{s.t. } K' = (1-\delta)K + Y - C \\
%&\text{and }Y = F(K,AN)\\
%\Rightarrow V(k) = &\max_{c,k',N} u(c,1-N) + \beta(1+g) V(k') \\
%&\text{s.t. } k' = (1-\delta)k + y - c \\
%&\text{and }y = F(k,N)\\
%\Rightarrow V(k) &= \max_{c,N} u(c,1-N) + \beta (1+g) V((1-\delta)K + F(K,AN) - c)
%\end{align*}
%Taking FOCs and applying the envelope condition we find:
%\begin{align*}
%0 &= u_1(c,1-N) - \beta(1+g) V'((1-\delta)k + F(k,N) - c)\\
%V'(k) &= \beta(1+g) V'((1-\delta)k + F(k,N) - c)(1-\delta + F_1(k,N))\\
%\Rightarrow V'(k) &=u_1(c,1-N) (1-\delta + F_1(k,N)) \\
%\Rightarrow u_1(c,1-N) &= \beta(1+g) u_1(c,1-N) (1-\delta + F_1((1-\delta)k + F(k,N) - c,N))\\
%\Rightarrow F_1((1-\delta)k + F(k,N) - c,N) &= \frac{1}{\beta(1+g)} - 1 + \delta
%\end{align*}
%\begin{equation}
%\Rightarrow F_1(k',N) = \frac{1}{\beta(1+g)} - 1 + \delta \label{eqn:FOC1}
%\end{equation}
%
%\begin{align*}
%0 &= -u_2(c,1-N) - \beta(1+g) V'((1-\delta)k + F(k,N) - c)(F_2(k,N))\\
%\Rightarrow u_2(c,1-N) &= - \beta(1+g) u_1(c,1-N) (1-\delta + F_1((1-\delta)k + F(k,N) - c,N))(F_2(k,N))
%\end{align*}
%\begin{equation}
%\frac{ u_2(c,1-N)}{u_1(c,1-N)} = - \beta(1+g)(1-\delta + F_1(K',AN))(AF_2(K,AN)) \label{eqn:FOC2}
%\end{equation}
%
%Equations (\ref{eqn:FOC1}), (\ref{eqn:FOC2}), and our feasibility constraint $(k' = (1-\delta)k +F(k,N) - c)$ form our difference equations for $(k',c,n)$ (3 eqns in 3 unknowns).
%
%Now we must solve the firm side. 
%
%The firm maximizes profits:
%\begin{align*}
%\max_{k^d,n^d} \sum_{t=1}^{\infty} 
%\end{align*} 

We will write down the equations to characterize the system, normalize by $A$, and then solve for our difference equations. 

Households maximize utility subject to their budget constraints:

\begin{align*}
&\max_{\{C_t,I_t,N_t,K_t\}_{t=0}^{\infty}} \sum_{i=0}^{\infty}\beta^t u(C_t,1-N_t)\\
&\text{s.t. } \sum_{t=0}^{\infty} p_t (C_t + I_t) =  \sum_{t=0}^{\infty} p_t(r_tK_t + w_tA_tN_t) + \Pi_0\\
&\text{and } K_{t+1} = (1-\delta) K_t+ I_t
\end{align*}

Firms maximize profits:
\begin{align*}
\max \Pi_0 &=\sum_{t=0}^{\infty}p_t(Y_t -r_tK_t^d - w_t A_tN_t^d)\\
\text{s.t. } Y_t &= F(K_t,A_tN_t^d)
\end{align*}

Now we normalize. Let lowercase $x_t = X_t/A_t$:

HH:
\begin{align*}
&\max_{\{c_t,i_t,n_t,k_t\}_{t=0}^{\infty}} \sum_{i=0}^{\infty}\beta^t u(c_tA_t,1-N_t)\\
&\text{s.t. } \sum_{t=0}^{\infty} p_t (c_t + i_t) =  \sum_{t=0}^{\infty} p_t(r_tk_t + w_tN_t) + \Pi_0\\
&\text{and } k_{t+1}(1+g) = (1-\delta) k_t+ i_t
\end{align*}

Firms:
\begin{align*}
\max \Pi_0 &=\sum_{t=0}^{\infty}p_t(y_t -r_tk_t^d - w_t N_t^d)\\
\text{s.t. } y_t &= F(k_t^d,N_t^d)
\end{align*}

Now we begin to solve. We will start with the firm side:

\begin{align*}
F_k(k_t^d,N_t^d) &= r_t\\
F_N(k_t^d,N_t^d) &= w_t\\
\Rightarrow F(k_t^d,N_t^d) - F_k(k_t^d,N_t^d) r_t - F_N(k_t^d,N_t^d) w_t &= 0 \\\Rightarrow \Pi_0 &= 0.
\end{align*}

Now we move to the HH side. Our HH problem can be reduced to the following:
\begin{align*}
&\max_{c_t,n_t} \sum_{i=1}^{\infty}\beta^t u(A_tc_t,1-N_t)\\
&\text{s.t. } \sum_{t=0}^{\infty} p_t ( c_t + k_{t+1} - [r_t + (1-\delta)]k_t - w_tn_t ) = 0
\end{align*}
Taking first order conditions of our lagrangian:

\begin{align*}
A_t\beta^tu_c(c_tA_t,1-N_t) &= \lambda p_t\\
A_{t+1}\beta^{t+1}u_c(c_{t+1}A_{t+1},1-N_{t+1}) &= \lambda p_{t+1}\\
\Rightarrow (1+g)\frac{\beta u_c(c_{t+1}A_{t+1},1-N_{t+1})}{ u_c(c_tA_t,1-N_t)} &= \frac{p_{t+1}}{p_t}\\
-\beta^tu_n(c_tA_t,1-N_t) &= \lambda p_tw_t \\
-\beta^{t+1}u_n(c_{t+1}A_{t+1},1-N_{t+1}) &= \lambda p_{t+1}w_{t+1}\\
\Rightarrow \frac{\beta u_n(c_{t+1}A_{t+1},1-N_{t+1}) }{u_n(c_tA_t,1-N_t)} &= \frac{p_{t+1} w_{t+1}}{p_t w_t}
\end{align*}

In the competitive equilibrium, markets clear:
\begin{align*}
k_t^d &= k_t\\
N_t^d &= N_t\\
c_t + k_{t+1}(1+g) - (1-\delta)K_{t} &= F(k_t^d,N_t^d)
\end{align*}

The equations that give us $C_0,N_0,w_0,r_0$ are the following four equations:

\begin{align}
F_k(k_t,N_t) &= r_t\\
F_N(k_t,N_t) &= w_t\\
A_t u_c (c_t A_t, 1-N_t) &= - u_n (c_tA_t,1-N_t) / w_t
\end{align}
\end{document}
