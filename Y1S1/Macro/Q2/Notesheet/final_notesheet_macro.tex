% !TEX TS-program = pdflatex
% !TEX encoding = UTF-8 Unicode

% This is a simple template for a LaTeX document using the "article" class.
% See "book", "report", "letter" for other types of document.

\documentclass[11pt]{article} % use larger type; default would be 10pt

\usepackage[utf8]{inputenc} % set input encoding (not needed with XeLaTeX)

%%% PAGE DIMENSIONS
\usepackage{geometry} % to change the page dimensions
\geometry{a4paper} % or letterpaper (US) or a5paper or....

\usepackage{graphicx} % support the \includegraphics command and options

\usepackage{amssymb}
\usepackage{amsmath}
%%% PACKAGES
\usepackage{booktabs} % for much better looking tables
\usepackage{array} % for better arrays (eg matrices) in maths
\usepackage{paralist} % very flexible & customisable lists (eg. enumerate/itemize, etc.)
\usepackage{verbatim} % adds environment for commenting out blocks of text & for better verbatim
\usepackage{subfig} % make it possible to include more than one captioned figure/table in a single float
% These packages are all incorporated in the memoir class to one degree or another...

%%% HEADERS & FOOTERS
\usepackage{fancyhdr} % This should be set AFTER setting up the page geometry
\pagestyle{fancy} % options: empty , plain , fancy
\renewcommand{\headrulewidth}{0pt} % customise the layout...
\lhead{}\chead{}\rhead{}
\lfoot{}\cfoot{\thepage}\rfoot{}

%%% SECTION TITLE APPEARANCE
\usepackage{sectsty}
\allsectionsfont{\sffamily\mdseries\upshape} % (See the fntguide.pdf for font help)
% (This matches ConTeXt defaults)

%%% ToC (table of contents) APPEARANCE
\usepackage[nottoc,notlof,notlot]{tocbibind} % Put the bibliography in the ToC
\usepackage[titles,subfigure]{tocloft} % Alter the style of the Table of Contents
\renewcommand{\cftsecfont}{\rmfamily\mdseries\upshape}
\renewcommand{\cftsecpagefont}{\rmfamily\mdseries\upshape} % No bold!
\usepackage{graphicx}
\graphicspath{ {./pings/} }

\usepackage{amsmath}
\DeclareMathOperator*{\argmax}{arg\,max}
\DeclareMathOperator*{\argmin}{arg\,min}

\newcount\colveccount
\newcommand*\colvec[1]{
        \global\colveccount#1
        \begin{pmatrix}
        \colvecnext
}
\def\colvecnext#1{
        #1
        \global\advance\colveccount-1
        \ifnum\colveccount>0
                \\
                \expandafter\colvecnext
        \else
                \end{pmatrix}
        \fi
}

%%% END Article customizations

%%% The "real" document content comes below...

\title{Macro Notesheet}
\author{Michael B. Nattinger}

%\date{} % Activate to display a given date or no date (if empty),
         % otherwise the current date is printed 

\begin{document}
\maketitle
\section{Blackwell, Contractions, Fixed Points, etc.}
\begin{align*}
v(x) &= \max_{y\in\Gamma(A)} u(x,y) + \beta v(y); \\
T(z)(x) &= \max_{y\in\Gamma(A)} u(x,y) + \beta T(z)(y) \text{ (contraction-form)}
\end{align*}
For existence, uniqueness, strict monotonicity, and concavity of $v^{*}$ (as fixed point of contraction form of Bellman), we need $X$ (set of possible x choices) convex; $\Gamma(X)$ non-empty, compact-valued, continuous, and convex; $u(\cdot)$ continuous, bounded, strictly increasing and strictly concave; $\beta<1$. Concavity of V is given by the theorem of the maximum. We may be expected to explicitly show Blackwell holds, i.e. prove discounting and monotonicity.
 For differentiablility of $V$, we need that for $x_0 \in INT(X), D(x_0)$ neighborhood, $\exists W(x):D(x_0) \rightarrow \mathbb{R}$, concave, diff'l function s.t. $W(x_0) = V(x_0)$ and $W(x) \leq V(x) \forall x \in D$ (then V is differentiable by B-S).

$T$ maps continuous bounded functions into continuous bounded functions. $T: C(\mathbb{C})\rightarrow C(\mathbb{C})$. By the above assumptions, $T$ is a contraction of modulus $\beta$ by Blackwell sufficient conditions. $C(X)$ is a complete metric space, so by the contraction mapping theorem, there exists a unique $v^{*}$ (fixed point); and by CMT corollary $v^{*}$ is strictly increasing and strictly concave.
 
If we are not given that $u(\cdot)$ is bounded (as we typically are not), one possibility (unlikely on the exam) is if the utility is unbounded below (for example, see pset 1 question 2). The method is to find the upper bound of V, apply $T$ n times (probably have to guess and verify the form this takes), then take the limit as $n\rightarrow \infty.$ This will yield the fixed point solution.

Alternatively (and more likely to be on the exam, although still unlikely), if the utility function is homogenous of degree $0<1-\gamma<1$, if the constraint set is a convex cone, then V is homogenous of degree $1-\gamma$. That is, under a specific norm and set of functions on $X$, $H(X)$, $T:H\rightarrow H$ is a contraction mapping, $v$ is continuous via theorem of the maximum. 

If you need to explicity use Blackwell to show a contraction, or something else specific, follow closely the steps in handout 9 or 10.
\end{document}
