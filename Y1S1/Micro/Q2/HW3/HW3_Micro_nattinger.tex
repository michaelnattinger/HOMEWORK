% !TEX TS-program = pdflatex
% !TEX encoding = UTF-8 Unicode

% This is a simple template for a LaTeX document using the "article" class.
% See "book", "report", "letter" for other types of document.

\documentclass[11pt]{article} % use larger type; default would be 10pt

\usepackage[utf8]{inputenc} % set input encoding (not needed with XeLaTeX)

%%% PAGE DIMENSIONS
\usepackage{geometry} % to change the page dimensions
\geometry{a4paper} % or letterpaper (US) or a5paper or....

\usepackage{graphicx} % support the \includegraphics command and options

\usepackage{amssymb}
\usepackage{amsmath}
%%% PACKAGES
\usepackage{booktabs} % for much better looking tables
\usepackage{array} % for better arrays (eg matrices) in maths
\usepackage{paralist} % very flexible & customisable lists (eg. enumerate/itemize, etc.)
\usepackage{verbatim} % adds environment for commenting out blocks of text & for better verbatim
\usepackage{subfig} % make it possible to include more than one captioned figure/table in a single float
% These packages are all incorporated in the memoir class to one degree or another...

%%% HEADERS & FOOTERS
\usepackage{fancyhdr} % This should be set AFTER setting up the page geometry
\pagestyle{fancy} % options: empty , plain , fancy
\renewcommand{\headrulewidth}{0pt} % customise the layout...
\lhead{}\chead{}\rhead{}
\lfoot{}\cfoot{\thepage}\rfoot{}

%%% SECTION TITLE APPEARANCE
\usepackage{sectsty}
\allsectionsfont{\sffamily\mdseries\upshape} % (See the fntguide.pdf for font help)
% (This matches ConTeXt defaults)

%%% ToC (table of contents) APPEARANCE
\usepackage[nottoc,notlof,notlot]{tocbibind} % Put the bibliography in the ToC
\usepackage[titles,subfigure]{tocloft} % Alter the style of the Table of Contents
\renewcommand{\cftsecfont}{\rmfamily\mdseries\upshape}
\renewcommand{\cftsecpagefont}{\rmfamily\mdseries\upshape} % No bold!

\usepackage{amsmath}
\usepackage{graphicx}
\graphicspath{ {./pings/} }
\DeclareMathOperator*{\argmax}{arg\,max}
\DeclareMathOperator*{\argmin}{arg\,min}

\newcount\colveccount
\newcommand*\colvec[1]{
        \global\colveccount#1
        \begin{pmatrix}
        \colvecnext
}
\def\colvecnext#1{
        #1
        \global\advance\colveccount-1
        \ifnum\colveccount>0
                \\
                \expandafter\colvecnext
        \else
                \end{pmatrix}
        \fi
}

%%% END Article customizations

%%% The "real" document content comes below...

\title{Micro HW3}
\author{Michael B. Nattinger\footnote{I worked on this assignment with my study group: Alex von Hafften, Andrew Smith, Ryan Mather, and Tyler Welch. I have also discussed problem(s) with Emily Case, Sarah Bass, and Danny Edgel.}}

%\date{} % Activate to display a given date or no date (if empty),
         % otherwise the current date is printed 

\begin{document}
\maketitle

\section{Question 1}
\subsection{Part A}
\begin{align*}
u_i(w_i,w_j) &= \gamma w_i - \beta(w_i - w_j)^2 - \rho w_i - \alpha w_i^2/2 - \alpha w_jw_i/2\\
\Rightarrow \frac{\partial u}{\partial w_i}(0,0) &= \gamma-\rho>0 
\end{align*}
The assumption guarantees that the marginal utility from $w_i>0$ at $0,0$, so the gangs will set $w_i>0$.
\subsection{Part B}
The game is supermodular if the cross partials are (weakly) positive:

\begin{align*}
\frac{\partial^2 u}{\partial w_j \partial w_i} &= 2\beta - \alpha/2 \geq0 \\
\Rightarrow \beta&\geq\alpha/4.
\end{align*}
\subsection{Part C}
From our parametric assumption we know $w_i>0$ so we can take FOCs to optimize:

\begin{align*}
0 &= \gamma - 2\beta (w_i - w_j) - \rho -\alpha w_i - \alpha w_j/2\\
\Rightarrow w_i &= \frac{(\gamma-\rho) + (2\beta -\alpha/2)w_j }{2\beta +\alpha}\\
\Rightarrow w_i &= \frac{(\gamma-\rho) + (2\beta -\alpha/2)\frac{(\gamma-\rho) + (2\beta -\alpha/2)w_i }{2\beta +\alpha} }{2\beta +\alpha}\\
\Rightarrow (2\beta + \alpha)^2 w_i &= (2\beta+\alpha + 2\beta - \alpha/2)(\gamma - \rho) + (2\beta - \alpha/2)^2 w_i\\
\Rightarrow (6\alpha\beta + (3/4)\alpha^2)w_i &= (4\beta + \alpha/2)(\gamma - \rho)\\
\Rightarrow  w_i &= \left( \frac{4\beta+ \alpha/2}{6\alpha \beta + (3/4)\alpha^2} \right)(\gamma - \rho)\\
\Rightarrow w_i &= \frac{2(\gamma-\rho)}{3\alpha}
\end{align*}
\subsection{Part D}
\begin{itemize}
\item
$\rho$ is the base value of the price of weapons, i.e. the y intercept of the weapon supply equation. Higher values of $\rho$ result in directly higher prices of weapons, resulting in lower equilibrium quantities of weapons demanded.
\item
$\gamma$ is the direct positive impact (for the gang) of buying a weapon. Perhaps an interpretation of it would be expected revenue per weapon owned. Higher values of $\gamma$ increases the equilibrium quantity of weapons demanded.
\item
$\beta$ is the disutility from having a different number of weapons than the other gang. In equilibrium this has no effect as, due to symmetry, the gangs will own the same number of guns.
\item
$\alpha$ is the slope of the weapon supply curve. Higher values of $\alpha$ result in price responding more to an increase in weapon demand. Therefore, an increase in $\alpha$ reduces the amount of weapons demanded.
\end{itemize}
\subsection{Part E}
With $w_j$ as given, the best response of firm $i$ satisfies the following condition:
\begin{align*}
0 &= \gamma - 2\beta (w_i - w_j) - \rho -\alpha w_i - \alpha w_j/2\\
\end{align*}
We will try to find a value of $w_j$ which will result in the optimal $w_i$ being $0$:
\begin{align*}
0 &= \gamma +2\beta  w_j - \rho - \alpha w_j/2\\
\Rightarrow w_j &= -\frac{\gamma - \rho}{2\beta - \alpha/2}.
\end{align*}
We know that $\gamma>\rho$ so this quantity is positive iff $2\beta<\alpha/2 \Rightarrow \beta<\alpha/4$. Note that this is the opposite condition from what we found to satisfy supermodularity. %If the game is not supermodular then there exists equilibria where $w_i=0.$
Now, the question is whether the best response to the other playing 0 is this strategy.
\begin{align*}
w_i &= \frac{(\gamma - \rho) + (2\beta - \alpha /2)*0}{2\beta + \alpha}\\
&= \frac{\gamma - \rho}{2\beta + \alpha}.
\end{align*}
This is a nash equilibrium iff $ -\frac{\gamma - \rho}{2\beta - \alpha/2}= \frac{\gamma - \rho}{2\beta + \alpha} \Rightarrow \alpha/2 - 2\beta = 2\beta + \alpha \Rightarrow \alpha = -8\beta$. $\alpha,\beta>0$ so this is not possible. Thus, it is not possible for a nash equilibrium to have $0$ as a strategy.
\subsection{Part F}
Let player $j$ play a mixed strategy. Then, player 1 maximizes their expected utility.
\begin{align*}
w_i &= \argmax_{w_i} E[u_i(w_i,w_j)] \\
&=  \argmax_{w_i} E[\gamma w_i - \beta(w_i - w_j)^2 - \rho w_i - \frac{\alpha}{2}w_i^2 -\frac{\alpha}{2}w_iw_j ]
&= \argmax_{w_i} E[(\gamma - \rho) w_i - \beta(w_i - w_j)^2  - \frac{\alpha}{2}w_i^2 -\frac{\alpha}{2}w_iw_j ]
\end{align*}
We take first order conditions:

\begin{align*}
0&= E[(\gamma - \rho) - 2\beta w_i + 2\beta w_j - \alpha w_i - \alpha w_j]
&= (\gamma - \rho) - 2\beta w_i +2\beta E[w_j] - \alpha w_i - \alpha E[w_j]
&= f(E[w_j])
\end{align*}

Therefore, the best response to a strategy depends only on the expectation of mixed strategy, and thus the best response to any strategy is a pure strategy. Therefore, it is not possible to have a nash equilibrium with mixed strategies as any best response is necessarily pure.
\subsection{Part G}
%Gang 1's weapon demand will increase as the weapons are more valueable now. This increase will be larger in magnitude than it would be if both gangs were able to respond because, if gang 2 could also buy weapons then they would buy more weapons which would increase the price of weapons. When the second gang cannot buy weapons, the weapons are relatively cheaper for gang 1 so gang 1 will buy more weapons.
When $w_j$ can change then we have shown that $w_i = \frac{2(\gamma - \rho)}{3\alpha} \Rightarrow w_i^{',1} = \frac{2}{3\alpha}$ and when $w_j$ cannot change then $w_i = \frac{(\gamma - \rho + (2\beta - \alpha/2) \bar{w}_j)}{2\beta + \alpha} \Rightarrow w_i^{',2} = \frac{1}{2\beta + \alpha}$. Thus, gang 1 will buy more guns when gang 2 cannot buy more guns relative to if gang 2 can buy more guns if $w_{i}^{',1}<w_{i}^{',2} \Rightarrow  \frac{2}{3\alpha}< \frac{1}{2\beta + \alpha} \Rightarrow \beta < \frac{\alpha}{4}$
\section{Question 2}
Each gang takes the average as given and sets $w_i$ to satisfy their FOC:
\begin{align*}
0&=\gamma - 2\beta (w_i - \bar{w}) - \rho - \alpha \bar{w}\\
\Rightarrow 2\beta w_i &= ( 2\beta - \alpha)\bar{w} + (\gamma - \rho)\\
\end{align*}
In the symmetric equilibrium, $\bar{w} = w_i \forall i$, so:
\begin{align*}
 w_i = \frac{(\gamma - \rho)}{\alpha}
\end{align*}

We get a different equilibrium in this case because no individual gang has any ability to affect the average quantity. The $\gamma - \rho$ term is the same, and the part that changes is that $\alpha$ is less important when choosing prices because additional purchasing of weapons does not increase the average quantity of guns purchased, and thus does not affect the price of weapons through the $\alpha$ term in the same way that it does in part C.

\section{Question 3} 
\subsection{Part A}
%Each agent takes the actions of the other agents as given and reacts optimally. We will take first order conditions:
%
%\begin{align*}
% 2(x_i - \alpha) &= 2(x_i - \bar{x})
%\end{align*}
%In the symmetric nash equilibrium, $x_i = \bar{x} \Rightarrow x_i = \alpha$.
There exist 3 pure strategy Nash equilibria: $x_i=0,x_i=1,x_i = \alpha.$ If all other agents choose $x_i=0$, then choosing any number in $(0,\alpha)$ will result in strictly lower utility as the $(x_i - \alpha)^2$ term will decrease and the $-(x_i - \bar{x})^2$ term will decrease. Moreover, choosing any number in $[\alpha,1]$ will also result in strictly lower utility as the $(x_i - \alpha)^2$ term will increase by less than the $-(x_i - \bar{x})^2$ term will decrease. Therefore, $x_i = 0$ is a pure strategy nash equilibrium. Parallel logic shows that $x_i=1$ is also a pure strategy nash equilibrium.

If everyone chooses $\alpha$ then any individual would be indifferent from following the nash equilibrium and deviating. Therefore a pure strategy nash equilibrium exists

\subsection{Part B}
Any distribution such that the average is at $\alpha$ is a nash equilibria as the disutility of being far away from the mean is equal to the utility gained by being close to the average, so the agents would be indifferent between staying with the equilibria or moving anywhere else.

We can also show that no value of $\bar{x} \in (0,1) \setminus \{ \alpha\}$ would result in a nash equilibrium. For purpose of contradiction, say there exists a distributional nash equilibrium with $\bar{x} \in (0,\alpha).$ Then, there exists either some $x_i \geq \bar{x}$. If $x_i>\bar{x}$ then individual $i$ would receive more utility by choosing $\bar{x}.$ Similarly, if $x_i = \bar{x}$ then individual $i$ would receive higher utility by setting $x_i = 0$. Thus, in either case individual $i$ is better off by deviating, which is a contradiction. By parallel logic, there cannot be a nash equilibrium with  $\bar{x} \in (\alpha,1).$ Therefore, the nash equilibria are any distributions with $\bar{x} = \alpha,$ and the symmetric nash equilibria at $\bar{x} = 1,$ $\bar{x} = 0.$
%The quantile function will equate the utiltiy of each quantile with the utility the agent at that quantile would get by picking the optimal choice:
%\begin{align*}
%u_i(q(x),\bar{x},\alpha) &= u_i(\alpha,\bar{x},\alpha)\\
%\Rightarrow (q(x) - \alpha)^2 - (q(x) - \bar{x})^2 &= - (\alpha - \bar{x})^2\\
%\Rightarrow q^2(x) - 2\alpha + \alpha^2 - q^2(x) + 2q(x)\bar{x} - \bar{x}^2 &= - \alpha^2 + 2\alpha\bar{x} - \bar{x}^2\\
%\Rightarrow  ( \bar{x} - \alpha) q(x) &= \alpha \bar{x} - \alpha^2
%\Rightarrow q(x) &= \alpha %?
%\end{align*}
\subsection{Part C}
If agents can choose any $x\in \mathbb{R}$ then nash equilibria would take the form of any distribution such that the average was at $\alpha$. If that is the case, then the disutility of moving away from the mean value would exactly offset the utility of moving away from $\alpha$ for any individual agent, so the agent would be indifferent between staying with the equilibria or moving. If instead the mean is below $\alpha$, then any agent could receive unboundedly more utility by moving to the left of the number line than they lose by moving away from the mean, and similarly if the mean is above $\alpha$ then they can receive unboundedly more utility by moving to the right of the number line than they lose utility from moving away from the mean.
\section{Question 4}
First we will check for symmetric nash equilibria. We start by taking first order conditions:
%\begin{align*}
%1+(q_j - 1)^{1/3} &= q_i\\
%1+(q_i - 1)^{1/3} &= q_i\\
%\Rightarrow q_i &= 1 = q_{-i}, q_i = 2 = q_{-i}.
%\end{align*}
%Now we will check for potentially non-symmetric pure nash equilibria:
\begin{align*}
1+(q_j - 1)^{1/3} &= q_i\\
1+(1+(q_i - 1)^{1/3} - 1)^{1/3} &= q_i\\
\Rightarrow q_i = q_j = 0, q_i = q_j = 1, q_i = q_j &= 2.
\end{align*}
Therefore, $(0,0),(1,1),(2,2)$ are all pure strategy nash equilibria.

As for mixed strategy nash equilibria, since the payoff function is strictly quasi-concave, ergo the game does not have mixed nash equilibria.

\section{Question 5}

There are no pure strategy nash equilibria. We can show this easily. Assume a pure strategy nash equilibria exists. Then, it is not the case that both players are receiving 10 utility. Therefore, at least one of the players would be better off by changing strategies to whatever would yield them 10 utility. This is a contradiction, so no pure strategy nash equilibria exist.

Assume a nash equilibrium exists where both player 1 and player 2 are mixing. Then, as the game is zero loss, with an extra cost of mixing for both players, at least one of the players must have negative utility. Then, that player would be better off by choosing a pure strategy where they purely play the strategy which wins against one of the strategies which recieves the largest weight from the other player's mixture. Therefore, a mixed strategy is not a best response to a mixed strategy.

Moreover, assume player 1 is playing a pure strategy. Then, if player 2 were to play a pure strategy that wins against player 1, they would receive 10 expected utility. If they instead were to mix, they would receive less than 9 expected utility. Therefore, the best response to a pure strategy is a pure strategy.

The above arguments show that a pure strategy is a best response to a mixed strategy, and that a pure strategy is a best response to a mixed strategy. Therefore, for any nash equilibria that were to exist, the equilibria could only contain pure strategies. However, we have also shown that no pure strategy nash equilibria can exist. Therefore, no nash equilibria can exist.
\end{document}
