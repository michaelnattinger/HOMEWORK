% !TEX TS-program = pdflatex
% !TEX encoding = UTF-8 Unicode

% This is a simple template for a LaTeX document using the "article" class.
% See "book", "report", "letter" for other types of document.

\documentclass[11pt]{article} % use larger type; default would be 10pt

\usepackage[utf8]{inputenc} % set input encoding (not needed with XeLaTeX)

%%% PAGE DIMENSIONS
\usepackage{geometry} % to change the page dimensions
\geometry{a4paper} % or letterpaper (US) or a5paper or....

\usepackage{graphicx} % support the \includegraphics command and options

\usepackage{amssymb}
\usepackage{amsmath}
%%% PACKAGES
\usepackage{booktabs} % for much better looking tables
\usepackage{array} % for better arrays (eg matrices) in maths
\usepackage{paralist} % very flexible & customisable lists (eg. enumerate/itemize, etc.)
\usepackage{verbatim} % adds environment for commenting out blocks of text & for better verbatim
\usepackage{subfig} % make it possible to include more than one captioned figure/table in a single float
% These packages are all incorporated in the memoir class to one degree or another...

%%% HEADERS & FOOTERS
\usepackage{fancyhdr} % This should be set AFTER setting up the page geometry
\pagestyle{fancy} % options: empty , plain , fancy
\renewcommand{\headrulewidth}{0pt} % customise the layout...
\lhead{}\chead{}\rhead{}
\lfoot{}\cfoot{\thepage}\rfoot{}

%%% SECTION TITLE APPEARANCE
\usepackage{sectsty}
\allsectionsfont{\sffamily\mdseries\upshape} % (See the fntguide.pdf for font help)
% (This matches ConTeXt defaults)

%%% ToC (table of contents) APPEARANCE
\usepackage[nottoc,notlof,notlot]{tocbibind} % Put the bibliography in the ToC
\usepackage[titles,subfigure]{tocloft} % Alter the style of the Table of Contents
\renewcommand{\cftsecfont}{\rmfamily\mdseries\upshape}
\renewcommand{\cftsecpagefont}{\rmfamily\mdseries\upshape} % No bold!
\usepackage{tikz,forest}

\usepackage{amsmath}
\usepackage{graphicx}
\graphicspath{ {./pings/} }
\DeclareMathOperator*{\argmax}{arg\,max}
\DeclareMathOperator*{\argmin}{arg\,min}

\newcount\colveccount
\newcommand*\colvec[1]{
        \global\colveccount#1
        \begin{pmatrix}
        \colvecnext
}
\def\colvecnext#1{
        #1
        \global\advance\colveccount-1
        \ifnum\colveccount>0
                \\
                \expandafter\colvecnext
        \else
                \end{pmatrix}
        \fi
}
\usetikzlibrary{arrows.meta}

\forestset{
    .style={
        for tree={
            base=bottom,
            child anchor=north,
            align=center,
            s sep+=1cm,
    straight edge/.style={
        edge path={\noexpand\path[\forestoption{edge},thick,-{Latex}] 
        (!u.parent anchor) -- (.child anchor);}
    },
    if n children={0}
        {tier=word, draw, thick, rectangle}
        {draw, diamond, thick, aspect=2},
    if n=1{%
        edge path={\noexpand\path[\forestoption{edge},thick,-{Latex}] 
        (!u.parent anchor) -| (.child anchor) node[pos=.2, above] {Y};}
        }{
        edge path={\noexpand\path[\forestoption{edge},thick,-{Latex}] 
        (!u.parent anchor) -| (.child anchor) node[pos=.2, above] {N};}
        }
        }
    }
}

%%% END Article customizations

%%% The "real" document content comes below...

\title{Micro HW6}
\author{Michael B. Nattinger\footnote{I worked on this assignment with my study group: Alex von Hafften, Andrew Smith, Ryan Mather, and Tyler Welch. I have also discussed problem(s) with Emily Case, Sarah Bass, and Danny Edgel.}}

%\date{} % Activate to display a given date or no date (if empty),
         % otherwise the current date is printed 

\begin{document}
\maketitle

\section{Question 1}
If both play C, they will receive 2 utility in each period. one diverges to D, they will receive 8 utility in that period and 1 utility in all periods moving forwards. Playing (C,C) in all periods can be supported if:
\begin{align*}
\frac{2}{1-\delta} &\geq 8+\frac{\delta}{1-\delta}\\
\Rightarrow 2 &\geq 8-8\delta + \delta\\
\Rightarrow \delta &\geq \frac{6}{7}.
\end{align*}

\section{Question 2}
\subsection{Part A}
The strategy profile will be a subgame perfect equilibrium iff no player has a profitable deviation from the strategy after any history. 

There will not be a deviation from $(C,C),\dots$ to $(C,D),(P,P),(C,C),\dots$ if:

\begin{align*}
\frac{2}{1-\delta} &\geq 3+ \frac{2\delta^2 }{1-\delta}\\
\Rightarrow \delta&\geq \frac{1}{2}
\end{align*}

Assume there is a deviation. There will not be a deviation from $(P,P),(C,C),\dots$ to $(P,D),(P,P),(C,C),\dots$ if:

\begin{align*}
\frac{\delta 2}{1-\delta} &\geq 1+\frac{\delta^2 2}{1-\delta}\\
\delta 2 &\geq 1-\delta + 2\\
\Rightarrow  \delta&\geq \frac{1}{2}
\end{align*}

Both will hold for $\delta \in [\frac{1}{2},1)$.
\subsection{Part B}
There will not be a deviation from $(C,C),\dots$ to $(C,D),(P,P),(C,C),\dots$ if:

\begin{align*}
\frac{2}{1-\delta} &\geq 3 + \frac{1}{2} + \frac{2\delta^2 }{1-\delta}\\
\Rightarrow \delta&\geq \frac{2}{3}
\end{align*}

Assume there is a deviation. There will not be a deviation from $(P,P),(C,C),\dots$ to $(P,D),(P,P),(C,C),\dots$ if:

\begin{align*}
\frac{1}{2} + \frac{\delta 2}{1-\delta} &\geq 1+\frac{\delta}{2}+ \frac{\delta^2 2}{1-\delta}\\
\delta 2 &\geq 1-\delta + 2\\
\Rightarrow  \delta&\geq \frac{1}{3}
\end{align*}

Both will hold for $\delta \in [\frac{2}{3},1)$.

\subsection{Part C}
Increasing the (P,P) payoff increases the relative payoff from deviating in the initial type of deviation, but decreases the relative payoff of deviating in the second type of deviation. This increases the level of $\delta$ required in the first type of deviation for the SPE to hold, but decreased the level of $\delta$ required in the second type of deviation for the SPE to hold.

\subsection{Part D}

D strictly dominates C and strictly dominates P, so there are no correlated equilibria possible.

\section{Question 3}
\subsection{Part A}
The profile I will construct is to play (I), and play (D,D,D) forever after any deviation.

By symmetry, the SPE will hold if all 3 people are willing to in the first period:

\begin{align*}
2  + \delta^2(-1) +\delta^3(2  + \delta^2(-2) ) + \dots &\geq 0\\
\Rightarrow \frac{2-\delta^2}{1-\delta^3} &\geq 0\\
\Rightarrow \delta &\geq 0
\end{align*}

\begin{align*}
 \delta(-1) + 2\delta^2 +\delta^3(\delta(-1) + 2\delta^2 ) + \dots &\geq 0\\
\Rightarrow \frac{\delta(-1) +2 \delta^2}{1-\delta^3} &\geq 0\\
\Rightarrow \delta &\geq  \frac{1}{2}
\end{align*}

\begin{align*}
-1 + 2\delta   +\delta^3(-1 + 2\delta  ) + \dots &\geq 0\\
\Rightarrow \frac{-1 + 2\delta }{1-\delta^3} &\geq 0\\
\Rightarrow \delta &\geq \frac{1}{2}
\end{align*}

\subsection{Part B}
A smaller set will allow the SPE to hold, which we will show below.

By symmetry, the SPE will hold if all 3 people are willing to in the first period:

\begin{align*}
2  + \delta(-1) +\delta^3(2  + \delta(-1) ) + \dots &\geq 0\\
\Rightarrow \frac{2  + \delta(-1)}{1-\delta^3} &\geq 0\\
\Rightarrow \delta &\geq 0
\end{align*}

\begin{align*}
 2\delta -\delta^2 +\delta^3(2\delta -\delta^2  ) + \dots &\geq 0\\
\Rightarrow \frac{2\delta -\delta^2 }{1-\delta^3} &\geq 0\\
\Rightarrow \delta &\geq 0
\end{align*}

\begin{align*}
-1 + 2\delta^2   +\delta^3(-1 + 2\delta^2  ) + \dots &\geq 0\\
\Rightarrow \frac{-1 + 2\delta^2 }{1-\delta^3} &\geq 0\\
\Rightarrow \delta &\geq \frac{1}{\sqrt{2}}
\end{align*}

This is a different condition than Part A. Intuitively, the third person gets hit with the C payoff immediately and have to wait two periods to get the A payoff where in Part B they only had to wait one period to get the A payoff. They will only agree to this payoff if they are relatively patient.
\section{Question 4}
\subsection{Part A}
\begin{align*}
\beta_1 &= \left((\beta_1(A),\beta_1(B)),(\beta_1(E),\beta_1(F),\beta_1(G))\right)\\
&= ((1/2,1/2),(0,1/2,1/2))\\
\beta_2 &= (\beta_2(C),\beta_2(D))\\
&= (2/3,1/3)
\end{align*}
The above behavior strategy is equivalent to the mixed strategy given.
\subsection{Part B}
\begin{align*}
\sigma_1 &= \left(\sigma_1(AE),\sigma_1(AF),\sigma_1(AG),\sigma_1(BE),\sigma_1(BF),\sigma_1(BG) \right) \\
&= (a/3,b/3,(1-a-b)/3,c/3,c/6,c/6)
\end{align*}
For $a,b,c \in [0,1]$, the above mixed strategies are equivalent to the behavior strategy.
\section{Question 5}
Person 2 will play a if they think person 3 has a $<2/3$ chance of playing R. 
\section{Question 6}
\section{Question 7}
\section{Question 8}
\end{document}
