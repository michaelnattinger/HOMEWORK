% !TEX TS-program = pdflatex
% !TEX encoding = UTF-8 Unicode

% This is a simple template for a LaTeX document using the "article" class.
% See "book", "report", "letter" for other types of document.

\documentclass[11pt]{article} % use larger type; default would be 10pt

\usepackage[utf8]{inputenc} % set input encoding (not needed with XeLaTeX)

%%% PAGE DIMENSIONS
\usepackage{geometry} % to change the page dimensions
\geometry{a4paper} % or letterpaper (US) or a5paper or....

\usepackage{graphicx} % support the \includegraphics command and options

\usepackage{amssymb}
\usepackage{amsmath}
%%% PACKAGES
\usepackage{booktabs} % for much better looking tables
\usepackage{array} % for better arrays (eg matrices) in maths
\usepackage{paralist} % very flexible & customisable lists (eg. enumerate/itemize, etc.)
\usepackage{verbatim} % adds environment for commenting out blocks of text & for better verbatim
\usepackage{subfig} % make it possible to include more than one captioned figure/table in a single float
% These packages are all incorporated in the memoir class to one degree or another...

%%% HEADERS & FOOTERS
\usepackage{fancyhdr} % This should be set AFTER setting up the page geometry
\pagestyle{fancy} % options: empty , plain , fancy
\renewcommand{\headrulewidth}{0pt} % customise the layout...
\lhead{}\chead{}\rhead{}
\lfoot{}\cfoot{\thepage}\rfoot{}

%%% SECTION TITLE APPEARANCE
\usepackage{sectsty}
\allsectionsfont{\sffamily\mdseries\upshape} % (See the fntguide.pdf for font help)
% (This matches ConTeXt defaults)

%%% ToC (table of contents) APPEARANCE
\usepackage[nottoc,notlof,notlot]{tocbibind} % Put the bibliography in the ToC
\usepackage[titles,subfigure]{tocloft} % Alter the style of the Table of Contents
\renewcommand{\cftsecfont}{\rmfamily\mdseries\upshape}
\renewcommand{\cftsecpagefont}{\rmfamily\mdseries\upshape} % No bold!

\usepackage{amsmath}
\DeclareMathOperator*{\argmax}{arg\,max}
\DeclareMathOperator*{\argmin}{arg\,min}

\newcount\colveccount
\newcommand*\colvec[1]{
        \global\colveccount#1
        \begin{pmatrix}
        \colvecnext
}
\def\colvecnext#1{
        #1
        \global\advance\colveccount-1
        \ifnum\colveccount>0
                \\
                \expandafter\colvecnext
        \else
                \end{pmatrix}
        \fi
}

%%% END Article customizations

%%% The "real" document content comes below...

\title{HW4}
\author{Michael B. Nattinger\footnote{I worked on this assignment with my study group: Alex von Hafften, Andrew Smith, Ryan Mather, and Tyler Welch. I have also discussed problem(s) with Emily Case, Sarah Bass, and Danny Edgel.}}

%\date{} % Activate to display a given date or no date (if empty),
         % otherwise the current date is printed 

\begin{document}
\maketitle

\section{Question 1}
This problem asks a set of questions revolving around the concept of the law of supply. We use this law to investigate the case of a firm which uses goods one and two as inputs and good 2 as input. Formally, $y \in Y$ requires $y_1,y_2 \leq 0$. It is then immediate that $y_3 \geq 0$.

The law of supply invokes the following inequality, which will be the basis of the solutions which follow: $\Delta p \cdot \Delta y \geq 0$.

\subsection{If $p_3$ falls and $p_1,p_2$ stay the same, can $y_3$ go up?}
Let us construct our vectors $\Delta p , \Delta y$. $\Delta p = \colvec{3}{\Delta p_1}{\Delta p_2}{\Delta p_3} = \colvec{3}{0}{0}{\Delta p_3} , \Delta y = \colvec{3}{\Delta y_1}{\Delta y_2}{\Delta y_3}.$ Then, $\Delta p \cdot \Delta y \geq 0 \Rightarrow \Delta p_1 \Delta y_1 + \Delta p_2 \Delta y_2 + \Delta p_3 \Delta y_3 \geq 0 \Rightarrow \Delta p_3 \Delta y_3 \geq 0.$ Since $p_3$ falls, $\Delta p_3 < 0 \Rightarrow \Delta y_3 \leq 0$ so $y_3$ cannot increase.

\subsection{If $p_1$ rises and $p_2,p_3$ stay the same, can $y_3$ go up?}
It can and we will construct an example which demonstrates the possibility. Let the set of possible production vectors be $\{ (-5,-1,30) , (-1,-8,32)\}$. For the price vector $(1,1,1)$ the profit maximizing production vector is $(-5,-1,30)$ while for the price vector $(2,1,1)$ the profit maximizing production vector is $(-1,-8,32)$. In this case, $p_1$ rises while  $p_2,p_3$ stay the same, yet $y_3$ rises from $30$ to $32$.

\subsection{If $p_1,p_2$ both increase and $p_3$ stays the same, can $y_3$ go up?}
\subsubsection{What if both $p_1,p_2$ rise by $10\%$?}

\section{Question 2}

\section{Question 3}

\end{document}
