% !TEX TS-program = pdflatex
% !TEX encoding = UTF-8 Unicode

% This is a simple template for a LaTeX document using the "article" class.
% See "book", "report", "letter" for other types of document.

\documentclass[11pt]{article} % use larger type; default would be 10pt

\usepackage[utf8]{inputenc} % set input encoding (not needed with XeLaTeX)

%%% PAGE DIMENSIONS
\usepackage{geometry} % to change the page dimensions
\geometry{a4paper} % or letterpaper (US) or a5paper or....

\usepackage{graphicx} % support the \includegraphics command and options

\usepackage{amssymb}
\usepackage{amsmath}
%%% PACKAGES
\usepackage{booktabs} % for much better looking tables
\usepackage{array} % for better arrays (eg matrices) in maths
\usepackage{paralist} % very flexible & customisable lists (eg. enumerate/itemize, etc.)
\usepackage{verbatim} % adds environment for commenting out blocks of text & for better verbatim
\usepackage{subfig} % make it possible to include more than one captioned figure/table in a single float
% These packages are all incorporated in the memoir class to one degree or another...

%%% HEADERS & FOOTERS
\usepackage{fancyhdr} % This should be set AFTER setting up the page geometry
\pagestyle{fancy} % options: empty , plain , fancy
\renewcommand{\headrulewidth}{0pt} % customise the layout...
\lhead{}\chead{}\rhead{}
\lfoot{}\cfoot{\thepage}\rfoot{}

%%% SECTION TITLE APPEARANCE
\usepackage{sectsty}
\allsectionsfont{\sffamily\mdseries\upshape} % (See the fntguide.pdf for font help)
% (This matches ConTeXt defaults)

%%% ToC (table of contents) APPEARANCE
\usepackage[nottoc,notlof,notlot]{tocbibind} % Put the bibliography in the ToC
\usepackage[titles,subfigure]{tocloft} % Alter the style of the Table of Contents
\renewcommand{\cftsecfont}{\rmfamily\mdseries\upshape}
\renewcommand{\cftsecpagefont}{\rmfamily\mdseries\upshape} % No bold!

\usepackage{amsmath}
\usepackage{graphicx}
\graphicspath{ {./pings/} }
\DeclareMathOperator*{\argmax}{arg\,max}
\DeclareMathOperator*{\argmin}{arg\,min}

\newcount\colveccount
\newcommand*\colvec[1]{
        \global\colveccount#1
        \begin{pmatrix}
        \colvecnext
}
\def\colvecnext#1{
        #1
        \global\advance\colveccount-1
        \ifnum\colveccount>0
                \\
                \expandafter\colvecnext
        \else
                \end{pmatrix}
        \fi
}

%%% END Article customizations

%%% The "real" document content comes below...

\title{Micro HW3}
\author{Michael B. Nattinger\footnote{I worked on this assignment with my study group: Alex von Hafften, Andrew Smith, Ryan Mather, and Tyler Welch. I have also discussed problem(s) with Emily Case, Sarah Bass, and Danny Edgel.}}

%\date{} % Activate to display a given date or no date (if empty),
         % otherwise the current date is printed 

\begin{document}
\maketitle

\section{Question 1}
\subsection{Prove that the firm's objective function has strictly increasing differences in $q$ and $-\tau$. Prove that this implies a monotone selection rule.}
The firm's objective function is $g(q,t):= (1-\tau)pq - c(q)$ where $t = -\tau$. For $q^{'}>q,t^{'} = (-\tau^{'})>(-\tau) = t$,
\begin{align*}
g(q^{'},t^{'}) - g(q,t^{'}) &=  (1-\tau^{'})pq^{'} - c(q^{'}) - (1-\tau^{'})pq + c(q)\\
&=   (1+(-\tau^{'}))p(q^{'} - q) - (c(q^{'}) -c(q))\\
&> (1+(-\tau))p(q^{'} - q) - (c(q^{'}) -c(q))\\
&= g(q^{'},t) - g(q,t)
\end{align*}
So $g$ has strictly increasing differences in $q,-\tau$. Now, let $q,q^{'}$ be optimal at $t = -\tau,t^{'} = -\tau^{'}$ with $t<t^{'}$. Then, if $q>q^{'}$, by optimality and increasing differences,
\begin{equation*}
0 \geq g(q,t^{'}) - g(q^{'},t^{'})> g(q,t) - g(q^{'},t) \geq 0
\end{equation*}
which is a contradiction. So, our optimal choice at $t^{'}$ must be at least as large as our choice at $t$.

Why is this stronger? This stronger because all possible optimal choices for production at $t^{'}$ must be at least as large as all possible optimal choices for production at $t$.  This is not necessarily guaranteed by baby Topkis if there are more than one possible choices for production at $t^{'},t$.
\subsection{Suppose the firm is not a price-taker in the output market. It faces an inverse demand function $P(q)$, where $P(q)$ is the price the firm can sell $q$ units of output. Show that the firm's objective function no longer has increasing differences in $q,-\tau$.}
The firm's new objective function is $h(q,t) = (1-\tau)P(q)q - c(q)$, where $t=(-\tau)$. $\frac{\partial h}{\partial t} = P(q)q$. If price is decreasing in $q$ faster than $q$ is itself increasing,  then $h$ does not have increasing differences in $q,-\tau$ because $\frac{\partial h}{\partial t}$ is decreasing in $q$.

\subsection{Show that if $c$ is strictly increasing, the firm's objective function still has strictly single-crossing differences; prove that an increase in $\tau$ cannot result in an increase in output.}
Let $h(q^{'},t) - h(q,t)\geq0$ for $t = -\tau <-\tau^{'} = t^{'} $. Then, 
\begin{align*}
0\leq(1+t)P(q^{'})q^{'} - c(q^{'}) - (1+t)P(q)q + c(q) &= (1+t)(P(q^{'})q^{'} - P(q)q) - (c(q^{'}) - c(q))\\
&< (1+t^{'})(P(q^{'})q^{'} - P(q)q) - (c(q^{'}) - c(q))\\
&= h(q^{'},t^{'}) - h(q,t^{'})
\end{align*}
so $h$ has strictly single-crossing differences, and $ h(q^{'},t^{'}) > h(q,t^{'})$. If $q$ is optimal at $t$ and $q^{'}$ is optimal at $t^{'}$, and $t<t^{'}$ and if $q>q^{'}$, then
\begin{align*}
h(q,t^{'}) - h(q^{'},t^{'})  \leq 0 \leq   h(q,t) - h(q^{'},t)
\end{align*}
 is a contradiction as single-crossing differences implies $h(q,t^{'}) - h(q^{'},t^{'})>0$, so $q$ cannot decrease from an increase in $t$, and $q$ cannot increase from a decrease in $\tau$.
\section{Question 2}
The firm maximizes $\pi = p(l^{0.5}m^{0.3} + r^{0.7}e^{0.1})^z - w_ll - w_mm - w_rr - w_ee$ with $z = 1.1$
\subsection{What effect does a reduction in $w_e$ have on the firm's demand for each input?}
The partials of the firm's objective function with regards to each input are as follows:
\begin{align*}
\frac{\partial \pi}{\partial l} &= 1.1p(l^{0.5}m^{0.3} + r^{0.7}e^{0.1})^{0.1}(0.5m^{0.3}l^{-0.5}) - w_l, \\
\frac{\partial \pi}{\partial m} &= 1.1p(l^{0.5}m^{0.3} + r^{0.7}e^{0.1})^{0.1}(0.3l^{0.5}m^{-0.7}) - w_m, \\
\frac{\partial \pi}{\partial r} &= 1.1p(l^{0.5}m^{0.3} + r^{0.7}e^{0.1})^{0.1}(0.7r^{-0.3}e^{0.1}) - w_r, \\
\frac{\partial \pi}{\partial e} &= 1.1p(l^{0.5}m^{0.3} + r^{0.7}e^{0.1})^{0.1}(0.1r^{0.7}e^{-0.9}) - w_e. 
\end{align*}
Each partial is increasing in the other inputs so the objective function is supermodular in $(l,m,r,e)$. Also, each input is weakly increasing in $-w_e$ so a decrease in $w_e$ will lead to an increase in the use of all inputs via Topkis' theorem in the sense of strong set order. Since the production function is concave, the objective function will have a unique maximum so the use of all inputs will increase.\footnote{Note: the use of the inputs will strictly increase here as the use of $e$ will strictly increase from the reduction in $w_e$, and the use of the other inputs will strictly increase as a result of the increase in $e$. Similar logic will hold in the second part of the question, with the signs of some of the inputs flipped as described in the solution to that question.}
\subsection{Over time, $z$ changes to $0.9$. With this new value for $z$, what effect will the wage subsidy have on the firm's demand for each input?}
With $z = 0.9$ the following are the partials to the firm's objective function:
\begin{align*}
\frac{\partial \pi}{\partial l} &= 0.9p(l^{0.5}m^{0.3} + r^{0.7}e^{0.1})^{-0.1}(0.5m^{0.3}l^{-0.5}) - w_l, \\
\frac{\partial \pi}{\partial m} &= 0.9p(l^{0.5}m^{0.3} + r^{0.7}e^{0.1})^{-0.1}(0.3l^{0.5}m^{-0.7}) - w_m, \\
\frac{\partial \pi}{\partial r} &= 0.9p(l^{0.5}m^{0.3} + r^{0.7}e^{0.1})^{-0.1}(0.7r^{-0.3}e^{0.1}) - w_r, \\
\frac{\partial \pi}{\partial e} &= 0.9p(l^{0.5}m^{0.3} + r^{0.7}e^{0.1})^{-0.1}(0.1r^{0.7}e^{-0.9}) - w_e. 
\end{align*}
Let us focus first on $\frac{\partial \pi}{\partial l}.$ This is decreasing in $r,e$. However, it is stil increasing in $m: \frac{\partial \pi}{\partial l} = 0.9p(l^{0.5} + r^{0.7}e^{0.1}m^{-0.3})^{-0.1}(0.5(m^{0.3})^{0.9}l^{-0.5}) - w_l$. We can similarly manipulate the other partials and it is evident that our objective function is supermodular in $(l,m,-r,-e)$ and $(l,m,-r,-e)$ are all weakly increasing in $w_e$ so Topkis' theorem implies that a decrease in $w_e$ will lead to a reduction in $l,m$ and an increase in $e,r$, as just as before our objective function will have a unique maximum.
\subsection{If the supply of managers is fixed in the short-run, would the subsidy's effect on unskilled labor be larger in the short-run or long-run?}
Since we found in the previous section that our objective function is supermodular in $(l,m,-r,-e)$, then we can apply Le Chatelier's principle and find that the effect on unskilled labor is larger in the long-run than the short-run.
\end{document}
