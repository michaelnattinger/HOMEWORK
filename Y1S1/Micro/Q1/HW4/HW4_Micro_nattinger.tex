% !TEX TS-program = pdflatex
% !TEX encoding = UTF-8 Unicode

% This is a simple template for a LaTeX document using the "article" class.
% See "book", "report", "letter" for other types of document.

\documentclass[11pt]{article} % use larger type; default would be 10pt

\usepackage[utf8]{inputenc} % set input encoding (not needed with XeLaTeX)

%%% PAGE DIMENSIONS
\usepackage{geometry} % to change the page dimensions
\geometry{a4paper} % or letterpaper (US) or a5paper or....

\usepackage{graphicx} % support the \includegraphics command and options

\usepackage{amssymb}
\usepackage{amsmath}
%%% PACKAGES
\usepackage{booktabs} % for much better looking tables
\usepackage{array} % for better arrays (eg matrices) in maths
\usepackage{paralist} % very flexible & customisable lists (eg. enumerate/itemize, etc.)
\usepackage{verbatim} % adds environment for commenting out blocks of text & for better verbatim
\usepackage{subfig} % make it possible to include more than one captioned figure/table in a single float
% These packages are all incorporated in the memoir class to one degree or another...

%%% HEADERS & FOOTERS
\usepackage{fancyhdr} % This should be set AFTER setting up the page geometry
\pagestyle{fancy} % options: empty , plain , fancy
\renewcommand{\headrulewidth}{0pt} % customise the layout...
\lhead{}\chead{}\rhead{}
\lfoot{}\cfoot{\thepage}\rfoot{}

%%% SECTION TITLE APPEARANCE
\usepackage{sectsty}
\allsectionsfont{\sffamily\mdseries\upshape} % (See the fntguide.pdf for font help)
% (This matches ConTeXt defaults)

%%% ToC (table of contents) APPEARANCE
\usepackage[nottoc,notlof,notlot]{tocbibind} % Put the bibliography in the ToC
\usepackage[titles,subfigure]{tocloft} % Alter the style of the Table of Contents
\renewcommand{\cftsecfont}{\rmfamily\mdseries\upshape}
\renewcommand{\cftsecpagefont}{\rmfamily\mdseries\upshape} % No bold!

\usepackage{amsmath}
\usepackage{graphicx}
\graphicspath{ {./pings/} }
\DeclareMathOperator*{\argmax}{arg\,max}
\DeclareMathOperator*{\argmin}{arg\,min}

\newcount\colveccount
\newcommand*\colvec[1]{
        \global\colveccount#1
        \begin{pmatrix}
        \colvecnext
}
\def\colvecnext#1{
        #1
        \global\advance\colveccount-1
        \ifnum\colveccount>0
                \\
                \expandafter\colvecnext
        \else
                \end{pmatrix}
        \fi
}

%%% END Article customizations

%%% The "real" document content comes below...

\title{Micro HW4}
\author{Michael B. Nattinger\footnote{I worked on this assignment with my study group: Alex von Hafften, Andrew Smith, Ryan Mather, and Tyler Welch. I have also discussed problem(s) with Emily Case, Sarah Bass, and Danny Edgel.}}

%\date{} % Activate to display a given date or no date (if empty),
         % otherwise the current date is printed 

\begin{document}
\maketitle

\section{Question 1}
Let $A,B \subset X$, and let $x,y\in A\cap B.$ Let $x \in C(A),y \in C(B)$. Then, $x \succeq y,y \succeq x$ because $y\in A,x \in B$. Then, $\forall z\in A$, $y\succeq x \succeq z \Rightarrow y \in C(A)$ by transitivity. Similarly, $\forall z \in B, x\succeq y \succeq z \Rightarrow x \in C(B)$ by transitivity. Therefore, $C(\cdot)$ satisfies WARP.
  
\section{Question 2}
Let $C(\cdot)$ satisfy WARP. 

Let $x,y \in A.$ Then either $x\in C(\{ x,y\})$ or $y\in C(\{ x,y\})$ because $C$ is nonempty so $x\succeq y$ or $y \succeq x$. Thus, $\succeq$ is complete.

Let $x,y,z \in X$ such that $x\succeq y,y\succeq z.$ Then, $\exists A,B\subset X s.t. x,y\in A, x \in C(A);y,z\in B, y \in C(B).$ Assume for the purpose of contradiction that $z\succ x$. Then, $z\in C(A\cup B), y\in C(B) \Rightarrow y \in C(A \cup B)$ by WARP, and similarly by WARP $x \in C(A\cup B).$ Then, $x\sim z$ which is a contradiction.

Let $A\subset X$ and define the choice rule implied by $\succeq$ to be $C^I(A) := \{x\in A : x \succeq y \forall y \in A \}$. Let $x \in C^I(A).$ Then, $\forall y \in A, x \succeq Y \Rightarrow x \in C(A) \Rightarrow C(A)\subseteq C^I(A)$. Now let $x \in C(A).$ Since $x\in C(A)$ then $x\in A$. Then, $A$ is nonempty. For any $y \in A, x,y\in A$ and $x\in C(A)$ imply $ x\succeq y \Rightarrow x \in C^I(A) \Rightarrow C^I(A) \subseteq C(A)\Rightarrow C^I(A) = C(A).$ 
\section{Question 3}
\subsection{Show that the induced choice rule is nonempty.}
We fix finite $X$. Let $A \subseteq X$ have one element, $a$. Then, $C(A) = a$ because $a\succeq a$.

Now assume $A_{n+1}\subseteq X$ has $n+1$ elements, and that $C(A_n)$ is nonempty for any $A_n \subset X$ where $A_n$ contains $n$ elements. Then, for any $a \in A_{n+1}, C(A_{n+1} \setminus \{ a\})$ is nonempty so let $c \in  C(A_{n+1} \setminus \{ a\}).$ Then, either $a \succeq c \Rightarrow a \in C(A_{n+1})$ or $c\succeq a \Rightarrow c \in C(A_{n+1})$ by completeness. In either case, $C(A_{n+1})$ is nonempty.
\subsection{Show that a utility representation exists.}
Assume $X$ has one element, $x$. Then we can set $u(x) = 1$ so immediately $\forall x,y \in X, u(x) = u(y) = 1$ and $x = y$ so $\succeq$ can be represented by a utility representation with range $\{ 1\}$.

Assume that for all $X_n$ with $n$ elements, $\succeq$ can be represented by a utility representation with range $\{ 1 ,\dots, n\}$. Now, let $X_{n+1}$ have $n+1$ elements. Then, let $x\in C(X_{n+1}).$ $X_{n+1}\setminus \{x \}$ can be represented by a utility representation with range  $\{ 1 ,\dots, n\}$. Denote this representation as $u$. For each element $y$ of $X_{n+1}\setminus \{x\}$, $x \succeq y$. Now define $v$ as the following:
\begin{equation*}
v(z) = \begin{cases}u(z), z \in X_{n+1}\setminus \{x\} \\  n+1, z=x \end{cases}
\end{equation*} 
Now, let $a,b\in X_{n+1}.$ If $a,b \in X_{n+1} \setminus \{x\}$, $v(a)=u(a),v(b)=u(b)$ so $a\succeq b$ iff $u(a)\geq u(b).$ Otherwise, one or both of $a,b$ is $x$. WLOG, say $a=x$.  Then $a\in C(X_{n+1})$ so $a\succeq b$, $v(a)\geq v(b)$ so the iff required for preferences to be represented by a utility function is trivially satisfied. Therefore, $v$ is a utility representation with range $\{ 1,\dots,n+1\}$. Therefore, a utility representation exists.
\end{document}
