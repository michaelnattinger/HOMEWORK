% !TEX TS-program = pdflatex
% !TEX encoding = UTF-8 Unicode

% This is a simple template for a LaTeX document using the "article" class.
% See "book", "report", "letter" for other types of document.

\documentclass[11pt]{article} % use larger type; default would be 10pt

\usepackage[utf8]{inputenc} % set input encoding (not needed with XeLaTeX)

%%% PAGE DIMENSIONS
\usepackage{geometry} % to change the page dimensions
\geometry{a4paper} % or letterpaper (US) or a5paper or....

\usepackage{graphicx} % support the \includegraphics command and options

\usepackage{amssymb}
\usepackage{amsmath}
%%% PACKAGES
\usepackage{booktabs} % for much better looking tables
\usepackage{array} % for better arrays (eg matrices) in maths
\usepackage{paralist} % very flexible & customisable lists (eg. enumerate/itemize, etc.)
\usepackage{verbatim} % adds environment for commenting out blocks of text & for better verbatim
\usepackage{subfig} % make it possible to include more than one captioned figure/table in a single float
% These packages are all incorporated in the memoir class to one degree or another...

%%% HEADERS & FOOTERS
\usepackage{fancyhdr} % This should be set AFTER setting up the page geometry
\pagestyle{fancy} % options: empty , plain , fancy
\renewcommand{\headrulewidth}{0pt} % customise the layout...
\lhead{}\chead{}\rhead{}
\lfoot{}\cfoot{\thepage}\rfoot{}

%%% SECTION TITLE APPEARANCE
\usepackage{sectsty}
\allsectionsfont{\sffamily\mdseries\upshape} % (See the fntguide.pdf for font help)
% (This matches ConTeXt defaults)

%%% ToC (table of contents) APPEARANCE
\usepackage[nottoc,notlof,notlot]{tocbibind} % Put the bibliography in the ToC
\usepackage[titles,subfigure]{tocloft} % Alter the style of the Table of Contents
\renewcommand{\cftsecfont}{\rmfamily\mdseries\upshape}
\renewcommand{\cftsecpagefont}{\rmfamily\mdseries\upshape} % No bold!

\usepackage{amsmath}
\usepackage{graphicx}
\graphicspath{ {./pings/} }
\DeclareMathOperator*{\argmax}{arg\,max}
\DeclareMathOperator*{\argmin}{arg\,min}

\newcount\colveccount
\newcommand*\colvec[1]{
        \global\colveccount#1
        \begin{pmatrix}
        \colvecnext
}
\def\colvecnext#1{
        #1
        \global\advance\colveccount-1
        \ifnum\colveccount>0
                \\
                \expandafter\colvecnext
        \else
                \end{pmatrix}
        \fi
}

%%% END Article customizations

%%% The "real" document content comes below...

\title{Micro HW7}
\author{Michael B. Nattinger\footnote{I worked on this assignment with my study group: Alex von Hafften, Andrew Smith, Ryan Mather, and Tyler Welch. I have also discussed problem(s) with Emily Case, Sarah Bass, and Danny Edgel.}}

%\date{} % Activate to display a given date or no date (if empty),
         % otherwise the current date is printed 

\begin{document}
\maketitle

\section{Question 1}
\subsection{Part A}
Let $u$ be linear, so for some $c,d \in \mathbb{R},$ $u(m) = cm + d.$ Then, $U(a) = pu(w+2a ) + (1-p)u(w-a) = p(c(w+2a)+d) + (1-p)(c(w-a)+d) = pcw +2pca + cw - ca + d -pcw +pca -pd \\ =(3p -1)ca  +cw +d  -pd.$ $a = \argmax_{0\leq a\leq w}(3p -1)ca  +cw +d  -pd =  \argmax_{0\leq a\leq w}(3p -1)ca.$ If $3p-1>0 \Rightarrow p>\frac{1}{3}$ then the objective function is maximized when $a$ is maximized, so $a=w$. If $3p-1<0 \Rightarrow p<\frac{1}{3}$ then the objective function is maximized when $a$ is minimized, so $a=0.$
\subsection{Part B}
$\frac{\partial U}{\partial a}(0) = 2pu'(w) -(1-p)u'(w)>0$ because $u'(x)$ is positive as $u$ is strictly increasing, and $2p>1-p$ as $p>\frac{1}{3}$.
\subsection{Part C}
Let $a,b \in (0,w)$ and let $t \in (0,1)$. By the strict concavity of $u$, since $u''<0$, $U(ta+(1-t)b) = pu(w+2(ta+(1-t)b)) +(1-p)u(w-(ta+(1-t)b)) = pu(t(w+2a) + (1-t)(w+2b)) +(1-p)u(t(w-a) +(1-t)(w-b))> p(tu(w+2a) +(1-t)u(w+2b)) + (1-p)(tu(w-a)+(1-t)u(w+2b)) = tU(a) +(1-t)U(b)$ so $U$ is concave.
\subsection{Part D}
If $u'(0)$ is infinite, $\frac{\partial U}{\partial a}(w) =  2pu'(3w) -(1-p)u'(0) = -\infty$ so investing all of your wealth is not optimal. If $u'(0)$ is not infinite then  $\frac{\partial U}{\partial a}(w) =  2pu'(3w) -(1-p)u'(0)\geq 0 \Rightarrow p(2u'(3w)+u'(0)) \geq u'(0) \Rightarrow p\geq\frac{u'(0)}{2u'(3w)+u'(0)}$ so if $p\geq\bar{p} = \frac{u'(0)}{2u'(3w)+u'(0)} = \bar{p}$ then it is optimal to invest all of your wealth. 
\subsection{Part E}
$\argmax_{0\leq a\leq w}p (1-e^{-c(w+2a)}) +(1-p)(1-e^{-c(w-a)}) = \argmax_{0\leq a\leq w}p-pe^{-cw}e^{-2ca} +(1-p)- (1-p)e^{-cw}e^{ca} =  \argmax_{0\leq a\leq w}-pe^{-cw}e^{-2ca}- (1-p)e^{-cw}e^{ca}  = \argmax_{0\leq a\leq w}e^{-cw}(-pe^{-2ca}- (1-p)e^{ca})=  \argmax_{0\leq a\leq w}-cw +log(-pe^{-2ca}- (1-p)e^{ca}) =  \argmax_{0\leq a\leq w}log(-pe^{-2ca}- (1-p)e^{ca})$ which does not depend on wealth.
\subsection{Part F}
Let $A(x)$ be decreasing. Then,
\begin{align*}
\frac{d}{dw}U'(a) &= 2pu''(w+2a) - (1-p)u''(w-a) = -2pu'(w+2a)A(w+2a) + (1-p)u'(w-a)A(w-a).
\end{align*}
At the optimum, $U'(x^{*}(w)) = 0 \Rightarrow 2pu'(w+2a) = (1-p)u'(w-a) $ so,
\begin{align*}
\frac{d}{dw}U'(a)|_{a=a^{*}(w)} &=(1-p)u'(w+2a^{*})(-A(w+2a^{*}) + A(w-a^{*})).
\end{align*}
Since $u'(x)>0$ and $A$ is decreasing $\Rightarrow (-A(w+2a^{*}) + A(w-a^{*}))>0$ so $\frac{d}{dw}U'(a)|_{a=a^{*}(w)}>0$ so our marginal utility from a is strictly increasing in $w$, so $a^{*}$ is strictly increasing in $w$.
\subsection{Part G}
$\argmax_{0\leq t\leq 1}p u(w(1+2t)) +(1-p)u(w(1-t)) =  \argmax_{0\leq t\leq 1}p u(w(1+2t)) +(1-p)u(w(1-t)) =  \argmax_{0\leq t\leq 1}p \frac{1}{1-\rho}(w(1+2t))^{1-\rho} +(1-p)\frac{1}{1-\rho}(w(1-t))^{1-\rho} =  \argmax_{0\leq t\leq 1}w^{1-\rho}(p \frac{1}{1-\rho}(1+2t)^{1-\rho} +(1-p)\frac{1}{1-\rho}(1-t)^{1-\rho} ) =  \argmax_{0\leq t\leq 1}log(w^{1-\rho})+log(p \frac{1}{1-\rho}(1+2t)^{1-\rho} +(1-p)\frac{1}{1-\rho}(1-t)^{1-\rho} ) \\=  \argmax_{0\leq t\leq 1}log(p \frac{1}{1-\rho}(1+2t)^{1-\rho} +(1-p)\frac{1}{1-\rho}(1-t)^{1-\rho} ) $ which does not depend on $w$.
\subsection{Part H}
Let $R(x)$ be increasing. 
\begin{align*}
U'(t) &= pu'(w(1+2t)) + (1-p)u'(w(1-t))\\
&= pu'(w(1+2t))2w - (1-p)wu'(w(1-t)) \\
\frac{\partial U'(t)}{\partial w} &= 2wp(1+2t)u''(w(1+2t)) - (1-p)w(1-t)u''(w(1-t)) \\
&= -2pu'(w(1+2t))R(w(1+2t)) + (1-p)R(w(1-t))u'(w(1-t))\\
\frac{\partial U'(t)}{\partial w}|_{t = t^{*}} &=-2pu'(w(1+2t^{*}))R(w(1+2t)) + 2pR(w(1-t^{*}))u'(w(1+2t^{*}))\\
&= 2pu'(w(1+2t^{*}))(-R(w(1+2t^{*})) +R(w(1-t^{*}))).
\end{align*}
This is negative as $R$ is increasing.
\end{document}
