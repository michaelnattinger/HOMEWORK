% !TEX TS-program = pdflatex
% !TEX encoding = UTF-8 Unicode

% This is a simple template for a LaTeX document using the "article" class.
% See "book", "report", "letter" for other types of document.

\documentclass[11pt]{article} % use larger type; default would be 10pt

\usepackage[utf8]{inputenc} % set input encoding (not needed with XeLaTeX)

%%% PAGE DIMENSIONS
\usepackage{geometry} % to change the page dimensions
\geometry{a4paper} % or letterpaper (US) or a5paper or....

\usepackage{graphicx} % support the \includegraphics command and options

\usepackage{amssymb}
\usepackage{amsmath}
%%% PACKAGES
\usepackage{booktabs} % for much better looking tables
\usepackage{array} % for better arrays (eg matrices) in maths
\usepackage{paralist} % very flexible & customisable lists (eg. enumerate/itemize, etc.)
\usepackage{verbatim} % adds environment for commenting out blocks of text & for better verbatim
\usepackage{subfig} % make it possible to include more than one captioned figure/table in a single float
% These packages are all incorporated in the memoir class to one degree or another...

%%% HEADERS & FOOTERS
\usepackage{fancyhdr} % This should be set AFTER setting up the page geometry
\pagestyle{fancy} % options: empty , plain , fancy
\renewcommand{\headrulewidth}{0pt} % customise the layout...
\lhead{}\chead{}\rhead{}
\lfoot{}\cfoot{\thepage}\rfoot{}

%%% SECTION TITLE APPEARANCE
\usepackage{sectsty}
\allsectionsfont{\sffamily\mdseries\upshape} % (See the fntguide.pdf for font help)
% (This matches ConTeXt defaults)

%%% ToC (table of contents) APPEARANCE
\usepackage[nottoc,notlof,notlot]{tocbibind} % Put the bibliography in the ToC
\usepackage[titles,subfigure]{tocloft} % Alter the style of the Table of Contents
\renewcommand{\cftsecfont}{\rmfamily\mdseries\upshape}
\renewcommand{\cftsecpagefont}{\rmfamily\mdseries\upshape} % No bold!

\usepackage{amsmath}
\usepackage{graphicx}
\graphicspath{ {./pings/} }
\DeclareMathOperator*{\argmax}{arg\,max}
\DeclareMathOperator*{\argmin}{arg\,min}

\newcount\colveccount
\newcommand*\colvec[1]{
        \global\colveccount#1
        \begin{pmatrix}
        \colvecnext
}
\def\colvecnext#1{
        #1
        \global\advance\colveccount-1
        \ifnum\colveccount>0
                \\
                \expandafter\colvecnext
        \else
                \end{pmatrix}
        \fi
}

%%% END Article customizations

%%% The "real" document content comes below...

\title{Micro HW6}
\author{Michael B. Nattinger\footnote{I worked on this assignment with my study group: Alex von Hafften, Andrew Smith, Ryan Mather, and Tyler Welch. I have also discussed problem(s) with Emily Case, Sarah Bass, and Danny Edgel.}}

%\date{} % Activate to display a given date or no date (if empty),
         % otherwise the current date is printed 

\begin{document}
\maketitle

\section{Question 1}
\subsection{Part A}
For the first three data points, $p \cdot x = 100,$ and for the final data point $p \cdot x = 150.$ Thus the data is consistent with Walras law.

\subsection{Part B}
$x^3>x^4>\{x^1,x^2\}$ which means that $x^3$ was not affordable at $p^4,p^2,p^1.$ We also know that $\{ x^1,x^2,x^3\}$ were affordable at $p^3$ but not chosen, so $x^3\succ \{x^1,x^2,x^4 \}$ and there were no conflicting revealed preferences. By similar logic $x^4 \succ \{x^1,x^2 \}$ with no conflicting revealed preferences. Now, $p^1\cdot x^2 = 85<100$ so $x_1 \succ x_2$. $p^2 \cdot x^1 = 120>100$ so we have no conflicting revealed preferences. Thus, $x^3\succ x^4 \succ x^1 \succ x^2$ so GARP is satisfied, and the data is rationalizable.

\section{Question 2}
\subsection{Part A}
Using Roy's Identity, $x^i(p,w_i) = -\frac{\partial v^i}{\partial p} /\frac{\partial v^i}{\partial w_i} = -\frac{a'_i(p) + b'(p)w_i}{b(p)}$.
\subsection{Part B}
 Again, applying Roy's identity: $X(p,W) = -\frac{\partial V}{\partial p} /\frac{\partial V}{\partial W} = -\frac{\sum_{i=1}^{n} (a'_i(p)) + b'(p)W}{b(p)} = -\frac{\sum_{i=1}^{n} (a'_i(p) + b'(p)w_i)}{b(p)} =\sum_{i=1}^n -\frac{a'_i(p) + b'(p)w_i}{b(p)}$
\section{Question 3}
\subsection{Part A}
Let preferences be homothetic. Then, for any $x(p,tw) = \argmax_{x\in B(p,tw)}u(x)$, $x(p,w) = \argmax_{x\in B(p,w)}u(x)$, notice that $p\cdot x(p,w) \leq w \Rightarrow p \cdot t x(p,w) \leq tw$ so $tx(p,w)$ is affordable for wealth level $tw.$ Next, let $ty\in B(p,tw)$ be arbitrary. Then, note that $y \in B(p,w) \Rightarrow x(p,w) \succeq y \Rightarrow tx(p,w)\succeq ty$ so $tx(p,w) =  \argmax_{x\in B(p,tw)}u(x) = x(p,tw)$
\subsection{Part B}
We will use the same utility representation as in class, where $u(x) = \alpha$ is the value of $\alpha$ for which $x \sim (\alpha,\dots,\alpha)$. We showed in class that, if preferences were continuous and monotone, $u(x)$ represents the preference relation.  We will now show that this is homogenous of degree 1 for homothetic preferences:

Let $u(x) = \alpha.$ This means that $x \sim (\alpha,\dots,\alpha) \Rightarrow tx \sim t(\alpha,\dots,\alpha)$ because $tx\succeq t(\alpha,\dots,\alpha),t(\alpha,\dots,\alpha)\succeq tx.$ Thus, $u(tx) = t\alpha$ so $u(.)$ is homogenous of degree 1.

\subsection{Part C}
$v(p,w) = u(x(p,w))= wu(x(p,1)) = wb(p)$ where $b(p) = u(x(p,1))$.
\section{Question 4}
\subsection{Part A}
Due to our utility function strictly increasing in $x_1,$ our preferences are LNS so our budget constraint will hold with equality, $w = x_1 + \sum_{i=2}^k p_ix_i \Rightarrow x_1 = w - \sum_{i=2}^k p_ix_i$. We then have the following:
\begin{align*}
X(p,w) &= \argmax_{x\in B(p,w)} u(x) = \argmax_{x\in B(p,w)} x_1 + U(x_2,\dots,x_k) = \argmax_{x}  w - \sum_{i=2}^k p_ix_i + U(x_2,\dots,x_k)\\&=\argmax_{x}  - \sum_{i=2}^k p_ix_i + U(x_2,\dots,x_k) = X_{2,\dots,k}(p), 
\end{align*}
so Marshallian demand for goods $2-k$ does not depend on wealth.

\subsection{Part B}
\begin{align*}
v(p,w) &= u(X(p,w)) \\%= u\left(\left(\left(w - \sum_{i=2}^k p_i X_i\right) \text{ } X_{2,\dots,k}^{T}(p)\right)^{T}\right)\\
&= w - \sum_{i=2}^k p_i X_i+ U(X_{2,\dots,k}) = w - g(X_{2,\dots,k}) +U(X_{2,\dots,k}) \\&= w + \tilde{v}(X_{2,\dots,k})
\end{align*}
\subsection{Part C}
\begin{align*}
e(p,u) = \min_{u(x)\geq u} p \cdot x = \min_{u(x)\geq u} x_1 + \sum_{i=2}^k p_ix_i.
\end{align*}
If the constraint were to hold without equality, we could reduce our spending on $x_1$ until the constraint held with equality and reduce costs. Thus, the constraint must hold with equality. Thus, $u = x_1 +U(x_2,\dots,x_k) \Rightarrow x_1 = u - U(x_2,\dots,x_k)$. Plugging this constraint into the objective function,
\begin{align*}
e(p,u) &= \min_{x} u - U(x_2,\dots,x_k) + \sum_{i=2}^k p_ix_i = u - \min_{x} - U(x_2,\dots,x_k) + \sum_{i=2}^k p_ix_i \\&= u - f(p).
\end{align*}
\subsection{Part D}
\begin{align*}
h(p,u) &= \argmin_{u(x)\geq u} p \cdot x = \argmin_{u(x)\geq u} x_1 + \sum_{i=2}^k p_ix_i = \argmin_{x} u - U(x_2,\dots,x_k) +  \sum_{i=2}^k p_ix_i \\&=  \argmin_{x} - U(x_2,\dots,x_k) +  \sum_{i=2}^k p_ix_i = h(p).
\end{align*}
\subsection{Part E}
Compensating variation is $\int_{p_i^1}^{p_i^0}h_i(p,u^0)dpi = \int_{p_i^1}^{p_i^0}h_i(p)dpi = \int_{p_i^1}^{p_i^0}h_i(p,u^1)dpi$, which is equivalent variation. Consumer surplus is $\int_{p_i^1}^{p_i^0}x_i(p,w)dpi = \int_{p_i^1}^{p_i^0}h_i(p,v(p,w))dpi =  \int_{p_i^1}^{p_i^0}h_i(p)dpi $ as both Hicksian and Marshallian demand for good $i$ are functions only of price, so at each price they must be equal.%is $e(p^0,u^0) -e(p^1,u^0) = u^0 - f(p^0) - u^0 + f(p^1) =  f(p^1) - f(p^0) .$ Equivalent variation is $e(p^0,u^1) -e(p^1,u^1) = u^1 - f(p^0) - u^1 + f(p^1) =  f(p^1) - f(p^0).$
\end{document}
