% !TEX TS-program = pdflatex
% !TEX encoding = UTF-8 Unicode

% This is a simple template for a LaTeX document using the "article" class.
% See "book", "report", "letter" for other types of document.

\documentclass[11pt]{article} % use larger type; default would be 10pt

\usepackage[utf8]{inputenc} % set input encoding (not needed with XeLaTeX)

%%% PAGE DIMENSIONS
\usepackage{geometry} % to change the page dimensions
\geometry{a4paper} % or letterpaper (US) or a5paper or....

\usepackage{graphicx} % support the \includegraphics command and options

\usepackage{amssymb}
\usepackage{amsmath}
%%% PACKAGES
\usepackage{booktabs} % for much better looking tables
\usepackage{array} % for better arrays (eg matrices) in maths
\usepackage{paralist} % very flexible & customisable lists (eg. enumerate/itemize, etc.)
\usepackage{verbatim} % adds environment for commenting out blocks of text & for better verbatim
\usepackage{subfig} % make it possible to include more than one captioned figure/table in a single float
% These packages are all incorporated in the memoir class to one degree or another...

%%% HEADERS & FOOTERS
\usepackage{fancyhdr} % This should be set AFTER setting up the page geometry
\pagestyle{fancy} % options: empty , plain , fancy
\renewcommand{\headrulewidth}{0pt} % customise the layout...
\lhead{}\chead{}\rhead{}
\lfoot{}\cfoot{\thepage}\rfoot{}

%%% SECTION TITLE APPEARANCE
\usepackage{sectsty}
\allsectionsfont{\sffamily\mdseries\upshape} % (See the fntguide.pdf for font help)
% (This matches ConTeXt defaults)

%%% ToC (table of contents) APPEARANCE
\usepackage[nottoc,notlof,notlot]{tocbibind} % Put the bibliography in the ToC
\usepackage[titles,subfigure]{tocloft} % Alter the style of the Table of Contents
\renewcommand{\cftsecfont}{\rmfamily\mdseries\upshape}
\renewcommand{\cftsecpagefont}{\rmfamily\mdseries\upshape} % No bold!

\usepackage{hyperref}
\usepackage{amsmath}
\usepackage{graphicx}
\graphicspath{ {./pings/} }
\DeclareMathOperator*{\argmax}{arg\,max}
\DeclareMathOperator*{\argmin}{arg\,min}

\newcount\colveccount
\newcommand*\colvec[1]{
        \global\colveccount#1
        \begin{pmatrix}
        \colvecnext
}
\def\colvecnext#1{
        #1
        \global\advance\colveccount-1
        \ifnum\colveccount>0
                \\
                \expandafter\colvecnext
        \else
                \end{pmatrix}
        \fi
}

%%% END Article customizations

%%% The "real" document content comes below...

\title{Micro HW1}
\author{Michael B. Nattinger\footnote{I worked on this assignment with my study group: Alex von Hafften, Andrew Smith, Ryan Mather, and Tyler Welch. I have also discussed problem(s) with Emily Case, Sarah Bass, Katherine Kwok, and Danny Edgel.}}

%\date{} % Activate to display a given date or no date (if empty),
         % otherwise the current date is printed 

\begin{document}
\maketitle

\section{Question 1}
\subsection{Part A}
The game consists of players $I = \{ 1,\dots,N\}$. Each player is of type $\theta \sim  F$. Each player plays a strategy $b:[0,1]\rightarrow \mathbb{R}$ which is a function of their type. The a priori belief is that everyone's type is drawn from $F$. The interim belief is full knowledge of one's own type, and knowing that everyone else's type is drawn from $F$. The ex ante belief is full knowledge of all types. The person with the highest bid wins the auction and thus gains their valuation, and all players (including the winner) pays their bid regardless of winning or losing.
\subsection{Part B}
Each player faces the following utility maximization problem:
\begin{align*}
\max_{b_i} E[U(b_i)]\\
\max_{b_i} \theta_i Pr(b_i>\max_{j\neq i}\{b_{j}\}) - b
\end{align*}
In our case, with $iid$ observations, $Pr(b_i>\max_{j\neq i}\{b_{j}\}) = \Pi_{j\neq i}Pr(b_i>b_{j})$. We can also now plug in the strategy of the other players as a function. Our maximization problem now takes the following form, where $\theta_{-i}$ is any draw from $F$: 

\begin{align*}
\max_{b_i} \theta_i (Pr(b_i>b(\theta_{-i})))^{N-1} - b_i\\
\max_{b_i} \theta_i (Pr(b^{-1}(b_i)>\theta_{-i}))^{N-1} - b_i \\
\max_{b_i} \theta_i (b^{-1}(b_i))^{aN-a} - b_i
\end{align*}

Taking first order conditions,
\begin{align*}
1 &= \theta_i(aN-a) (b^{-1}(b_i))^{aN-a-1}\frac{1}{b'(b^{-1}(b_i))}\\
&= \frac{ (aN-a) (\theta_i)^{aN-a}}{b'(\theta_i))}\\
b'(\theta_i) &= (aN-a) (\theta_i)^{aN-a}\\
\Rightarrow b(\theta_i) &= \frac{aN-a}{aN-a+1}(\theta_i)^{aN-a+1} + c_{\theta_i}
\end{align*}
Players with a valuation of $0$ will bid $0$ so $c_{\theta_i}=0$. Therefore, $ b(\theta_i) = \frac{aN-a}{aN-a+1}(\theta_i)^{aN-a+1}$
%Each player will bid such that their expected payoff is $0$:
%\begin{align*}
%b(\theta) &= \theta Pr(b_i>\max_{j\neq i}\{b_{j}\})\\
%&= \theta \Pi_{j\neq i}Pr(b_i>b_{j}) \\
%&= \theta \Pi_{j\neq i}Pr(\theta_i>\theta_{j}) \\
%&= \theta (\theta^a )^{N-1}\\
%&= \theta^{1+aN-a}.
%\end{align*}
\subsection{Part C}
We can verify by rewriting the optimization problem as we did originally with the known function $b$ and show that the best reply is to follow said function.
\begin{align*}
\max_{b_i} \theta_i (Pr(b_i > b(\theta_{-i})))^{N-1} - b_i\\
\max_{b_i} \theta_i (Pr(b_i >  \frac{aN-a}{aN-a+1}(\theta_{-i})^{aN-a+1}))^{N-1} - b_i\\
\max_{b_i} \theta_i (Pr(\left(\frac{aN-a+1}{aN-a}b_i\right)^{\frac{1}{aN-a+1}} > \theta_{-i})^{N-1} - b_i\\
\max_{b_i} \theta_i \left(\frac{aN-a+1}{aN-a}b_i\right)^{\frac{aN-a}{aN-a+1}}  - b_i\\
\end{align*}
Taking FOCs:
\begin{align*}
1 &= \theta_i \frac{aN-a}{aN-a+1} \left(\frac{aN-a+1}{aN-a}b_i\right)^{\frac{aN-a}{aN-a+1} - 1}\frac{aN-a+1}{aN-a}\\
\left(\frac{aN-a+1}{aN-a}b_i\right)^{\frac{1}{aN-a+1}} &= \theta_i \\
b_i &=  \frac{aN-a}{aN-a+1}(\theta_i)^{aN-a+1} 
\end{align*}
Therefore , $b(\theta) =  \frac{aN-a}{aN-a+1}(\theta)^{aN-a+1}$ is a best response to the other players playing the same strategy. Thus, it constitutes an equilibrium.
\subsection{Part D}
As $a$ increases, bids will shrink for given types less than 1, so the bidding will become less competitive for a given type. Intuitively, as $a$ rises, more of the distribution mass is near $1$ so for a fixed $\theta$ value less than one the odds of winning decrease, and therefore the bid decreases. However, note that there is a competing effect here: the bids for a given type is decreasing, however as $a$ increases the probability mass of types shifts to the right so the drawn types tend to be higher. In the limit as $a\rightarrow \infty$ this effect dominates and the bids tend to increase. 
\subsection{Part E}

Before drawing a type, the expected bid is the following:
\begin{align*}
E[b_i(\theta)] &= \int_{0}^{1} \frac{aN-a}{aN-a+1}(\theta)^{aN-a+1}  a\theta^{a-1} d\theta \\
&=  \frac{a(aN-a)}{aN-a+1} \int_{0}^{1}\theta^{aN}d\theta\\
&=   \frac{a(aN-a)}{aN-a+1}\frac{1}{(aN+1)} 
\end{align*}

I originally interpreted the question as being the expected return of the bidder before and after drawing the type. This is not what the question asks but I still have it below.

After drawing a type, person $i$ with type $\theta_i \sim F$ bids $b(\theta_i) =  \frac{aN-a}{aN-a+1}(\theta_i)^{aN-a+1}$ and wins payoff $\theta_i$ with some probability:
\begin{align*}
E[\pi|\theta] &= \theta_i Pr(\theta_i>\theta_{-i})^{N-1} - b(\theta_i)\\
&= \theta_i^{aN-a+1} - \frac{aN-a}{aN-a+1}(\theta_i)^{aN-a+1}\\
&= \frac{1}{aN - a + 1}\theta_i^{aN-a+1}
\end{align*}

We can find the expected winnings prior to drawing a type by taking the expectation of winnings by type across the distribution of types. Note that $F(\theta) = \theta^a \Rightarrow f(\theta) = a\theta^{a-1}$. Then, we have the following:
\begin{align*}
E[\pi] &= \int_{0}^1 E[\pi|\theta] f(\theta)d\theta \\
&=  \int_{0}^1\frac{1}{aN - a + 1}\theta^{aN-a+1} a\theta^{a-1}d\theta\\
&= \frac{a}{aN - a + 1}\int_{0}^1\theta^{aN} d\theta\\
&=  \frac{a}{aN - a + 1} \frac{1}{aN+1}.
\end{align*}
\section{Question 2}
\subsection{Part A}
In the first-price auction, both people will bid below their valuations, to try to earn surplus. If the highest bid is above $r$ then the seller will sell the object to the highest bidder at the highest bid price. If none of the bids is above $r$ then the seller keeps the object.
\subsection{Part B}
In the second-price auction, both people will bid their valuations. If both bids are above $r$, then the seller with the higher bid will pay the seller the second-highest price, and will receive the object. If fewer than two bids are above the reserve price then the seller will not sell the object.
\subsection{Part C}
The seller should choose the first-price auction as the two auction types have the same expected revenue, but the first price auction has the smaller spread. They should set the reserve price to be their own cost, c. The person with type drawn from $F(x) = x^2$ has more probability mass near 1 so this person wins more often.
\subsection{Part D}
The discount should be offered to the type that tends to have the lowest valuation $F(x) = x$. The idea is to try to get them to bid over their valuation, so that their bid gets above the bid of the type which tends to be high, raising the sale price of the good. We calculate the optimal discount.

The type that receives the discount will bid such that they would receive no surplus buying the item at that price: $\alpha b - x = 0\Rightarrow b = \frac{x}{\alpha}$. The other type bids their valuation. The seller chooses $\alpha$ to maximize expected revenue:

\begin{align*}
&\max_{\alpha} E\left[\frac{x_1}{\alpha}|\frac{x_1}{\alpha}<x_2\right] P\left(\frac{x_1}{\alpha}<x_2\right) + E\left[x_2|\frac{x_1}{\alpha}>x_2\right] P\left(\frac{x_1}{\alpha}>x_2\right)\\
&\max_{\alpha} \frac{1}{\alpha}\int_{0}^1\int_{0}^{\alpha x_2} x_1 dx_1 2x_2 dx_2 \int_{0}^1\int_{0}^{\alpha x_2} dx_1 2x_2 dx_2 + \int_{0}^1\int_{0}^{x_1/\alpha} 2x_2^2 dx_2 dx_1 \int_{0}^1\int_{0}^{x_1/\alpha}2x_2 dx_2  dx_1 \\
&\max_{\alpha} 2\alpha^2\int_{0}^1 x_2^3  dx_2 \int_{0}^1 x_2^2 dx_2 + \frac{2}{3\alpha^5}\int_{0}^1 x_1^3 dx_1 \int_{0}^1 x_1^2 dx_1 \\
&\max_{\alpha} \frac{\alpha^2}{6}+ \frac{1}{18\alpha^5} 
\end{align*}
Taking FOCs:
\begin{align*}
\frac{\alpha}{3} &= \frac{5}{18\alpha^6}\\
\alpha^7 &= \frac{5}{6}\\
\alpha &= \left( \frac{5}{6} \right)^{1/7}.
\end{align*}

\section{Question 3}
\subsection{Part A}
The game consists of players $I = \{ 1,2,3\}$. Each player is of type $v \sim  [0,1]$. Each player plays a strategy $b:[0,1]\rightarrow \mathbb{R}$ which is a function of their type. The a priori belief is that everyone's type is drawn from $[0,1]$. The interim belief is full knowledge of one's own type, and knowing that everyone else's type is drawn from $[0,1]$. The ex ante belief is full knowledge of all types. The person with the highest bid wins the auction and thus gains their valuation, and that player pays the lowest bid.
\subsection{Part B}
Player $i\in I$ maximizes their expected value. We will assume that  players $-i$ play the strategy $b(v) = \frac{n-1}{n-2}v $, then prove that the best response is the same strategy.

The optimization problem is the following:
\begin{align*}
%\max_{b_i} (v_i - \frac{n-1}{n-2}\min_{j \neq i}v_j)Pr(b_i>\frac{n-1}{n-2}\max_{j\neq i} v_j)\\
%\max_{b_i} (v_i - \frac{n-1}{n-2}\min_{j \neq i}v_j)Pr(b_i>\frac{n-1}{n-2}v_{-i})^2\\
%\max_{b_i} (v_i - \frac{n-1}{n-2}\min_{j \neq i}v_j)(b_i\frac{n-2}{n-1})^{n-1}
\max_{b_i}E[v_i - \frac{n-1}{n-2}\text{second}_{j \neq i}v_j|b_i>\frac{n-1}{n-2}\max_{j\neq i} v_j]Pr(b_i>\frac{n-1}{n-2}\max_{j\neq i} v_j) 
\end{align*}

\begin{align*}
Pr(b_i>\frac{n-1}{n-2}\max_{j\neq i} v_j) &= Pr\left(b_i>\frac{n-1}{n-2}v_{-i}\right)^{n-1}\\
&= \left(b_i\frac{n-2}{n-1}\right)^{n-1}\\
E[v_i - \frac{n-1}{n-2}\text{second}_{j \neq i}v_j|b_i>\frac{n-1}{n-2}\max_{j\neq i} v_j]  &= E[v_i - \frac{n-1}{n-2}\text{second}_{j \neq i}v_j|\frac{n-2}{n-1}b_i>v_{j}, j\neq i ] \\
&= v_i - \frac{n-1}{n-2}E[\text{second}_{j \neq i}v_j|\frac{n-2}{n-1}b_i>v_{j}, j\neq i ]
\end{align*}

This conditional expectation is equivalent to the expectation of the second largest of $n-1$ draws from the uniform distribution $U\left(0,\frac{n-2}{n-1}b_i\right)$. The expectation of this is the expectation of the second largest of $n-1$ draws from a standard uniform distribution, scaled by $\frac{n-2}{n-1}b_i$. This expectation is, therefore, $\frac{n-2}{n-1}b_i \frac{n-2}{n}$.\footnote{See \href{https://www2.stat.duke.edu/courses/Spring12/sta104.1/Lectures/Lec15.pdf}{these lecture slides} from Duke which contain formulas from which the expectation of the $k$th order statistic of a standard uniform can be constructed. Specifically, slides 8 and 11.} We then have the following maximization problem:
\begin{align*}
\max_{b_i} v_i \left(b_i\frac{n-2}{n-1}\right)^{n-1} - \frac{n-1}{n-2} \frac{(n-2)^2}{n-1} \frac{b_i}{n} \left(b_i\frac{n-2}{n-1}\right)^{n-1}\\
\max_{b_i} v_i \left(b_i\frac{n-2}{n-1}\right)^{n-1} -   \frac{n-2}{n} b_i^n\left(\frac{n-2}{n-1}\right)^{n-1}\\
\end{align*}
Taking FOCs:
\begin{align*}
(n-1)v_ib_i^{n-2}\left(\frac{n-2}{n-1}\right)^{n-1} &= (n-2)b_i^{n-1}\left(\frac{n-2}{n-1} \right)^{n-1}\\
\Rightarrow b_i &= \frac{n-1}{n-2}v_i
\end{align*}
Thus, the best response to the other players playing $b(v_i) = \frac{n-1}{n-2}v_i$ is also to play $(v_i) = \frac{n-1}{n-2}v_i$, so the bid function $b$ is a symmetric Bayes-Nash equilibrium of the third-price auction.

\subsection{Part C}
The expected revenue of the seller is the expectation of the third largest bid:
\begin{align*}
R^3 &= E[\text{third}_{i\in I} b_i]\\
&= E[\text{third}_{i\in I}  \frac{n-1}{n-2}v_i]\\
&=  \frac{n-1}{n-2}E[\text{third}_{i\in I}  v_i]\\
&=  \frac{n-1}{n-2}\frac{n-2}{n+1}\\
&= \frac{n-1}{n+1}.
\end{align*}
\subsection{Part D}
We know from class that for $k =1$ the equilibrium strategy is to underbid and for $k=2$ the equilibrium strategy is to bid the valuation. For $k=3$ we showed above that the equilibrium strategy is to overbid. As $k$ continues to increase above $3$, the equilibrium strategy is to overbid by an increasing amount. One can show that $b^k_i(v_i) = \frac{n-1}{n-k+1}$ is the equilibrium strategy for the $k$'th price auction. I do this below:

\begin{align*}
\max_{b_i}E[v_i - \frac{n-1}{n-k+1}\text{k-1'th}_{j \neq i}v_j|b_i>\frac{n-1}{n-k+1}\max_{j\neq i} v_j]Pr(b_i>\frac{n-1}{n-k+1}\max_{j\neq i} v_j)  \\ %
\max_{b_i}(v_i - \frac{n-1}{n-k+1}E[\text{k-1'th}_{j \neq i}v_j|b_i>\frac{n-1}{n-k+1}\max_{j\neq i} v_j])\left(b_i\frac{n-k+1}{n-1}\right)^{n-1} \\
\max_{b_i}(v_i - \frac{n-1}{n-k+1}\frac{n-k+1}{n-1} b_i \frac{n-k+1}{n})\left(b_i\frac{n-k+1}{n-1}\right)^{n-1}\\
\max_{b_i}v_i\left(b_i\frac{n-k+1}{n-1}\right)^{n-1} -  b_i \frac{n-k+1}{n}\left(b_i\frac{n-k+1}{n-1}\right)^{n-1}
\end{align*}

Taking FOCs:
\begin{align*}
(n-1)v_ib_i^{n-2}\left(\frac{n-k+1}{n-1}\right)^{n-1} &= b_i^n (n-k+1)\left(\frac{n-k+1}{n-1}\right)^{n-1}\\
\Rightarrow b_i &=\frac{n-1}{n-k+1}v_i .
\end{align*}
Therefore, $b^k_i(v_i) = \frac{n-1}{n-k+1}$ is the equilibrium strategy for the $k$'th price auction.
\end{document}
