% !TEX TS-program = pdflatex
% !TEX encoding = UTF-8 Unicode

% This is a simple template for a LaTeX document using the "article" class.
% See "book", "report", "letter" for other types of document.

\documentclass[11pt]{article} % use larger type; default would be 10pt

\usepackage[utf8]{inputenc} % set input encoding (not needed with XeLaTeX)

%%% Examples of Article customizations
% These packages are optional, depending whether you want the features they provide.
% See the LaTeX Companion or other references for full information.

%%% PAGE DIMENSIONS
\usepackage{geometry} % to change the page dimensions
\geometry{a4paper} % or letterpaper (US) or a5paper or....
\geometry{margin=1in} % for example, change the margins to 2 inches all round
% \geometry{landscape} % set up the page for landscape
%   read geometry.pdf for detailed page layout information

\usepackage{graphicx} % support the \includegraphics command and options

% \usepackage[parfill]{parskip} % Activate to begin paragraphs with an empty line rather than an indent
\usepackage{amssymb}
\usepackage{amsmath}
%%% PACKAGES
\usepackage{booktabs} % for much better looking tables
\usepackage{array} % for better arrays (eg matrices) in maths
\usepackage{paralist} % very flexible & customisable lists (eg. enumerate/itemize, etc.)
\usepackage{verbatim} % adds environment for commenting out blocks of text & for better verbatim
\usepackage{subfig} % make it possible to include more than one captioned figure/table in a single float
% These packages are all incorporated in the memoir class to one degree or another...

%%% HEADERS & FOOTERS
\usepackage{fancyhdr} % This should be set AFTER setting up the page geometry
\pagestyle{fancy} % options: empty , plain , fancy
\renewcommand{\headrulewidth}{0pt} % customise the layout...
\lhead{}\chead{}\rhead{}
\lfoot{}\cfoot{\thepage}\rfoot{}

%%% SECTION TITLE APPEARANCE
\usepackage{sectsty}
\allsectionsfont{\sffamily\mdseries\upshape} % (See the fntguide.pdf for font help)
% (This matches ConTeXt defaults)

%%% ToC (table of contents) APPEARANCE
\usepackage[nottoc,notlof,notlot]{tocbibind} % Put the bibliography in the ToC
\usepackage[titles,subfigure]{tocloft} % Alter the style of the Table of Contents
\usepackage{bbm}
\usepackage{endnotes}

\renewcommand{\cftsecfont}{\rmfamily\mdseries\upshape}
\renewcommand{\cftsecpagefont}{\rmfamily\mdseries\upshape} % No bold!
\DeclareMathOperator*{\argmax}{arg\,max}
\DeclareMathOperator*{\argmin}{arg\,min}

\usepackage{graphicx}
\graphicspath{ {./pings/} }

\newcount\colveccount
\newcommand*\colvec[1]{
        \global\colveccount#1
        \begin{pmatrix}
        \colvecnext
}
\def\colvecnext#1{
        #1
        \global\advance\colveccount-1
        \ifnum\colveccount>0
                \\
                \expandafter\colvecnext
        \else
                \end{pmatrix}
        \fi
}

\newcommand{\norm}[1]{\left\lVert#1\right\rVert}

\title{IO Problem Set 1}
\author{Michael B. Nattinger}

\begin{document}
\maketitle

\section{Question 1}
The demand curve with constant elasticity can be written as $Q = aP^{-c}$. Rewriting, the corresponding inverse demand function is $P(Q) = a^{1/c}Q^{-1/c}.$ Then, $P'(Q) = (-1/c)a^{1/c}Q^{-(1+c)/c}$, $P''(Q) = (-1/c)(-(1+c)/c) a^{1/c}Q^{-(1+2c)/c} = ((1+c)/c^2) a^{1/c}Q^{-(1+2c)/c}$ We then have the following:
\begin{align*}
P'(Q) + Q P''(Q) &=  (-1/c)a^{1/c}Q^{-(1+c)/c} + Q( ((1+c)/c^2) a^{1/c}Q^{-(1+2c)/c}) \\
 &=  (-1/c)a^{1/c}Q^{-(1+c)/c} + ( ((1+c)/c^2) a^{1/c}Q^{-(1+c)/c}) \\
&=(1/c^2)a^{1/c}Q^{-(1+c)/c}>0.
\end{align*}

Next, let N firms be competing a la Cournot.\\ Assumption (A1) is that $0\geq P''(Y)y_i + P'(Y) \forall y_i<Y.$ \\Assumption (A2) states that $0\geq P'(Y) - C''_i(y_i) \forall y_i < Q.$ We are given that each firm has identical cost functions, so $C_i(y) = C(y).$ Note that (A2) therefore states that $C''(y_i) \geq P'(Y)$. With identical costs, in equilibrium $y_i = y = Y/N$. Using this, (A1) becomes \\ $0\geq P''(Y)Y/N + P'(Y)$ % show (A1),(A2) imply price and per-firm quantity in eqlm decreasing in N. 

Let us set up the maximization problem for each firm:
\begin{align*}
&\max_{y_i} P(y_i + Y_{-i})y_i - C(y_i)\\
&\Rightarrow P'(Y)y_i + P(Y) - C'(y_i) = 0\\
&\Rightarrow P(Ny) = C'(y) - P'(Ny)y
\end{align*}
Differentiating both sides with respect to $N$,
\begin{align*}
\frac{\partial P(Y)}{\partial N} &= -P''(Y)y^2 \\
&\geq P'(Y)y 
\end{align*}

\section{Question 2}
Each bidder chooses a bid $b_i \in \mathbb{R}$ to maximize their payoffs:
\begin{align*}
b_i = \argmax_{b} \pi(b,b_{-1}),
\end{align*}
where the payoff $\pi(b,b_{-i})$ is:
\begin{align*}
\pi(b,b_{-i}) = \begin{cases} V - b, b>b_{-i} \\ 0, b<b_{-i}\\ (1/2)(V-b), b=b_{-i} \end{cases}.
\end{align*}

The equilibrium is $b_i = V \forall i$. Why is this the case? Suppose instead player $i$ bid $b_i>V$. Then, their payoff would be $V-b_i<0$. Moreover, suppose $b_i<V$. Then, person $i$ still only gets $0$ payoff. So, $b_i = V \forall i$ is an equilibrium. No other equilibrium can exist. To see why, first note that equilibria can only exist with $b_i = b_{-i}$ as otherwise the bidder with the largest bid is strictly better off reducing their bid by some $\epsilon$. If $b_i = b_{-i}<V$ then bidder $i$ is better off increasing their bid by an epsilon and winning positive payoff. If $b_i = b_{-i}>V$ then the best response for $i$ is to reduce their bid by an $\epsilon$ such that they are sure to receive $0$ payoff instead of negative expected payoff. So, the only equilibrium is $b_i= b_{-i} = V.$

Now consider the all-pay auction. The expected payoff $\pi(b,b_{-i})$ is given by the following:
\begin{align*}
\pi(b,b_{-i}) &= \begin{cases} V - b , b>b_{-i}, \\ -b, b<b_{-i} \\ (1/2)V - b, b=b_{-i}  \end{cases}
\end{align*}

First we will show that a pure strategy Nash equilibrium does not exist. Suppose it does. Then, $b_i = b_{-i}$ because otherwise the highest bidder would be strictly better off by reducing their bid by an $\epsilon$. Consider, then, $b_i = b_{-i} = b$. If $b<V$ then either bidder would be better off increasing their bid by an $\epsilon$ and winning $V$ surely. If $b\geq V$ then either bidder would be better off not bidding (or bidding zero). Therefore no pure strategy Nash equilibrium can exist.

Consider an equilibrium where each bidder bids $b>0$ with probability $p$ and $0$ with probability $1-p$. Each player must be indifferent between bidding $b$ and not bidding in order to mix. Taking as given that player $-i$ is playing this strategy, indifference of player $i$ implies:
\begin{align*}
p((1/2)V-b) + (1-p)(V - b) &= 0 \\
(p/2 + (1-p))V &= b
\end{align*}

Moreover, the $b$ must be such that one is weakly better off choosing $0$ than $b-\epsilon$, with equality in the limit:
\begin{align*}
p(0) + (1-p)V - b+\epsilon \leq 0\\
\Rightarrow b \geq \epsilon + (1-p)V\\
\end{align*}

Taking $\lim_{\epsilon\rightarrow 0}$, $b = (1-p)V$. Combining with our previous expression, $(p/2 + (1-p))V = b = (1-p)V$. Simplifying, this expression yields a unique $p$: $p = 0$.

%Moreover, $b$ must be the best choice of bid for any nonzero bid. Bidding $b+\epsilon$ to win surely must be worse, with equality in the limit as $\epsilon \rightarrow 0$.
%\begin{align*}
%p(p(1/2)V + (1-p)V - b) \geq p(V - b-\epsilon)\\
%p(1/2)V \geq \epsilon
%\end{align*}
\end{document}
