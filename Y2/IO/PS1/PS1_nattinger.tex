% !TEX TS-program = pdflatex
% !TEX encoding = UTF-8 Unicode

% This is a simple template for a LaTeX document using the "article" class.
% See "book", "report", "letter" for other types of document.

\documentclass[11pt]{article} % use larger type; default would be 10pt

\usepackage[utf8]{inputenc} % set input encoding (not needed with XeLaTeX)

%%% Examples of Article customizations
% These packages are optional, depending whether you want the features they provide.
% See the LaTeX Companion or other references for full information.

%%% PAGE DIMENSIONS
\usepackage{geometry} % to change the page dimensions
\geometry{a4paper} % or letterpaper (US) or a5paper or....
\geometry{margin=1in} % for example, change the margins to 2 inches all round
% \geometry{landscape} % set up the page for landscape
%   read geometry.pdf for detailed page layout information

\usepackage{graphicx} % support the \includegraphics command and options

% \usepackage[parfill]{parskip} % Activate to begin paragraphs with an empty line rather than an indent
\usepackage{amssymb}
\usepackage{amsmath}
%%% PACKAGES
\usepackage{booktabs} % for much better looking tables
\usepackage{array} % for better arrays (eg matrices) in maths
\usepackage{paralist} % very flexible & customisable lists (eg. enumerate/itemize, etc.)
\usepackage{verbatim} % adds environment for commenting out blocks of text & for better verbatim
\usepackage{subfig} % make it possible to include more than one captioned figure/table in a single float
% These packages are all incorporated in the memoir class to one degree or another...

%%% HEADERS & FOOTERS
\usepackage{fancyhdr} % This should be set AFTER setting up the page geometry
\pagestyle{fancy} % options: empty , plain , fancy
\renewcommand{\headrulewidth}{0pt} % customise the layout...
\lhead{}\chead{}\rhead{}
\lfoot{}\cfoot{\thepage}\rfoot{}

%%% SECTION TITLE APPEARANCE
\usepackage{sectsty}
\allsectionsfont{\sffamily\mdseries\upshape} % (See the fntguide.pdf for font help)
% (This matches ConTeXt defaults)

%%% ToC (table of contents) APPEARANCE
\usepackage[nottoc,notlof,notlot]{tocbibind} % Put the bibliography in the ToC
\usepackage[titles,subfigure]{tocloft} % Alter the style of the Table of Contents
\usepackage{bbm}
\usepackage{endnotes}

\renewcommand{\cftsecfont}{\rmfamily\mdseries\upshape}
\renewcommand{\cftsecpagefont}{\rmfamily\mdseries\upshape} % No bold!
\DeclareMathOperator*{\argmax}{arg\,max}
\DeclareMathOperator*{\argmin}{arg\,min}

\usepackage{graphicx}
\graphicspath{ {./pings/} }

\newcount\colveccount
\newcommand*\colvec[1]{
        \global\colveccount#1
        \begin{pmatrix}
        \colvecnext
}
\def\colvecnext#1{
        #1
        \global\advance\colveccount-1
        \ifnum\colveccount>0
                \\
                \expandafter\colvecnext
        \else
                \end{pmatrix}
        \fi
}

\newcommand{\norm}[1]{\left\lVert#1\right\rVert}

\title{IO Problem Set 1}
\author{Michael B. Nattinger}

\begin{document}
\maketitle

\section{Question 1}
\subsection{Part (a)}
The demand curve with constant elasticity can be written as $Q = aP^{-c}$. Rewriting, the corresponding inverse demand function is $P(Q) = a^{1/c}Q^{-1/c}.$ Then, $P'(Q) = (-1/c)a^{1/c}Q^{-(1+c)/c}$, $P''(Q) = (-1/c)(-(1+c)/c) a^{1/c}Q^{-(1+2c)/c} = ((1+c)/c^2) a^{1/c}Q^{-(1+2c)/c}$ We then have the following:
\begin{align*}
P'(Q) + Q P''(Q) &=  (-1/c)a^{1/c}Q^{-(1+c)/c} + Q( ((1+c)/c^2) a^{1/c}Q^{-(1+2c)/c}) \\
 &=  (-1/c)a^{1/c}Q^{-(1+c)/c} + ( ((1+c)/c^2) a^{1/c}Q^{-(1+c)/c}) \\
&=(1/c^2)a^{1/c}Q^{-(1+c)/c}>0.
\end{align*}

\subsection{Part (b)}
Next, let N firms be competing a la Cournot.\\ Assumption (A1) is that $0\geq P''(Y)y_i + P'(Y) \forall y_i<Y.$ \\Assumption (A2) states that $0\geq P'(Y) - C''_i(y_i) \forall y_i < Q.$ We are given that each firm has identical cost functions, so $C_i(y) = C(y).$ Note that (A2) therefore states that $C''(y_i) \geq P'(Y)$. With identical costs, in equilibrium $y_i = y = Y/N$. Using this, (A1) becomes \\ $0\geq P''(Y)Y/N + P'(Y)$ % show (A1),(A2) imply price and per-firm quantity in eqlm decreasing in N. 

Let us set up the maximization problem for each firm:
\begin{align*}
&\max_{y_i} P(y_i + Y_{-i})y_i - C(y_i)\\
&\Rightarrow P'(Y)y_i + P(Y) - C'(y_i) = 0\\
&\Rightarrow P(Y) = C'(Y/N) - P'(Y)Y/N \label{eq1} \tag{1}
\end{align*}
Differentiating both sides with respect to $N$,
\begin{align*}
\frac{\partial P(Y)}{\partial N} &= -C''(Y/N)(Y/N^2) +P'(Y)(Y/N^2) \\
&= (Y/N^2)(P'(Y) - C''(y))\\
&\leq 0.
\end{align*}

Now we can rewrite (\ref{eq1}) for $y$ and differentiate with respect to $N$:
\begin{align*}
y &= \frac{C'(y) - P(Ny)}{P'(Ny)}\\
\frac{\partial y}{\partial N} &= -\frac{C'(y)}{(P'(Ny))^2 }P''(Ny)y - y + \frac{P(Ny)}{(P'(Ny))^2}P''(Ny)y\\
&= -y - \frac{P''(Y)y}{P'(Y)^2}(P(Y) - C'(y)) \\
&= -y- \frac{P''(Y)y}{P'(Y)^2}P'(Y)y \\
&= y\left(-1 -\frac{P''(Y)y}{P'(Y)}\right)
&\leq 0,
\end{align*}
where the final inequality uses our rewritten form of (A1).

\section{Question 2}
Each bidder chooses a bid $b_i \in \mathbb{R}$ to maximize their payoffs:
\begin{align*}
b_i = \argmax_{b} \pi(b,b_{-1}),
\end{align*}
where the payoff $\pi(b,b_{-i})$ is:
\begin{align*}
\pi(b,b_{-i}) = \begin{cases} V - b, b>b_{-i} \\ 0, b<b_{-i}\\ (1/2)(V-b), b=b_{-i} \end{cases}.
\end{align*}

The equilibrium is $b_i = V \forall i$. Why is this the case? Suppose instead player $i$ bid $b_i>V$. Then, their payoff would be $V-b_i<0$. Moreover, suppose $b_i<V$. Then, person $i$ still only gets $0$ payoff. So, $b_i = V \forall i$ is an equilibrium. No other equilibrium can exist. To see why, first note that equilibria can only exist with $b_i = b_{-i}$ as otherwise the bidder with the largest bid is strictly better off reducing their bid by some $\epsilon$. If $b_i = b_{-i}<V$ then bidder $i$ is better off increasing their bid by an epsilon and winning positive payoff. If $b_i = b_{-i}>V$ then the best response for $i$ is to reduce their bid by an $\epsilon$ such that they are sure to receive $0$ payoff instead of negative expected payoff. So, the only equilibrium is $b_i= b_{-i} = V.$

Now consider the all-pay auction. The expected payoff $\pi(b,b_{-i})$ is given by the following:
\begin{align*}
\pi(b,b_{-i}) &= \begin{cases} V - b , b>b_{-i}, \\ -b, b<b_{-i} \\ (1/2)V - b, b=b_{-i}  \end{cases}
\end{align*}

First we will show that a pure strategy Nash equilibrium does not exist. Suppose it does. Then, $b_i = b_{-i}$ because otherwise the highest bidder would be strictly better off by reducing their bid by an $\epsilon$. Consider, then, $b_i = b_{-i} = b$. If $b<V$ then either bidder would be better off increasing their bid by an $\epsilon$ and winning $V$ surely. If $b\geq V$ then either bidder would be better off not bidding (or bidding zero). Therefore no pure strategy Nash equilibrium can exist.

We will now search for a mixed-strategy Nash equilibrium. The equilibrium solves for a pair of distributions from which bids are drawn, $(F_1(b),F_2(b)$ are the corresponding CDFs. We know that the distributions satisfy boundary conditions $F_i(0) = 0, F_i(V) = 1$ by nonnegativity of our bids and from the value of the object being $V$. Given these distributions, and assuming differentiability, the maximization problem and first order conditions are the following:

\begin{align*}
\max_{b} F_{-i}(b)V - b\\
Vf_{-i}(b) &= 1\\
\end{align*}
From this first order condition we see that the pdf $f_{j}(b)$ is constant across the support of the distribution, implying a uniform distribution across our support. Our CDFs then take the functional form $F_i(b) = b/V$. The seller's expected revenue is $E[R] = 2E[b_i] = 2V/2 = V$.

It is important to note that the bidders are just indifferent between mixing in this way and not bidding (or bidding zero surely). That is, each individual's $E[\pi] = P(\text{winning}_{i})V - E[b_i] = (1/2)V - V/2 = 0$. If the seller sets a reserve price then (applying symmetry) the probability of winning for each bidder will not change, so their expected bid will not increase (or they would be strictly better off by not bidding at all). Since each bidder's expected bid is will not change, the expected revenue for the seller will not increase.



%Consider an equilibrium where each bidder bids $b>0$ with probability $p$ and $0$ with probability $1-p$. Each player must be indifferent between bidding $b$ and not bidding in order to mix. Taking as given that player $-i$ is playing this strategy, indifference of player $i$ implies:
%\begin{align*}
%p((1/2)V-b) + (1-p)(V - b) &= 0 \\
%(p/2 + (1-p))V &= b
%\end{align*}
%
%Moreover, the $b$ must be such that one is weakly better off choosing $0$ than $b-\epsilon$, with equality in the limit:
%\begin{align*}
%p(0) + (1-p)V - b+\epsilon \leq 0\\
%\Rightarrow b \geq \epsilon + (1-p)V\\
%\end{align*}
%
%Taking $\lim_{\epsilon\rightarrow 0}$, $b = (1-p)V$. Combining with our previous expression, $(p/2 + (1-p))V = b = (1-p)V$. Simplifying, this expression yields a unique $p$: $p = 0$.

%Moreover, $b$ must be the best choice of bid for any nonzero bid. Bidding $b+\epsilon$ to win surely must be worse, with equality in the limit as $\epsilon \rightarrow 0$.
%\begin{align*}
%p(p(1/2)V + (1-p)V - b) \geq p(V - b-\epsilon)\\
%p(1/2)V \geq \epsilon
%\end{align*}

\section{Question 3}

Consumers derive $v = 3$ from coffee. Starbucks coffee locations are at locations $0,1$ and have prices $p_0,p_1$ respectively. Esquire is at location $0.5$ and has price $q$. Marginal cost is 0. Travel cost of the consumer is $d^2$ where $d$ is the distance (in miles) from the consumer to the coffee shop. We first solve for the indifferent consumers, under the assumption of existence. The indifferent consumer between Starbucks $0$ and Esquire has the following expressions:

\begin{align*}
v - p_0 - x^2 &= v - q - (0.5 - x)^2 \\
q-p_0 &= x^2 - x^2 + x - 0.25\\
x &= (q-p_0) + (1/4)
\end{align*}

The indifferent consumer between Starbucks $1$ and Esquire has the following similar expression:

\begin{align*}
 v - q - (0.5 - y)^2 &= v - p_1 - (1-y)^2\\
p_1 - q - y^2 + y - 0.25 &= -y^2 + 2y - 1\\
(3/4) + p_1 - q  &= y .
\end{align*}

Given $p_0,p_1$, Esquire sets $q$ to maximize profits:

\begin{align*}
&\max_{q} q(y - x) \\
&\max_{q} q((1/2) + p_1 - 2q  + p_0)\\
&\Rightarrow (1/2) + p_1 + p_0 = 4q \\
&\Rightarrow q = 1/8 + (1/4)(p_1 + p_0)
\end{align*}

Given $q$, Starbucks sets $p_1,p_0$ to maximize profits:
\begin{align*}
&\max_{p_0,p_1} p_0 x + p_1 (1-y)\\
&\max_{p_0,p_1} p_0 ((q-p_0) + (1/4)) + p_1(1-((3/4) + p_1 - q)) \\
&\Rightarrow p_0 = q/2 +1/8\\
&\Rightarrow p_1 =  q/2 +1/8 = p_0
\end{align*}

Combining our expressions, we reach the following:
\begin{align*}
q &= 1/8 + (1/2)(q/2 + 1/8)\\
q &= 1/4\\
p_0 = p_1 &= 1/4
\end{align*}

Given these prices, the indifferent consumers are at $x = 1/4, y = 3/4$. From this we can see that the market shares are split evenly between Esquire and Starbucks, each with a share of $1/2$ of the market.

Now we can consider what would happen if Starbucks and Esquire were to swap houses. WLOG let Starbucks control the coffee houses at points $(1/2),1$ and let Esquire control the coffee house at $0$. Given $p_0,p_1$ Esquire sets $q$ to maximize profits:

\begin{align*}
&\max_{q} qx\\
&\max_{q} q((p_0 - q) + 1/4)\\
&\Rightarrow q = p_0/2 + 1/8.
\end{align*}

Given $q,$ Starbucks sets $p_0,p_1$ to solve the following:

\begin{align*}
&\max_{p_0,p_1} p_0(y-x) + p_1(1-y)\\
&\max_{p_0,p_1} p_0(p_1 - p_0 + (3/4) - (1/4) - (p_0 - q)) + p_1(1 - (3/4) - p_1 + p_0)\\
&\Rightarrow p_1 - 4p_0 + (1/2) + q  = 0 \\
&\Rightarrow p_0  + (1/4) - 4p_1 + p_0 = 0  \\
&\Rightarrow p_0 = (5/28) + (2/7)q \\
&\Rightarrow p_1 = (3/14) + (1/7)q
\end{align*}

Combining our expressions,
\begin{align*}
q &= (1/8) + (1/7)q + (5/56)\\
q &= 1/4 \\
p_0 &= (5/28) + (2/7)(1/2) \\
p_0 &= 9/28\\
p_1 &= (3/14) + 1/28 \\
p_1 &= 1/4
\end{align*}

In terms of market shares, Esquire now controls $x = 1/4 + p_0 - q = 1/4 + 1/14 = 9/28$. Starbucks controls $1-x = 19/28$. Starbucks wins out by increasing their market share.

If they sold to Seattle Best, the situation is equivalent to the original setup and the prices will be $p_0 = p_1 = q = 1/4$ and the market shares will be $(1/4,1/2,1/4)$. 

\section{Question 4}
Agents Jack and Jim select locations to maximize profits. With prices $p$ fixed, denote $D_i(x,x_{-i})$ as demand for agent $i$ given they decide to locate at $x$. Taking as given the location of the other agent, agent $i$ sets location to maximize demand:

\begin{align*}
\max_{x}D_i(x,x_{-i}).
\end{align*}

We must deduce the form of $d_i(x,x_{-i})$. First, note that with equal prices the consumers will attend the bar which is closest to them:

\begin{align*}
D_i(x,x_{-i}) &= \begin{cases} \int_{0}^{(1/2)x_{-i} + (1/2) x } dx, x\leq x_{-i}\\ \int_{(1/2)x_{-i} + (1/2) x }^{1} dx, x>x_{-i}  \end{cases}.
\end{align*}

In equilibrium, both establishments will be at $(1/2)$. First note that this is indeed an equilibrium - deviation away will only reduce profits by the deviating party. Next note that any other set of locations cannot be an equilibrium. Suppose not. Then, for any other set of locations there will either be one bar with more market share, in which case the losing bar will be best off moving such that they instead control the majority of market power, or the market share will be split. If the market share is split then 2 cases are possible: the bars are symmetrically located about $1/2$, in which case either bar is better off deviating towards (1/2) to increase market power, or the bars are located on the same location (not at $1/2$), again at which point either bar has incentive to deviate by moving an $\epsilon$ towards $1/2$ which will strictly increase market share. So, the only equilibrium is for both establishments to be located at $1/2$.

No, I do not think this is optimal, the social planner would like to distribute the bars in different places to reduce total costs. The planner solves:
\begin{align*}
\min_{a,b}\int_{0}^{1/2}(i-a)^2di + \int_{1/2}^1 (i-b)^2 di \\
\min_{a,b}(1/24) - a/4 + a^2/2 + (7/24) - (3/4)b + b^2/2 \\
\end{align*}

Taking FOCs,
\begin{align*}
(1/4) = a \\
(3/4) = b. 
\end{align*}

So, the social planner would locate one bar at $1/4$ and the other at $3/4$.

\section{Question 5}
Firm $2$ takes the prices $p_{1L},p_{1R}$ as given and and sets $p_2$ to maximize profit. Note that they will never set $p_{2}>p_{1R}$ because otherwise they will receive no profit. The indifferent consumer, given prices, solves the following: $1 - i^{*} - P_{1L}  = 1 - (1- i^{*}) - P_2 \\ \Rightarrow i^{*} = (1/2) + (1/2)(p_2 - p_{1L})$. Consumers to the right of $i^{*} $ buy from Firm 2, consumers to the left buy from firm 1. Firm 2 chooses $p_2$ to maximize the following:
\begin{align*}
\max_{p_2\leq p_{1R}} p_2(1 - ((1/2) + (1/2)( p_2 - p_{1L}))) \\
\max_{p_2\leq p_{1R}} (1/2)p_2( 1 - ( p_2 - p_{1L})) 
\end{align*}

\begin{align*}
1 + p_{1L} &= 2p_2 \\
p_2 &= (1/2) + (1/2)p_{1L}
\end{align*}
Note that the above is assuming $p_2<p_{1R}$. As I will show imminently firm 1 will actually set $p_{1R} = 0$ so firm 2 doesn't actually have a choice of $p_2$ to make any profit.

Firm 1 takes $p_{2}$ as fixed and will set $p_{1R},p_{1L}$ to maimize profits. Note that, since the indifferent consumer will purchase from Firm 2, Firm 1 takes this as given and will set $p_{1R} = p_{2} - \epsilon$. Note that this means that the only equilibrium has $p_{1R} = p_{2} = 0$. The consumer that is indifferent between buying from firm 2 at price 0 and buying from firm 1 has $1 - i - p_{1L} = 1-(1-i) \Rightarrow i^{*} = (1/2)(1 - p_{1L})$. The firm then solves the following:
\begin{align*}
&\max_{p_{1L}} (1/2)p_{1L}(1-p_{1L}) \\
&\Rightarrow p_{1L} = 1/2 
\end{align*}

In equilibrium, profits for firm 1 are $1/8$.

If instead firm 1 exited the market at R then now the problem is simple. Firm 2 takes $p_{1} = p_{1L}$ as given and sets $p_2$:
\begin{align*}
&\max_{p_2} (1/2)p_2( 1 - ( p_2 - p_{1})) \\
&\Rightarrow p_2 = (1/2) + (1/2)p_{1}
\end{align*}

Now firm 1 takes as given $p_2$ and sets $p_1$. The indifferent consumer satisfies $1 - i - p_{1L} = 1-(1-i) - p_2$:
\begin{align*}
&\max_{p_{1}} (1/2)p_{1}(1+p_2-p_{1}) \\
&\Rightarrow p_{1} = 1/2 + (1/2)p_2 
\end{align*}

In equilibrium,
\begin{align*}
p_1 = p_2 = 1\\
(1/2)p_{1}(1+p_2-p_{1}) = 1/2\\
 (1/2)p_2( 1 - ( p_2 - p_{1})) = 1/2
\end{align*}
So, here firm 1 is better off exiting the $R$ market. By trying to compete in that market, they bring the price down so far that it influences their profit in the L market. In this case, the loss of profit in the L market makes them worse off overall by trying to compete in each.

\section{Question 6}
If the monopolist has two goods with qualities $s=1,s=2$, then the marginal cost of these goods are $c,2c$ respectively. Consumer $\theta$ who is indifferent between purchasing the two goods has $2(\theta - p_2) = \theta - p_1 \Rightarrow \theta = 2p_2 - p_1.$ Similarly, consumer $\theta'$ who is indifferent between buying and not buying the cheap good will satisfy $\theta' = p_1$. The monopolist then solves:
\begin{align*}
\max_{p_1,p_2} (1 - 2p_2 + p_1)(p_2 - 2c) + (2p_2 -  2p_1)(p_1 - c) \\
\Rightarrow p_2 - 2c + 2p_2 - 4p_1 + 2c = 0\\
\Rightarrow p_1 = (3/4)p_2 \\
\Rightarrow 1 - 4p_2 + p_1 +4c + 2p_1 - 2c = 0 \\
\Rightarrow p_2 = (1/4) + (3/4)p_1 +(1/2)c \\
\Rightarrow p_2 = (1/4) + (9/16)p_2 + (1/2)c \\
\Rightarrow p_2 = (4/7) + (8/7)c \\
\Rightarrow p_1 = (3/7) +(6/7)c
\end{align*}

If the monopolist has only one good with quality $s=1$, the marginal cost of this good is $c$. Consumer $\theta$ has utility $U(p) = \theta - p$. They will purchase so long as $\theta>p$. The firm solves:
\begin{align*}
\max_{p} p(1-p) - c(1-p) \\
\Rightarrow 1 - 2p +c = 0 \\
\Rightarrow p = \frac{1+c}{2}.
\end{align*}

%Under the one-good case, the profit of the monopolist is: $\pi ' =\left(\frac{1-c}{2}\right)^2$. Under the two-good case, the profit of the monopolist is the following:
%\begin{align*}
%\pi &= 
%\end{align*}

Consider the efficient solution. Note that this will correspond to the case where price equals marginal cost. It is efficient for consumer $\theta$ to consume good $2$ if $\theta>2c$, and efficient for consumer $\theta'$ to consume good 1 if $c\leq \theta' < 2c$. Note that this implies that for $c<1/2$ that producing only quality $s=1$ is strictly inefficient. Consider now the case of the monopolist producing both goods. The mass of households that buy good 2 is $1 - 2((4/7) + (8/7)c) + (3/7) +(6/7)c = (2/7) - (10/7)c$. This equals the mass in the efficient case iff $ (2/7) - (10/7)c = 1 - 2c \Rightarrow c = 5/4$ which is above $1/2$ and therefore is a contradiction.

Now let $p_2 \in [2c,1]$. Person $\theta$ will buy product 1 if both $2(\theta - p_2)<\theta - p_1 \Rightarrow \theta<2p_2 - p_1$ and $\theta>p_1$. The demand for firm 1 is then $D(p;p_2) = (2p_2 - 2p)$. Firm 1 then solves the following:

\begin{align*}
\max_{p} (p-c)(2p_2 - 2p)\\
\Rightarrow 2p_2 + 2c &= 4p\\
p_{br} &= (p_2+c)/2
\end{align*}

Let firm 1 have price $p_1 \in [c,1]$. Then, firm 2 solves the following:
\begin{align*}
\max_p (p-2c)(1-2p + p_1) \\
1 - 4p + p_1 + 4c = 0\\
p_{br} = c+\frac{1+p_1}{4}.
\end{align*}

Solving for an equilibrium,
\begin{align*}
p_2 &= c + (1/4) + (1/8)(p_2 + c) \\
p_2 &= (1/7)(9c + 2) \\
p_1 &= (1/7)(8c + 1)
\end{align*}

So long as $1\geq(1/7)(9c + 2)\geq (1/7)(8c + 1)\geq c$ this is an equilibrium. Note that the final inequality is trivial. Note also that the first inequality holds iff $5\geq 9c$ which is also trivially satisfied since $c\leq (1/2)$. Finally, note that the middle inequality holds. One more inequality must be satisfied: demand for good 2 must be greater than zero. Algebraically, $1-2p_2 + p_1 \geq 0 \Rightarrow 1 - (2/7)(9c+2) + (1/7)(8c+1) \geq 0 \Rightarrow c \leq 2/5.$

\end{document}
