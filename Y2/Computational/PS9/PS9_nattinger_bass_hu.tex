% !TEX TS-program = pdflatex
% !TEX encoding = UTF-8 Unicode

% This is a simple template for a LaTeX document using the "article" class.
% See "book", "report", "letter" for other types of document.

\documentclass[11pt]{article} % use larger type; default would be 10pt

\usepackage[utf8]{inputenc} % set input encoding (not needed with XeLaTeX)

%%% Examples of Article customizations
% These packages are optional, depending whether you want the features they provide.
% See the LaTeX Companion or other references for full information.

%%% PAGE DIMENSIONS
\usepackage{geometry} % to change the page dimensions
\geometry{a4paper} % or letterpaper (US) or a5paper or....
\geometry{margin=1in} % for example, change the margins to 2 inches all round
% \geometry{landscape} % set up the page for landscape
%   read geometry.pdf for detailed page layout information

\usepackage{graphicx} % support the \includegraphics command and options

% \usepackage[parfill]{parskip} % Activate to begin paragraphs with an empty line rather than an indent
\usepackage{amssymb}
\usepackage{amsmath}
%%% PACKAGES
\usepackage{booktabs} % for much better looking tables
\usepackage{array} % for better arrays (eg matrices) in maths
\usepackage{paralist} % very flexible & customisable lists (eg. enumerate/itemize, etc.)
\usepackage{verbatim} % adds environment for commenting out blocks of text & for better verbatim
\usepackage{subfig} % make it possible to include more than one captioned figure/table in a single float
% These packages are all incorporated in the memoir class to one degree or another...

%%% HEADERS & FOOTERS
\usepackage{fancyhdr} % This should be set AFTER setting up the page geometry
\pagestyle{fancy} % options: empty , plain , fancy
\renewcommand{\headrulewidth}{0pt} % customise the layout...
\lhead{}\chead{}\rhead{}
\lfoot{}\cfoot{\thepage}\rfoot{}

%%% SECTION TITLE APPEARANCE
\usepackage{sectsty}
\allsectionsfont{\sffamily\mdseries\upshape} % (See the fntguide.pdf for font help)
% (This matches ConTeXt defaults)

%%% ToC (table of contents) APPEARANCE
\usepackage[nottoc,notlof,notlot]{tocbibind} % Put the bibliography in the ToC
\usepackage[titles,subfigure]{tocloft} % Alter the style of the Table of Contents
\usepackage{bbm}
\usepackage{endnotes}
\usepackage{rotating}

\renewcommand{\cftsecfont}{\rmfamily\mdseries\upshape}
\renewcommand{\cftsecpagefont}{\rmfamily\mdseries\upshape} % No bold!
\DeclareMathOperator*{\argmax}{arg\,max}
\DeclareMathOperator*{\argmin}{arg\,min}

\usepackage{graphicx}
\graphicspath{ {./pings/} }

\newcount\colveccount
\newcommand*\colvec[1]{
        \global\colveccount#1
        \begin{pmatrix}
        \colvecnext
}
\def\colvecnext#1{
        #1
        \global\advance\colveccount-1
        \ifnum\colveccount>0
                \\
                \expandafter\colvecnext
        \else
                \end{pmatrix}
        \fi
}

\newcommand{\norm}[1]{\left\lVert#1\right\rVert}

\title{Computational Problem Set 9}
\author{Michael B. Nattinger, Sarah J. Bass, Xinxin Hu}

\begin{document}
\maketitle
\section{Results}
I wrote code for all three methods. Log likelihoods for the three methods are similar, with values near $-10000$. All three methods are computed very quickly, with runtimes for a single likelihood costing around a second of computational time for GHK and quadrature, and under 1/10 of a second under accept/reject. To speed up computation, all random draws come from various precomputed Halton draws of appropriate dimensions, mapped into the appropriate distributions via the inverse CDF.

\begin{center}
\begin{tabular}{ll}
\hline 
Method & Log Likelihood \\ 
\hline 
Quadrature & -1.201e+04 \\ 
GHK & -1.288e+04 \\ 
Accept/Reject & -1.481e+04 \\ 
\hline 
\end{tabular}
\end{center}

Maximum likelihood estimates of the parameters are given. Computational runtime is around 30 minutes using the default settings in fminunc.

Due to issues (as of yet undiagnosed) in the likelihood estimates from my implementation of the quadrature method under extreme paramterizations, I used the GHK method in the log likelihood. This fixed the issues in optimization procedure and yields realistic looking results. 


\begin{center}
\begin{tabular}{ll}
\hline 
  & $\hat{\theta}$ \\ 
\hline 
$\alpha_0$ & 3.331 \\ 
$\alpha_1$ & -2.138 \\ 
$\alpha_2$ & -6.593 \\ 
$\gamma$ & -0.5693 \\ 
$\rho$ & 0.03545 \\ 
score\_0 & -0.3621 \\ 
rate\_spread & -1.49 \\ 
i\_large\_loan & -0.7106 \\ 
i\_medium\_loan & -0.1296 \\ 
i\_refinance & 15.05 \\ 
age\_r & 10.71 \\ 
cltv & 5.886 \\ 
\hline 
\end{tabular}
\end{center}
\end{document}
