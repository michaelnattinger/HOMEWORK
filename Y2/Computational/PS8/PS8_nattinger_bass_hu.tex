% !TEX TS-program = pdflatex
% !TEX encoding = UTF-8 Unicode

% This is a simple template for a LaTeX document using the "article" class.
% See "book", "report", "letter" for other types of document.

\documentclass[11pt]{article} % use larger type; default would be 10pt

\usepackage[utf8]{inputenc} % set input encoding (not needed with XeLaTeX)

%%% Examples of Article customizations
% These packages are optional, depending whether you want the features they provide.
% See the LaTeX Companion or other references for full information.

%%% PAGE DIMENSIONS
\usepackage{geometry} % to change the page dimensions
\geometry{a4paper} % or letterpaper (US) or a5paper or....
\geometry{margin=1in} % for example, change the margins to 2 inches all round
% \geometry{landscape} % set up the page for landscape
%   read geometry.pdf for detailed page layout information

\usepackage{graphicx} % support the \includegraphics command and options

% \usepackage[parfill]{parskip} % Activate to begin paragraphs with an empty line rather than an indent
\usepackage{amssymb}
\usepackage{amsmath}
%%% PACKAGES
\usepackage{booktabs} % for much better looking tables
\usepackage{array} % for better arrays (eg matrices) in maths
\usepackage{paralist} % very flexible & customisable lists (eg. enumerate/itemize, etc.)
\usepackage{verbatim} % adds environment for commenting out blocks of text & for better verbatim
\usepackage{subfig} % make it possible to include more than one captioned figure/table in a single float
% These packages are all incorporated in the memoir class to one degree or another...

%%% HEADERS & FOOTERS
\usepackage{fancyhdr} % This should be set AFTER setting up the page geometry
\pagestyle{fancy} % options: empty , plain , fancy
\renewcommand{\headrulewidth}{0pt} % customise the layout...
\lhead{}\chead{}\rhead{}
\lfoot{}\cfoot{\thepage}\rfoot{}

%%% SECTION TITLE APPEARANCE
\usepackage{sectsty}
\allsectionsfont{\sffamily\mdseries\upshape} % (See the fntguide.pdf for font help)
% (This matches ConTeXt defaults)

%%% ToC (table of contents) APPEARANCE
\usepackage[nottoc,notlof,notlot]{tocbibind} % Put the bibliography in the ToC
\usepackage[titles,subfigure]{tocloft} % Alter the style of the Table of Contents
\usepackage{bbm}
\usepackage{endnotes}
\usepackage{rotating}

\renewcommand{\cftsecfont}{\rmfamily\mdseries\upshape}
\renewcommand{\cftsecpagefont}{\rmfamily\mdseries\upshape} % No bold!
\DeclareMathOperator*{\argmax}{arg\,max}
\DeclareMathOperator*{\argmin}{arg\,min}

\usepackage{graphicx}
\graphicspath{ {./pings/} }

\newcount\colveccount
\newcommand*\colvec[1]{
        \global\colveccount#1
        \begin{pmatrix}
        \colvecnext
}
\def\colvecnext#1{
        #1
        \global\advance\colveccount-1
        \ifnum\colveccount>0
                \\
                \expandafter\colvecnext
        \else
                \end{pmatrix}
        \fi
}

\newcommand{\norm}[1]{\left\lVert#1\right\rVert}

\title{Computational Problem Set 8}
\author{Michael B. Nattinger, Sarah J. Bass, Xinxin Hu}

\begin{document}
\maketitle
\section{Results}

First we compare the score with the numerical first derivative, then the hessian with the numerical second derivative:
\begin{center}
\begin{tabular}{|c|c|}
\hline 
Score & Numerical derivative \\
\hline \hline
-2.61e+03 & -2.61e+03 \\ 
-556 & -556 \\ 
-1.16e+03 & -1.16e+03 \\ 
-223 & -223 \\ 
-933 & -933 \\ 
-1.22e+03 & -1.22e+03 \\ 
-2.11e+03 & -2.11e+03 \\ 
-948 & -948 \\ 
-5.05e+03 & -5.05e+03 \\ 
-4.53e+03 & -4.53e+03 \\ 
-1.94e+04 & -1.94e+04 \\ 
-1.92e+04 & -1.92e+04 \\ 
-919 & -919 \\ 
-352 & -352 \\ 
-467 & -467 \\ 
-582 & -582 \\ 
-546 & -546 \\ 

\hline
\end{tabular}
\end{center}

\pagebreak

\begin{center}
\resizebox{.39\textwidth}{!}{%
\begin{sideways}
%\begin{center}
\begin{tabular}{|ccccccccccccccccc|}
\hline 
\multicolumn{17}{|c|}{Hessian}\\
\hline \hline
-3.22e+03 & -880 & -1.43e+03 & -388 & -1.31e+03 & -1.55e+03 & -2.62e+03 & -1.21e+03 & -6.3e+03 & -5.76e+03 & -2.38e+04 & -2.36e+04 & -1.4e+03 & -664 & -682 & -674 & -583 \\ 
-880 & -880 & 0 & -10.1 & -404 & -421 & -686 & -332 & -1.72e+03 & -1.66e+03 & -6.61e+03 & -6.56e+03 & -390 & -164 & -189 & -212 & -170 \\ 
-1.43e+03 & 0 & -1.43e+03 & -165 & -560 & -676 & -1.19e+03 & -545 & -2.8e+03 & -2.55e+03 & -1.05e+04 & -1.04e+04 & -587 & -284 & -309 & -299 & -267 \\ 
-388 & -10.1 & -165 & -716 & -186 & -188 & -325 & -152 & -784 & -694 & -2.74e+03 & -2.72e+03 & 43.9 & -59.8 & -105 & -92.4 & -77.3 \\ 
-1.31e+03 & -404 & -560 & -186 & -1.31e+03 & -694 & -973 & -501 & -2.56e+03 & -2.59e+03 & -9.55e+03 & -9.59e+03 & -502 & -215 & -291 & -299 & -193 \\ 
-1.55e+03 & -421 & -676 & -188 & -694 & -806 & -1.23e+03 & -585 & -3.02e+03 & -2.84e+03 & -1.14e+04 & -1.14e+04 & -660 & -312 & -326 & -326 & -283 \\ 
-2.62e+03 & -686 & -1.19e+03 & -325 & -973 & -1.23e+03 & -2.22e+03 & -992 & -5.13e+03 & -4.62e+03 & -1.92e+04 & -1.91e+04 & -1.23e+03 & -545 & -551 & -540 & -477 \\ 
-1.21e+03 & -332 & -545 & -152 & -501 & -585 & -992 & -529 & -2.37e+03 & -2.17e+03 & -8.87e+03 & -8.8e+03 & -557 & -248 & -257 & -253 & -220 \\ 
-6.3e+03 & -1.72e+03 & -2.8e+03 & -784 & -2.56e+03 & -3.02e+03 & -5.13e+03 & -2.37e+03 & -1.25e+04 & -1.13e+04 & -4.65e+04 & -4.61e+04 & -2.72e+03 & -1.3e+03 & -1.33e+03 & -1.32e+03 & -1.13e+03 \\ 
-5.76e+03 & -1.66e+03 & -2.55e+03 & -694 & -2.59e+03 & -2.84e+03 & -4.62e+03 & -2.17e+03 & -1.13e+04 & -1.08e+04 & -4.26e+04 & -4.23e+04 & -2.42e+03 & -1.17e+03 & -1.22e+03 & -1.21e+03 & -1.03e+03 \\ 
-2.38e+04 & -6.61e+03 & -1.05e+04 & -2.74e+03 & -9.55e+03 & -1.14e+04 & -1.92e+04 & -8.87e+03 & -4.65e+04 & -4.26e+04 & -1.77e+05 & -1.75e+05 & -1.01e+04 & -4.91e+03 & -5.02e+03 & -4.97e+03 & -4.27e+03 \\ 
-2.36e+04 & -6.56e+03 & -1.04e+04 & -2.72e+03 & -9.59e+03 & -1.14e+04 & -1.91e+04 & -8.8e+03 & -4.61e+04 & -4.23e+04 & -1.75e+05 & -1.74e+05 & -9.98e+03 & -4.85e+03 & -4.98e+03 & -4.94e+03 & -4.25e+03 \\ 
-1.4e+03 & -390 & -587 & 43.9 & -502 & -660 & -1.23e+03 & -557 & -2.72e+03 & -2.42e+03 & -1.01e+04 & -9.98e+03 & -1.4e+03 & -293 & -307 & -306 & -268 \\ 
-664 & -164 & -284 & -59.8 & -215 & -312 & -545 & -248 & -1.3e+03 & -1.17e+03 & -4.91e+03 & -4.85e+03 & -293 & -664 & 0 & 0 & 0 \\ 
-682 & -189 & -309 & -105 & -291 & -326 & -551 & -257 & -1.33e+03 & -1.22e+03 & -5.02e+03 & -4.98e+03 & -307 & 0 & -682 & 0 & 0 \\ 
-674 & -212 & -299 & -92.4 & -299 & -326 & -540 & -253 & -1.32e+03 & -1.21e+03 & -4.97e+03 & -4.94e+03 & -306 & 0 & 0 & -674 & 0 \\ 
-583 & -170 & -267 & -77.3 & -193 & -283 & -477 & -220 & -1.13e+03 & -1.03e+03 & -4.27e+03 & -4.25e+03 & -268 & 0 & 0 & 0 & -583 \\ 

\hline
\end{tabular}
%\end{center}
\end{sideways}
}
\end{center}

\pagebreak

\begin{center}
\resizebox{.39\textwidth}{!}{%
\begin{sideways}
%\begin{center}
\begin{tabular}{|ccccccccccccccccc|}
\hline 
\multicolumn{17}{|c|}{Hessian (numerical) }\\
\hline \hline
-7.34e+03 & -2.12e+03 & -3.26e+03 & -1.81e+03 & -2.79e+03 & -2.97e+03 & -5.05e+03 & -3.35e+03 & -8.21e+03 & -8.08e+03 & -2.51e+04 & -2.49e+04 & -3.25e+03 & -799 & -719 & -2.8e+03 & -2.47e+03 \\ 
-2.12e+03 & -2.12e+03 & 0 & -460 & -917 & -816 & -1.36e+03 & -1e+03 & -2.36e+03 & -2.35e+03 & -7.03e+03 & -6.97e+03 & -960 & -199 & -215 & -862 & -722 \\ 
-3.26e+03 & 0 & -3.26e+03 & -800 & -1.18e+03 & -1.3e+03 & -2.32e+03 & -1.55e+03 & -3.64e+03 & -3.58e+03 & -1.1e+04 & -1.1e+04 & -1.37e+03 & -335 & -327 & -1.23e+03 & -1.13e+03 \\ 
-1.81e+03 & -460 & -800 & -757 & -668 & -115 & -960 & -1.17e+03 & 638 & 198 & -2.5e+03 & -2.51e+03 & -564 & 3.64 & 83.7 & -1.01e+03 & -889 \\ 
-2.79e+03 & -917 & -1.18e+03 & -668 & -2.79e+03 & -1.11e+03 & -1.61e+03 & -1.23e+03 & -3.24e+03 & -3.25e+03 & -1e+04 & -1e+04 & -1.11e+03 & -260 & -285 & -1.2e+03 & -788 \\ 
-2.97e+03 & -816 & -1.3e+03 & -115 & -1.11e+03 & -769 & -1.88e+03 & -1.76e+03 & -1.56e+03 & -2.08e+03 & -1.12e+04 & -1.12e+04 & -1.28e+03 & -200 & -214 & -1.19e+03 & -1.11e+03 \\ 
-5.05e+03 & -1.36e+03 & -2.32e+03 & -960 & -1.61e+03 & -1.88e+03 & -3.8e+03 & -2.55e+03 & -4.27e+03 & -4.52e+03 & -1.97e+04 & -1.96e+04 & -2.39e+03 & -455 & -521 & -1.93e+03 & -1.69e+03 \\ 
-3.35e+03 & -1e+03 & -1.55e+03 & -1.17e+03 & -1.23e+03 & -1.76e+03 & -2.55e+03 & -1.25e+03 & -746 & -1.27e+03 & -9.84e+03 & -9.78e+03 & -1.66e+03 & -261 & -361 & -1.31e+03 & -1.23e+03 \\ 
-8.21e+03 & -2.36e+03 & -3.64e+03 & 638 & -3.24e+03 & -1.56e+03 & -4.27e+03 & -746 & -1.15e+04 & -1.08e+04 & -4.5e+04 & -4.48e+04 & -3.62e+03 & -899 & -899 & -2.8e+03 & -2.47e+03 \\ 
-8.08e+03 & -2.35e+03 & -3.58e+03 & 198 & -3.25e+03 & -2.08e+03 & -4.52e+03 & -1.27e+03 & -1.08e+04 & -1.08e+04 & -4.17e+04 & -4.15e+04 & -3.55e+03 & -885 & -879 & -2.8e+03 & -2.47e+03 \\ 
-2.51e+04 & -7.03e+03 & -1.1e+04 & -2.5e+03 & -1e+04 & -1.12e+04 & -1.97e+04 & -9.84e+03 & -4.5e+04 & -4.17e+04 & -1.76e+05 & -1.75e+05 & -1.07e+04 & -4.79e+03 & -4.91e+03 & -5.77e+03 & -5.02e+03 \\ 
-2.49e+04 & -6.97e+03 & -1.1e+04 & -2.51e+03 & -1e+04 & -1.12e+04 & -1.96e+04 & -9.78e+03 & -4.48e+04 & -4.15e+04 & -1.75e+05 & -1.74e+05 & -1.05e+04 & -4.76e+03 & -4.87e+03 & -5.76e+03 & -4.97e+03 \\ 
-3.25e+03 & -960 & -1.37e+03 & -564 & -1.11e+03 & -1.28e+03 & -2.39e+03 & -1.66e+03 & -3.62e+03 & -3.55e+03 & -1.07e+04 & -1.05e+04 & -3.25e+03 & -348 & -338 & -1.23e+03 & -1.12e+03 \\ 
-799 & -199 & -335 & 3.64 & -260 & -200 & -455 & -261 & -899 & -885 & -4.79e+03 & -4.76e+03 & -348 & -799 & -0.909 & 0 & 0 \\ 
-719 & -215 & -327 & 83.7 & -285 & -214 & -521 & -361 & -899 & -879 & -4.91e+03 & -4.87e+03 & -338 & -0.909 & -719 & 0 & 0 \\ 
-2.8e+03 & -862 & -1.23e+03 & -1.01e+03 & -1.2e+03 & -1.19e+03 & -1.93e+03 & -1.31e+03 & -2.8e+03 & -2.8e+03 & -5.77e+03 & -5.76e+03 & -1.23e+03 & 0 & 0 & -2.8e+03 & 0 \\ 
-2.47e+03 & -722 & -1.13e+03 & -889 & -788 & -1.11e+03 & -1.69e+03 & -1.23e+03 & -2.47e+03 & -2.47e+03 & -5.02e+03 & -4.97e+03 & -1.12e+03 & 0 & 0 & 0 & -2.47e+03 \\ 

\hline
\end{tabular}
%\end{center}
\end{sideways}
}
\end{center}

\pagebreak

Now we compute results from numerical log likelihood maximization.

\begin{center}
\begin{tabular}{llll}
\hline 
  & Newton & Nelder-Mead & BFGS \\ 
\hline 
computation time & 1.2663 & 20.505 & 12.161 \\ 
loglikelihood & -5311.5 & -5323.6 & -5311.5 \\ 
constant & -6.0564 & -5.4357 & -6.0563 \\ 
i\_large\_loan & 0.86759 & 0.93991 & 0.86761 \\ 
i\_medium\_loan & 0.52736 & 0.5692 & 0.52737 \\ 
rate\_spread & 0.59558 & 0.59846 & 0.5956 \\ 
i\_refinance & 0.16339 & 0.0028961 & 0.1634 \\ 
age\_r & 0.87122 & 1.3247 & 0.87114 \\ 
cltv & -0.056624 & 0.0059927 & -0.056576 \\ 
dti & 0.21508 & 0.0018195 & 0.21509 \\ 
cu & 1.0079 & 0.94179 & 1.0078 \\ 
first\_mort\_r & 0.3356 & 0.36558 & 0.33563 \\ 
score\_0 & -0.28419 & -0.18642 & -0.28418 \\ 
score\_1 & 0.18943 & 0.0018143 & 0.18943 \\ 
i\_FHA & 0.75858 & 0.72328 & 0.75859 \\ 
i\_open\_year2 & 1.1527 & 1.0868 & 1.1527 \\ 
i\_open\_year3 & 0.77016 & 0.73289 & 0.77018 \\ 
i\_open\_year4 & 0.37934 & 0.3476 & 0.37939 \\ 
i\_open\_year5 & 0.24021 & 0.18982 & 0.24023 \\ 
\hline 
\end{tabular}
\end{center}

As we can see above, the computational time of the code I wrote was much lower than the two other Matlab functions, fminsearch (Nelder-Mead) and fminunc (BFGS). This is primarily because the function I wrote is particularly tailored to the application and is quite streamlined, and uses the analytical hessian and gradient. The Matlab functions have no such analytical derivatives so they have to use numerical approximations, which are time consuming to compute.

We can see that the Nelder-Mead results are closer to the Stata results, but actually result in a lower likelihood than the Newton or BFGS methods. The Newton and BFGS methods seem to converge to approximately the same local maximum. I was unable to find any other local maxima that resulted in higher likelihoods than the ones presented.

Computational time is resulted in units of seconds, and measures total runtime from the beginning to end of the estimation procedure.
\end{document}
