% !TEX TS-program = pdflatex
% !TEX encoding = UTF-8 Unicode

% This is a simple template for a LaTeX document using the "article" class.
% See "book", "report", "letter" for other types of document.

\documentclass[11pt]{article} % use larger type; default would be 10pt

\usepackage[utf8]{inputenc} % set input encoding (not needed with XeLaTeX)

%%% Examples of Article customizations
% These packages are optional, depending whether you want the features they provide.
% See the LaTeX Companion or other references for full information.

%%% PAGE DIMENSIONS
\usepackage{geometry} % to change the page dimensions
\geometry{a4paper} % or letterpaper (US) or a5paper or....
\geometry{margin=1in} % for example, change the margins to 2 inches all round
% \geometry{landscape} % set up the page for landscape
%   read geometry.pdf for detailed page layout information

\usepackage{graphicx} % support the \includegraphics command and options

% \usepackage[parfill]{parskip} % Activate to begin paragraphs with an empty line rather than an indent
\usepackage{amssymb}
\usepackage{amsmath}
%%% PACKAGES
\usepackage{booktabs} % for much better looking tables
\usepackage{array} % for better arrays (eg matrices) in maths
\usepackage{paralist} % very flexible & customisable lists (eg. enumerate/itemize, etc.)
\usepackage{verbatim} % adds environment for commenting out blocks of text & for better verbatim
\usepackage{subfig} % make it possible to include more than one captioned figure/table in a single float
% These packages are all incorporated in the memoir class to one degree or another...

%%% HEADERS & FOOTERS
\usepackage{fancyhdr} % This should be set AFTER setting up the page geometry
\pagestyle{fancy} % options: empty , plain , fancy
\renewcommand{\headrulewidth}{0pt} % customise the layout...
\lhead{}\chead{}\rhead{}
\lfoot{}\cfoot{\thepage}\rfoot{}

%%% SECTION TITLE APPEARANCE
\usepackage{sectsty}
\allsectionsfont{\sffamily\mdseries\upshape} % (See the fntguide.pdf for font help)
% (This matches ConTeXt defaults)

%%% ToC (table of contents) APPEARANCE
\usepackage[nottoc,notlof,notlot]{tocbibind} % Put the bibliography in the ToC
\usepackage[titles,subfigure]{tocloft} % Alter the style of the Table of Contents
\usepackage{bbm}
\usepackage{endnotes}
\usepackage{rotating}

\renewcommand{\cftsecfont}{\rmfamily\mdseries\upshape}
\renewcommand{\cftsecpagefont}{\rmfamily\mdseries\upshape} % No bold!
\DeclareMathOperator*{\argmax}{arg\,max}
\DeclareMathOperator*{\argmin}{arg\,min}

\usepackage{graphicx}
\graphicspath{ {./pings/} }

\newcount\colveccount
\newcommand*\colvec[1]{
        \global\colveccount#1
        \begin{pmatrix}
        \colvecnext
}
\def\colvecnext#1{
        #1
        \global\advance\colveccount-1
        \ifnum\colveccount>0
                \\
                \expandafter\colvecnext
        \else
                \end{pmatrix}
        \fi
}

\newcommand{\norm}[1]{\left\lVert#1\right\rVert}

\title{Corporate Problem Set 3}
\author{Michael B. Nattinger}

\begin{document}
\maketitle
\section{Question 1}
The problems our perfectly competitive lenders solve are:
\begin{align*}
&\max_{R_b} p_H^i R_b\\
&\text{s.t. } p_H^i R_b \geq p_L^i R_b + B\\
&\text{and } p_H^i (R^i-R_b^i) \geq I-A \\
\iff &\max_{R_b} p_H^i R_b\\
&\text{s.t. } \Delta p R_b \geq  B \\
&\text{and } p_H^i (R^i-R_b^i) \geq I-A
\end{align*}
where the first constraint is the incentive compatability constraint on the borrower and the second constraint is the lender participation constraint. The LHS of the lender participation constraint declines in $R_b$ so our optimal allocation must satisfy that constraint with equality. Hence, $p_H^I(R^i-R_b^i) = I-A \Rightarrow R_b^i = R^i - \frac{I-A}{p_H^i}$. We do now have to check to see if the borrower has incentive to deviate. Our borrower will not deviate so long as:

\begin{align*}
\Delta p R_b^i &\geq B \\
\iff \Delta p \left( R^i - \frac{I-A}{p_H^i}\right) &\geq B\\
\iff A&\geq p_H^i \frac{B}{\Delta p} - (p_H^iR^i - I) = \bar{A}^i(B).
\end{align*}

Comparing asset constraints, $\bar{A}^A(B) - \bar{A}^B(B) = \frac{B}{\Delta p}(p_H^A -p_H^B) >0$ so project $A$ is more prone to rationing. Because project $B$ is riskier, the expected payoff of shirking is lower so it is easier for the firm to be convinced to put in high effort. Thus, project $B$ is less prone to credit rationing.

\section{Question 2}
In this case, the problem setup becomes:
\begin{align*}
&\max_{R_b} p_H R_b \\
&\text{s.t. } \Delta p R_b \geq BI\\
&\text{and } p_H(R(I) - R_b)\geq I-A,
\end{align*}
where as before the first constraint is the incentive compatibility constraint of the borrower, and the second constraint is the participation constraint of the lender.

As before, the participation constraint must bind at the optimum:
\begin{align*}
R(I) - R_b &= \frac{I-A}{p_H}\\
\Rightarrow R_b &= R(I) - \frac{I-A}{p_H}.
\end{align*}

Plugging into our IC constraint:

\begin{align*}
R(I) - \frac{I-A}{p_H} &\geq \frac{BI}{\Delta p}\\
\Rightarrow A &\geq \left[\frac{p_H B}{\Delta p} + 1\right] I - p_H R(I)
\end{align*}

Noting that surplus maximizing $I$ will be set such that this equation will hold with equality, $$A =  \left[\frac{p_H B}{\Delta p} + 1\right] I - p_H R(I) $$.

Here we apply total differentiation:
\begin{align*}
dA &=  \left[\frac{p_H B}{\Delta p} + 1 -p_H R'(I)\right] dI\\
\Rightarrow \frac{dI}{dA} &= \frac{\Delta p}{p_H B}>0,
\end{align*}
where in the last line we have used the fact that $p_HR'(I) = 1$. The more cash on hand a firm has, the more it will invest in equilibrium.

Now we are asked about the shadow price. The lender's utility in equilibrium is $U=p_H R_b - I+A$. Differentiating with respect to $A$, 
\begin{align*}
\frac{\partial I}{\partial A} &= [p_H R'(I) - 1]\frac{\partial I}{\partial A} + 1\\
&=  1,
\end{align*}

where we again used our optimality condition, $p_HR'(I) = 1$. Thus, the shadow value of cash is constant and positive, and the lender's utility increases linearly in cash.

\section{Question 3}
\subsection{Part (i)}
If a firm buys Treasuries then $(p_H - T)\leq p_0$, i.e. they must buy enough treasuries such that they avoid liquidity problems at $t=1$. Thus, a firm purchases $T=p_H-p_0$ Treasuries.

For a firm to invest, two further conditions hold: the benefit exceeds the cose, and the cost is below pledgable income. These constraints are the following:
\begin{align*}
(1-\lambda)(p_0 - p_L) + \lambda(p_0 - p_H)&\geq (q-1)(p_H - p_0), \\
\lambda(p_1 - p_H)&\geq  (q-1)(p_H - p_0).
\end{align*}

These are the constraints we are asked to solve for.
\subsection{Part (ii)}

Now $T<p_H - p_0$. Either (2) or (3) must bind because, if not, then firms will buy $(p_H-p_0)$ treasuries. This exceeds supply by assumption.

Suppose (2) binds, so $(1-\lambda)(p_0 - p_L) + \lambda(p_0 - p_H)= (q-1)(p_H - p_0)$. Let $\lambda \approx 0$. Then, $(q-1)(p_H-p_0) = (p_L - p_0) - I + A>0$. But then if $\lambda(p_1 - p_H)\approx 0>  (q-1)(p_H - p_0)$ then $0>0$, a contradiction.

So (3) binds, $\lambda(p_1 - p_H) =  (q-1)(p_H - p_0) \Rightarrow (q-1) = \lambda \left(\frac{p_1 - p_H}{p_H - p_0}\right)$.

\subsection{Part (iii)}
Since the new asset does not hedge against the liquidity shock, its price is its expected payoff, since firms are risk neutral. $q' = (1-\lambda).$

\end{document}
